\documentclass[oneside, a4paper, onecolumn, 11pt]{article}

\input{format}

\usepackage[left=2.05cm,top=2.05cm,right=2.05cm]{geometry}
\setlength{\bibsep}{0.0pt}


%% To fix list things: 
\setitemize{noitemsep,topsep=0pt,parsep=0pt,partopsep=0pt,leftmargin=*}

\pagestyle{fancy}
%\renewcommand{\headrulewidth}{0pt}      %% Remove line at top

%\pagestyle{empty}
\fancyhf{}
%\lhead{{\it ERC-2018-CoG}}
%\lhead{{\it DEQUASARS: Part B1 }}
\lhead{{\it Ross}}
\chead{{\it }}
\rhead{Part B1}
\setcounter{page}{1}
\lfoot{{\it ERC-2018-CoG}}
\rfoot{{\it Extended Synopsis}}
\cfoot{{\it Page \thepage\ of 5}}
%\rfoot{{\it FP7-PEOPLE-2013-IIF}}

\newenvironment{itemize*}%
  {\begin{itemize}%
    \setlength{\itemsep}{0pt}%
    \setlength{\parskip}{0pt}}%
  {\end{itemize}}


\begin{document}


\smallskip
\smallskip
\noindent
{\bf{\textcolor{Cerulean}{a. Extended Synopsis}}} 
\vspace{6pt}

\noindent
%\Huge \huge \LARGE \Large \large \normalsize (default) \small \footnotesize \scriptsize \tiny
\large
{\bf{\textcolor{Cerulean}{Overview and Objectives}}}
\normalsize

\smallskip
\noindent
Current theories of galaxy formation and evolution strongly suggest that central, supermassive black holes (SMBHs) have a profound effect on the galaxies that they live in \citep[e.g., ][]{Vogelsberger2014Nature, Schaye2015, SomervilleDave2015, Dave2017}. This is not surprising since the potential energy associated with mass accretion onto a supermassive black hole is comparable to that generated via the nuclear fusion in the galaxy's stars \citep[see e.g. ][]{Fabian2012}. Thus when a galaxy goes through a ``quasar'' phase, where gas is supplied and accreted by the SMBH, there is ample energy to potentially impact the host galaxy 
%\todo{Just an example of a ToDo} 
and the surrounding intergalactic medium. 

\smallskip
\smallskip
\noindent
However, the interaction and the details of physical processes involved in how this energy escapes the inner most regions of the galaxy and then interacts with the gas, dust, stars and dark matter, is currently poorly understood, with current observational data giving more puzzles than clues on how to make progress. Further issues arise since startling new observations by the P.I.’s (Nicholas P. Ross; NPR) research team \citep{MacLeod2016, Ross2018} show that {\it quasars vary significantly on timescales of weeks to months}, whereas the accretion disks (that are thought to supply the `fuel' for the quasar) should take thousands of years to change their optical emission \citep[see e.g. ][]{Lawrence2018}. 
%\todo{Just a second example of a ToDo, cause I like the colours.} 
Thus, it is unclear to what level we have an understanding of an physical phenomena prevalent in many astrophysical systems: the accretion disk. 

\smallskip
\smallskip
\noindent
The field of observational extragalactic astrophysics is poised for a fundamental and rapid change. Starting in late 2019, a fleet of new telescopes, instruments and missions will be commissioned, start data taking, and over the next few years will leap-frog the quality and quantity of data we have available today. These surveys and missions include: the fifth incarnation of the Sloan Digital Sky Survey (SDSS-V\footnote{\href{www.sdss.org/future/}{{\tt www.sdss.org/future/}}}); the Large Synoptic Survey Telescope (LSST\footnote{\href{lsst.org}{{\tt lsst.org}}}); the Dark Energy Spectroscopic Instrument (DESI\footnote{\href{desi.lbl.gov}{{\tt desi.lbl.gov}}}) survey; the 4-metre Multi-Object Spectroscopic Telescope (4MOST\footnote{\href{4most.eu}{{\tt 4most.eu}}}) survey, and the ESA {\it Euclid} mission\footnote{\href{sci.esa.int/euclid/}{{\tt sci.esa.int/euclid/}}}. Even more imminent is the launch of the {\it James Webb Space Telescope} (JWST\footnote{\href{jwst.stsci.edu}{{\tt jwst.stsci.edu}}}).

\smallskip
\smallskip
\noindent
This proposal has two broad and well-posed goals. First, we aim to elucidate in detail {\bf how the energy directly associated with a supermassive black holes impacts the universal galaxy population.} We will gain a deep understanding into the physical mechanisms related to central engine black holes; their accretion disk physics, their dynamics on both human and galactic timescales and the role they might play in forming and regulating the galaxy population. 
%%
Second, we anticipate {\bf to discover brand new extragalactic phenomena.}  
By tapping into the massive and raw discovery space that the new experiments will open up, there is the highly likely outcome of discovering something ``brand new''  \citep{Ivezic2008, LSST_ScienceBook}. The LSST will deliver a dataset so spectacularly different both in sky and time-sampling coverage that the Universe would have to be an exceptionally boring place to not have brand new astronomical objects and astrophysical phenomena waiting to be discovered.

\smallskip
\smallskip
\noindent

\smallskip
\smallskip
\noindent
Our major science objectives are:
\begin{enumerate}
% \item Discover 
%We will investigate what the observed rapid changes tell us about the SMBH and accretion discs, and does ``quasar feedback'' regulate galaxy formation? {\it Ultimately, 
\item Characterize the variable extragalactic universe and quasar population. 
\item Establish the energy transport mechanisms associated with the``quasar phase''. 
%(how does the energy get out of the central regions and dispersed into the galaxy at large?)

%\item Discover if there is a ``missing link'' between the activity from a quasar on sub-parsec scales that impacts on the galaxy-wide kiloparsec scales?} These are among the most prescient astrophysical issues of our time, and where major breakthroughs are expected.

\item Explain the relation between accretion rate, black hole mass build-up with observed light curve and spectral properties. 
%Discover how stars lose the vast majority of their angular momentum, which happens during the accretion phase.
%\item Explain the observed rotation-activity relationship and saturation in terms of the evolution of magnetic properties and coronal physics.
%\item Characterize coronal heating and mass loss across the full range of mass and age.
%\item Explain the Skumanich (1972) relationship and distributions of spin rates observed in young clusters and old field stars.
\item Develop and link theoretical accretion and galaxy formation models for a fully holistic theory of active galaxies. 
\item Discover new extragalactic variable objects. 
\end{enumerate}


\smallskip
\smallskip
\noindent
We will achieve this by leveraging several of the new, large-scale surveys that are coming online in the next few years. These critical observations are made by exploiting the large imaging and spectroscopic datasets that will be available from the SDSS-V, DESI, 4MOST, LSST and ESA {\it Euclid}. {\it Crucially, although these  projects individually will deliver new state-of-the-art datasets, it is our project that will be the first to break down the associated data  silos and combine these data in order to go beyond the state-of-the-art.}

\medskip
\medskip
\noindent
\large
{\bf{\textcolor{Cerulean}{1. Current State of the Art.}}}
\normalsize

\smallskip
\noindent
The current state-of-the-art data samples have either $\approx$10$^{6}$ quasars with one spectral epoch, or only a few objects with repeat photometric data, i.e. light-curve information and the accompanying repeat spectra (e.g., see Figure~\ref{fig:J110057}).  NPR has been involved in the production of both of these two types of samples \citep{Paris2017, MacLeod2016, Ross2018}. We plan to collate datasets so that the 10$^{6}$ sample have high-fidelity light-curves and ample repeat spectroscopy, 
and in doing so will kick start the new field of Variable Extragalactic Astrophysics. 


\begin{figure}[h]
  \begin{center}
    \hspace{-0.5cm}
    \includegraphics[height=6.25cm,width=17.2cm]
    {figures/J110057_LC_Spectra_20171024.pdf}
    \vspace{-10pt}
    \caption{%\small      
      \footnotesize 
      % \scriptsize
      % \tiny
      {\it (Left:)} The optical and infrared light-curve for the redshift $z=0.378$ quasar 
      J1100-0053 (Ross et al. 2018). 
      Note the fall in the infrared, whereas there is a decrease, but 
      then recovery in the optical. 
      {\it (Right:)} 
      Three epochs of spectra for J1100-0053. 
      The spectacular downturn in the blue for the 2010 spectrum 
      indicates a dramatic change in the accretion disk.
    }
  \vspace{-16pt}
 \label{fig:J110057}
\end{center}
\end{figure}


\smallskip
\smallskip
\noindent
During its initial phases of operation the Sloan Digital Sky
Survey (SDSS) obtained spectra of 1 million galaxies in the local
Universe. This dataset has become the {\it de facto} standard for
understanding the present day galaxy population, and sets the boundary
conditions for all theoretical comparisons. The paradigm changing
success of the SDSS was due to it having 1,000,000 objects {\it with very 
high signal-to-noise photometry and spectra}, enabling
multivariate analysis that is required for galaxy
astrophysics investigations.  {\it We desire the sample size and revolutionary
understanding with the new temporal dimension of the quasar population across all redshifts, as the SDSS had with the
low-redshift $z\sim0.1$ galaxy population.}  Our proposal takes quasar astrophysics
into the 2020s, going from single objects samples, to surveys and
samples of millions of objects, with massive spectroscopic monitoring 
giving access to the time-domain and leveraging these very large scale next
generation missions, telescopes and their datasets.




\begin{figure}[h]
  \begin{center}
   \hspace{-0.5cm}
%   trim=l b r t
    \includegraphics[width=16.0cm] %, trim={0.05cm 0 0.05cm 0},clip]
    {figures/Timelines_and_Facilites.pdf}
    \vspace{-10pt}
   \caption{Facilites, Timelines and Priorities. Withe SDSS and DESI in the Northern Hemisphere and 
4MOST, LSST in the South, we have full celestial sphere coverage.}
  \vspace{-12pt}
 \label{fig:Keynote_facilites}
\end{center}
\end{figure}

\smallskip
\smallskip
\noindent
{\it The timing for this proposal could not be better.}  The first of
the data ``firehoses'' turns on in late 2019, with the full datastream
from our key sources fully online by mid-2022.  As such, we have the
time to mature our analysis techniques, and then be in the ideal
position to take advantage of the initial data releases of all these
new projects. 
%%
Prompt ERC Consolidator level-support is also imperative since 
%Edinburgh is substantially ramping up its involvement in e.g. LSST this year, and 
final survey design and optimization trade-off studies are being made 
e.g. for DESI, SDSS-V and LSST over the next $\sim$12-18 months. Having the ability to influence these decisions to our science goals would be very powerful. Also, having the science teams and various collaborations know that the PI is embarking on this program would help attract the best personel for the PDRAs, who would be guaranteed ``First Light'' data and science. 

\smallskip
\smallskip
\noindent
The importance of this branch of astrophysics is already well
establish in Europe and is a priority for the next two decades. This
is demonstrated by noting that one of the two primary mission goals
for the Advanced Telescope for High-ENergy Astrophysics (ATHENA) mission 
is answering the question ``How do black holes grow and shape the
Universe?''.  ATHENA is ESA's second L-class flagship mission, due for
launch in 2028.


\smallskip
\smallskip
\noindent
{\it The scope and remit of an ERC Consolidator grant will allow us to
combine these data products in a manner that will not only establish
the new state-of-the-art in variable extragalactic astrophysics, but it 
will also establish and kickstart the new field of variable extragalactic
astrophysics itself.}




\medskip
\medskip
\noindent
\large
{\bf{\textcolor{Cerulean}{2. Methodology}}}
\normalsize

\noindent
Our proposal contains eight work packages that fall into three broad and 
complementary categories: observational studies of large numbers
(millons) of objects; high-risk, very high-reward observational
studies of a small number (10s) of objects; theoretical modeling
investigations. Several of these 
Table 1 summarises our overall WP plan. 
Risks and mitigation strategies are present for each WP. 
Key Deliverables are also given for each WP. 

\smallskip
\smallskip
\noindent
We define three PDRAs, ``PDRA1'', ``PDRA 2'', ``PDRA 3'' and one PhD student, ``PhD1''.  
The skill set of PDRA1 would include development of the underlying tool, and 
techniques necessary to extract meaning from large and/or complex data sets. 
%PDRA1 would hold a PhD in astronomy, astrophysics or computer science. 
The skill sets of PDRA2 would include expertise in time series analysis and 
primarily with optical data, but potentially also in other wavebands. 
%PDRA2 would likely hold a PhD in astronomy/astrophysics.
The skill set of PDRA3 would include experience with fluid mechanics modelling and/or 
large computer simulations. 
%PDRA3 would likely hold a PhD in astronomy/astrophysics. mathematics or computer science. 
PhD1 would have a Masters or a strong 4-year undergraduate degree in Physics or 
Mathematics with evidence of research-level project work.  


\smallskip
\smallskip
\noindent
\textbf{\textsc{WP1: Build an Event Broker:}} 
The LSST will deliver three levels of data products and
services and being in the U.K. gives us access to all three. 
%The ``Level 3'' Data Products are the user-created data product services that will enable our science case. 
 In order to utilize the LSST data for our science goals we will need to build an {\it event
broker}, an intermediary program module that interacts with primarily
the ``Level 3'' data products from the LSST.
%% \noindent
The goal of this WP is to build an Event Broker. 
{\bf WP1 is low-risk, high-reward.} 
The heavy-industry computing infrastructure is already being supplied
by the LSST Data Access Center and our task will be to build software
in a timely and robust manner. With PDRA1 and commitment from the P.I., (NPR) along with the algorithm
resources and key personel, e.g. Prof Andy Lawrence (AL) and Prof. Bob
Mann (RGM), at the Royal Observatory, Edinburgh, there is no element
of this which can be deemed high-risk.  {\bf Key Deliverables:} An
open-source, well-documented software package that can interact with
and return data from the LSST Data Access Center.


\smallskip
\smallskip
\noindent
\textbf{\textsc{WP2: Quasar Catalogue Generation:}} 
Building the quasar corpus and cataloguing the observational data will
be a large, but vital step in beginning to pursue our science
goals. This catalogue will be the glue that binds the observational
projects together and will have not only the data, but moreover the
metadata to enable the other WPs.
% \noindent
{\bf WP2 is low-risk, high-reward.}
The goal of this WP is to construct a quasar catalogue.
This is a full WP, and given the P.I.s (NPR) experience at this
specific task, plus the effort level of PDRA2 this WP is low-risk. 
{\bf Key Deliverables:} An open-source, well-documented science-enabling compendium that
will be the state-of-the-art quasar dataset for the 2020s. 


\smallskip
\smallskip
\noindent
\textbf{\textsc{WP3: Quasar Demographic studies:}} 
Another major scientific output that will originate from the quasar
corpus catalogue generation will be the study of the Quasar
Demographics from our datastreams. This is linked to, but different from, WP4 in that
these investigations wont necessarily be tied to the time-domain
aspect of our catalogue, but will be the crucial baseline that we, and
other researchers in general, will use to compare to the time-dependent
discoveries. Luminosity function, clustering and higher-order
statistics will be made in order to precisely determine the census of
AGN, their environments, their host galaxy preferences and their
evolution. All these are vital observational tests for galaxy
formation models and theory (see WP6 below).
The goal of this WP is to make the key observational tests that have
to be explained by any viable galaxy formation theory.  
%%
{\bf WP3 is low-risk, high-reward.}
This particular work package will be broken down into smaller projects
and the analyses that we envisaged will be broken down into smaller
projects with PDRA 1 and PDRA 2, as
well as the P.I. (NPR) and a PhD1 will be directed
towards this. There is no element of this which can be deemed
high-risk.
%%
{\bf Key Deliverables:} A suite of new, beyond-the-state-of-the-art 
quasar demographic measurements which are the input 
and boundary conditions for theoretical models. 


\smallskip
\smallskip
\noindent
\textbf{\textsc{WP4: Light-Curve and Spectral Analyses:}} 
Following on from the quasar corpus catalogue generation, one key
science output will be the full and detailed light-curve and spectral
analyses of the said catalogue. This will result in the discovery of
light-curve trends with quasar type, new methods to measure black hole
mass and the key science goal to see which quasars have become
``changing-look'' objects. This WP will have a data science/machine learning 
aspect.
%% 
The goal of this WP is to elucidate the physical processes that drive quasar variability.
The full Light-Curve and Spectral
Analyses that we envisaged will be a significant amount of work,
leading to significant high-reward science. 
{\bf WP4 is medium-risk, high-reward.}
Similar to WP3, this particular work
package will be broken down into small projects and 
PDRA1, PDRA2, as well as the P.I. (NPR) and PhD1 will be directed towards this. This level of investigation 
is highly novel, though we envisage no major barriers outside of our control to
 achieving our science goals. As such, we deem this medium-risk.
{\bf Key Deliverables:} Measurements, for the first time of how the 
light-curves and spectra of quasars depend on key physical 
quasar properties e.g. $M_{\rm black hole}$, (bolometric) luminosity, $\lambda = \log L / L_{\rm Edd}$, spin etc. 
These measurements will allow us to make {\it direct} comparisons to accretion disk models. 


\smallskip
\smallskip
\noindent
\textbf{\textsc{WP5: Accretion Disk Simulation:}} 
New accretion models are needed to fully explain the observational
data of ``changing look'' quasars that we have examples of today (see
e.g. Ross et al. 2018). New radiation MHD codes begin to explain the
observations here, but further development is needed to gain the
desired deep understanding. 
The goal of WP5 is to develop new accretion disk simulations that
explain our observational results.  This will be the lead WP for 
PDRA3 and a low level of NPRs time. 
%%
{\bf WP5 is lower-risk, very high-reward.} We
classify WP5 not as fully `low-risk', since we envisage some ramp-up
time to get our theoretical simulations to the level that will be required by 
our beyond-the-state-of-the-art dataset. However, we mitigate this risk
by invoking the collaboration with an accretion disk theorist
Prof. Ken Rice (WKMR) who is the Personal Chair of Computational
Astrophysics in the School of Physics and Astronomy here at the
University of Edinburgh. NPR and WKMR and PDRA3 would thus collaborate 
on this WP. 
%%
{\bf Key Deliverables:} New accretion disk models and theory that explain the 
light curve data of our beyond-the-state-of-the-art dataset. 


\smallskip
\smallskip
\noindent
\textbf{\textsc{WP6: AGN Feedback Simulations:}} 
Cosmological-scale hydrodynamic simulations are now coming online. 
While we do not seek to lead or generate new versions of these, we do 
envisaged using their outputs in order to `benchmark' our observational 
demographic work. 
% \noindent
All the data from these simulations is already in place today, though no one 
has embarked on doing any of the `heavy-lifting' and comparisons we will 
have the observational results for. Professor Romeel Dave (RSD) who is a Chair of Physics 
in the Institute for Astronomy will be a key collaborator here. 
NPR and RSD and PDRA3 and/or PDRA2 would thus collaborate on this WP. 
We thus classify {\bf WP6 as low-risk, high-reward.}
%%
{\bf Key Deliverables:} New galaxy evolution models, describing the hydrodynamics 
involved on galactic scales, but related to the quasar central engine. 


\smallskip
\smallskip
\noindent
\textbf{\textsc{WP7: Observations of Quasars by the James Webb Space Telescope:}} 
What are the star-formation properties of mid-infrared luminous quasars at the peak of quasar activity? 
We aim to answer this by looking for the presence of polycyclic aromatic hydrocarbon (PAH) spectral features 
in $z \approx 2.5$ infrared bright quasars with the James Webb Space Telescope (JWST). 
% \noindent
{\bf WP7 is high risk, high-reward.}
This is an ideal investigation for the JWST, but we classify this as
`high-risk' since this is the one telescope/survey/mission where we
would have to bid and apply for the telescope time and are not guaranteed
the data. We mitigate the risk here by saying that this will be the
one project the P.I. (NPR) would directly lead, does not impact in any direct way any of
the other WPs and would lead to very-high gain science.
{\bf Key Deliverables:} State-of-the-art data and data products from the JWST, with the observational evidence and physical interpretation of how ``quasar feedback'' regulates galaxy formation in high-redshift quasars. 


\smallskip
\smallskip
\noindent
\textbf{\textsc{WP8: New Object Discovery:}} 
The LSST will scan the sky repeatedly, enabling it, and us, to both
discover new, distant transient events and to study variable objects
throughout our universe. The most interesting science to come may well
be the discovery of new classes of objects.
%%
{\bf WP8 is medium-to-high risk, exceptionally high-reward.}
We class this as medium-to-high risk, since it is tricky to class a WP
with essentially unknown discovery potential as fully `low-risk'.
Suffice to say, this would be exceptionally high-reward
{\bf Key Deliverables:} The discovery of new classes of astronomical objects. 


\begin{figure}[h]
  \begin{center}
   \hspace{-0.5cm}
%   trim=l b r t
    \includegraphics[width=16.0cm] %, trim={0.05cm 0 0.05cm 0},clip]
    {figures/workplan.pdf}
    \vspace{-10pt}
  \caption{An overview of our WPs:  the personel attached to each WP and a guide to 
their start and duration is shown. The connection between the WPs is also shown, be it generally one-way 
(square starting points) or an iteration (both ends pointed). As such, the flow arrows are guides 
and not strictly specifying exact timescales. As given be the shadings, WP1, 2, 3 and 4 are 
observational studies of large numbers of objects; WP5 and 6 are theoretical modeling
investigations and WP7 and 8 are high-risk, very high-reward observational
studies of a small number of objects.}
  \vspace{-12pt}
 \label{fig:Keynote_facilites}
\end{center}
\end{figure}


\smallskip
\smallskip
\noindent
\large
{\bf{\textcolor{Cerulean}{3. Resources,  Survey `buy-in' and Budget}}}
\normalsize

\smallskip
\smallskip
\noindent
\textbf{\textsc{Personnel:}} 
We request the resources and support for 100\% of the time and effort
for the P.I. 
%% 
We request the resources and support for 3 Postdoctoral
Research Associates (PDRAs), for a total of 10 PDRA year
equivalents. This will be broken down as a three year term for PDRA
1, a three year term for PDRA2 and a 3+1 year term for PDRA
3.  We request the resources and support for 1 UK/EU PhD
studentship.


\smallskip
\smallskip
\noindent
\textbf{\textsc{Survey Buy-in:}} 
We request support for the ``buy-in'' to two of the new surveys,
SDSS-V and DESI. The costs here are \$230,000 (\euro184,100) for
SDSS-V and \$250,000 (\euro200,100) for DESI.  We ask this support to
come from the ``additional funds that can be made available to cover
access to large facilities.''  We specifically request access to these
funds as it gives our project access to telescopes and data in the North 
and Southern Hemispheres (for complete coverage of
the celestial sphere) and delivers the crucial early spectroscopy that will be
vital to train, test and build our data science and machine learning
codes and algorithms. 
We emphasise that the science return is `exponential' (rather than `linearly') 
dependent on the breadth of data available. Can one call this multimessenger 
and heralds a brand new regime of ``several-survey'' or ``multi-mission'' astronomy. 
%%
{\it Buy-in here would place the P.I. and the
University of Edinburgh as the only group and place in the world to be involved
in SDSS-V, DESI, 4MOST, LSST and ESA {\it Euclid} and JWST}.

\smallskip
\smallskip
\noindent
\textbf{\textsc{Computing Requirements:}} 
With the availability of the facilities at an institute (e.g. IfA
Cullen), university
(e.g. \href{https://www.ed.ac.uk/information-services/research-support/research-computing/ecdf}{``Edinburgh
Compute and Data Facility''}) and at a national
{\href{https://www.hartree.stfc.ac.uk/Pages/home.aspx}{(The Hartree
Centre)} level, the rate limiting factor will be how quickly and
efficiently we can deploy our codes, and analysis, i.e. person-power.
%The rate limiting factor, in the vast majority of endevours, is (a) data access and (b) development.


\smallskip
\smallskip
\noindent
\textbf{\textsc{Travel:}} 
We request support for travel for all 5 members of the group,
including repeat medium-term (i.e., few weeks) travel to the US and
ESO Chile to work with key collaborators at critical timings of the
First Light for the new telescopes.




\newpage
%%\thispagestyle{empty}
\fancyhf{}
%\lhead{{\it ERC-2018-CoG}}
%\lhead{{\it DEQUASARS: Part B1 }}
\lhead{{\it Ross}}
\chead{{\it }}
\rhead{Part B1}
\setcounter{page}{1}
\lfoot{{\it ERC-2018-CoG}}
\rfoot{{\it Extended Synopsis}}
\cfoot{{\it Page \thepage\ of 5}}

\bibliographystyle{plainnat}
\bibliography{/cos_pc19a_npr/LaTeX/tester_mnras}

%\input{references}


\end{document}

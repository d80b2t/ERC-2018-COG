%\documentclass[11pt,epsf]{article}
\documentclass[oneside, a4paper, onecolumn, 11pt]{article}

\usepackage{graphicx, amssymb, multicol, amsmath}
\usepackage{fancyhdr, hyperref, sidecap}
\usepackage[left=2.05cm,top=2.05cm,bottom=1.55cm,right=2.05cm]{geometry}
\usepackage[utf8]{inputenc}
\usepackage{natbib}	        %%  bibliography style
\setlength{\bibsep}{0.0pt}
\usepackage{eurosym}
\usepackage{enumitem}
\usepackage{nopageno}
\usepackage{fancyhdr}
\usepackage[usenames,dvipsnames,svgnames,table]{xcolor}

\usepackage{amsmath, amssymb}
\usepackage{booktabs, bm}           %%  bold math
\usepackage{cancel}
\usepackage{dcolumn}  %%  Align table columns on decimal point
\usepackage{epsfig, epsf, eurosym, enumitem}
\usepackage{fancyhdr}
\usepackage[T1]{fontenc}
\usepackage[para]{footmisc}
\usepackage{graphicx }
%\usepackage{lscape}
\usepackage{hyperref,ifthen}
\usepackage{mathptmx, multicol}
\usepackage[authoryear, round]{natbib}
\usepackage{nopageno}
\usepackage{subfigure}
\usepackage{verbatim}
\usepackage{threeparttable}
\usepackage[usenames,dvipsnames]{xcolor}
\usepackage{tcolorbox}
\usepackage{tabularx}
\usepackage{array}
\usepackage{colortbl}
\usepackage{framed}
\usepackage{todonotes}



%%%%%%%%%%%%%%%%%%%%%%%%%%%%%%%%%%%%%%%%%%%
%       define Journal abbreviations      %
%%%%%%%%%%%%%%%%%%%%%%%%%%%%%%%%%%%%%%%%%%%
\def\nat{Nat} \def\apjl{ApJ~Lett.} \def\apj{ApJ}
\def\apjs{ApJS} \def\aj{AJ} \def\mnras{MNRAS}
\def\prd{Phys.~Rev.~D} \def\prl{Phys.~Rev.~Lett.}
\def\plb{Phys.~Lett.~B} \def\jhep{JHEP}
\def\npbps{NUC.~Phys.~B~Proc.~Suppl.} \def\prep{Phys.~Rep.}
\def\pasp{PASP} \def\aap{Astron.~\&~Astrophys.} \def\araa{ARA\&A}
\def\jcap{\ref@jnl{J. Cosmology Astropart. Phys.}} 
\def\nar{New~A.R.} \def\aapr{A\&ARv}

\newcommand{\preep}[1]{{\tt #1} }

%%%%%%%%%%%%%%%%%%%%%%%%%%%%%%%%%%%%%%%%%%%%%%%%%%%%%
%              define symbols                       %
%%%%%%%%%%%%%%%%%%%%%%%%%%%%%%%%%%%%%%%%%%%%%%%%%%%%%
\def \Mpc {~{\rm Mpc} }
\def \Om {\Omega_0}
\def \Omb {\Omega_{\rm b}}
\def \Omcdm {\Omega_{\rm CDM}}
\def \Omlam {\Omega_{\Lambda}}
\def \Omm {\Omega_{\rm m}}
\def \ho {H_0}
\def \qo {q_0}
\def \lo {\lambda_0}
\def \kms {{\rm ~km~s}^{-1}}
\def \kmsmpc {{\rm ~km~s}^{-1}~{\rm Mpc}^{-1}}
\def \hmpc{~\;h^{-1}~{\rm Mpc}} 
\def \hkpc{\;h^{-1}{\rm kpc}} 
\def \hmpcb{h^{-1}{\rm Mpc}}
\def \dif {{\rm d}}
\def \mlim {m_{\rm l}}
\def \bj {b_{\rm J}}
\def \mb {M_{\rm b_{\rm J}}}
\def \mg {M_{\rm g}}
\def \mi {M_{\rm i}}
\def \qso {_{\rm QSO}}
\def \lrg {_{\rm LRG}}
\def \gal {_{\rm gal}}
\def \xibar {\bar{\xi}}
\def \xis{\xi(s)}
\def \xisp{\xi(\sigma, \pi)}
\def \Xisig{\Xi(\sigma)}
\def \xir{\xi(r)}
\def \max {_{\rm max}}
\def \gsim { \lower .75ex \hbox{$\sim$} \llap{\raise .27ex \hbox{$>$}} }
\def \lsim { \lower .75ex \hbox{$\sim$} \llap{\raise .27ex \hbox{$<$}} }
\def \deg {^{\circ}}
%\def \sqdeg {\rm deg^{-2}}
\def \deltac {\delta_{\rm c}}
\def \mmin {M_{\rm min}}
\def \mbh  {M_{\rm BH}}
\def \mdh  {M_{\rm DH}}
\def \msun {M_{\odot}}
\def \z {_{\rm z}}
\def \edd {_{\rm Edd}}
\def \lin {_{\rm lin}}
\def \nonlin {_{\rm non-lin}}
\def \wrms {\langle w_{\rm z}^2\rangle^{1/2}}
\def \dc {\delta_{\rm c}}
\def \wp {w_{p}(\sigma)}
\def \PwrSp {\mathcal{P}(k)}
\def \DelSq {$\Delta^{2}(k)$}
\def \WMAP {{\it WMAP \,}}
\def \cobe {{\it COBE }}
\def \COBE {{\it COBE \;}}
\def \HST  {{\it HST \,\,}}
\def \Spitzer  {{\it Spitzer \,}}
\def \ATLAS {VST-AA$\Omega$ {\it ATLAS} }
\def \BEST   {{\tt best} }
\def \TARGET {{\tt target} }
\def \TQSO   {{\tt TARGET\_QSO}}
\def \HIZ    {{\tt TARGET\_HIZ}}
\def \FIRST  {{\tt TARGET\_FIRST}}
\def \zc {z_{\rm c}}
\def \zcz {z_{\rm c,0}}


\newcommand{\sqdeg}{deg$^{-2}$}
\newcommand{\lya}{Ly$\alpha$\ }
%\newcommand{\lya}{Ly\,$\alpha$\ }
\newcommand{\lyaf}{Ly\,$\alpha$\ forest}
%\newcommand{\eg}{e.g.~}
%\newcommand{\etal}{et~al.~}
\newcommand{\cii}{C\,{\sc ii}\ }
\newcommand{\ciii}{C\,{\sc iii}]\ }
\newcommand{\civ}{C\,{\sc iv}\ }
\newcommand{\SiIV}{Si\,{\sc iv}\ }
\newcommand{\mgii}{Mg\,{\sc ii}\ }
\newcommand{\feii}{Fe\,{\sc ii}\ }
\newcommand{\feiii}{Fe\,{\sc iii}\ }
\newcommand{\caii}{Ca\,{\sc ii}\ }
\newcommand{\halpha}{H\,$\alpha$\ }
\newcommand{\hbeta}{H\,$\beta$\ }
\newcommand{\oi}{[O\,{\sc i}]\ }
\newcommand{\oii}{[O\,{\sc ii}]\ }
\newcommand{\oiii}{[O\,{\sc iii}]\ }
\newcommand{\heii}{[He\,{\sc ii}]\ }
\newcommand{\nii}{N\,{\sc ii}\ }
\newcommand{\nv}{N\,{\sc v}\ }

%% From:: /cos_pc19a_npr/LaTeX/proposals/JWST/JWST_ERS/Proposal/lines.tex
%%  
\newcommand{\imw}{$i$--$W3$}
\newcommand{\imwf}{$i$--$W4$}
\newcommand{\rmwf}{$r$--$W4$}
\newcommand{\imwt}{$i$--$W2$}
\newcommand{\wtmwf}{$W3$--$W4$}
%\newcommand{\kms}{km s$^{-1}$}
\newcommand{\cmN}{cm$^{-2}$}
\newcommand{\cmn}{cm$^{-3}$}
%\newcommand{\msun}{M$_{\odot}$}
\newcommand{\lsun}{L$_{\odot}$}
\newcommand{\lam}{$\lambda$}
\newcommand{\mum}{$\mu$m}
\newcommand{\ebv}{$E(B$$-$$V)$}
%\newcommand{\heii}{\mbox{He\,{\sc ii}}}
\newcommand{\cv}{\mbox{C\,{\sc v}}}
%\newcommand{\civ}{\mbox{C\,{\sc iv}}}
%\newcommand{\ciii}{\mbox{C\,{\sc iii}}}
%\newcommand{\cii}{\mbox{C\,{\sc ii}}}
%\newcommand{\nv}{\mbox{N\,{\sc v}}}
\newcommand{\niv}{\mbox{N\,{\sc iv}}}
\newcommand{\niii}{\mbox{N\,{\sc iii}}}
%\newcommand{\oi}{\mbox{O\,{\sc i}}}
%\newcommand{\oii}{\mbox{O\,{\sc ii}}}
%\newcommand{\oiii}{\mbox{[O\,{\sc iii}]}}
\newcommand{\oiv}{\mbox{O\,{\sc iv}}}
\newcommand{\ov}{\mbox{O\,{\sc v}}}
\newcommand{\ovi}{\mbox{O\,{\sc vi}}}
\newcommand{\ovii}{\mbox{O\,{\sc vii}}}

%\newcommand{\feii}{\mbox{Fe\,{\sc ii}}}
%\newcommand{\feiii}{\mbox{Fe\,{\sc iii}}}
%\newcommand{\mgii}{\mbox{Mg\,{\sc ii}}}
\newcommand{\neii}{[Ne\,{\sc ii}]\ }
\newcommand{\neiii}{[Ne\,{\sc ii}]\ }
\newcommand{\nev}{Ne\,{\sc v}\ }
\newcommand{\nevi}{[Ne\,{\sc vi}]\ }
\newcommand{\neviii}{\mbox{Ne\,{\sc viii}}}
\newcommand{\aliii}{\mbox{Al\,{\sc iii}}}
\newcommand{\siii}{\mbox{Si\,{\sc ii}}}
\newcommand{\siiii}{\mbox{Si\,{\sc iii}}}
\newcommand{\siiv}{\mbox{Si\,{\sc iv}}}
%\newcommand{\lya}{\mbox{Ly$\alpha$}}
%\newcommand{\lyb}{\mbox{Ly$\beta$}}
\newcommand{\hi}{\mbox{H\,{\sc i}}}
\newcommand{\snine}{\mbox{[S\,{\sc ix}]}}
\newcommand{\sivi}{\mbox{[Si\,{\sc vi}]}}
\newcommand{\sivii}{\mbox[{Si\,{\sc vii}]}}
\newcommand{\siix}{\mbox{[Si\,{\sc ix}]}}
\newcommand{\six}{\mbox{[Si\,{\sc x}]}}
\newcommand{\sixi}{\mbox{[Si\,{\sc xi}]}}
\newcommand{\caviii}{\mbox{[Ca\,{\sc viii}]}}
\newcommand{\arii}{\mbox{[Ar\,{\sc ii}]}}

%%[Ar II] 6.97
%% [S IX] 1.252 μm 328 
% [Si X] 1.430 μm 351 
% [Si XI] 1.932 μm 401 
% [Si VI] 1.962 μm 167 
% [Ca VIII] 2.321 μm 128 
% [Si VII] 2.483 μm 205 
% [Si IX] 3.935 μm 303
% [Ar II] 6.97


%\snine\ at 1.252$\mu$m, \six\ at 1.430$\mu$m, \sixi\ at 1.932$\mu$m, \sivi\ at
%1.962$\mu$m, \caviii\ at 2.321$\mu$m, \sivi\ at 2.483$\mu$m \siix\ at
%3.935$\mu$m and \arii\ at 6.97$\mu$m. 
%%
%% such as [Ne ii]12.8 μm, [Ne v]14.3 μm, [Ne iii]15.5 μm, [S iii]18.7 μm and 33.48 μm, [O iv]25.89 μm and [Si ii]34.8 μm (e.g
%%
%% MIR emission lines like [NeII] and [NeV] are ..
%%
%% Also,  arXiv:astro-ph/0003457v1 
%% [NeV] 14.32um & 24.32um and [NeVI] 7.65um imply an A(V)>160 towards the NLR...
%% [NeIII]15.56um/[NeII]12.81um
%%
%% [Ne V] 14.3, 24.2 μm 97.
%% [Ne II] 12.8 μm
%% [OIV] 26μm
%%


\tcbuselibrary{skins}
\newcolumntype{Y}{>{\raggedleft\arraybackslash}X}

\tcbset{tab1/.style={enhanced,fonttitle=\bfseries,fontupper=\normalsize\sffamily,
colback=yellow!10!white,colframe=red!50!black,colbacktitle=Cerulean!40!white,
coltitle=black,center title}}

%% To fix list things: 
\setitemize{noitemsep,topsep=0pt,parsep=0pt,partopsep=0pt,leftmargin=*}
\renewcommand{\labelitemi}{\tiny$\blacksquare$}

\pagestyle{fancy}
\renewcommand{\headrulewidth}{0pt}  %% Remove line at top

%\pagestyle{empty}
\fancyhf{}
%\lhead{{\it ERC-2018-CoG}}
%\lhead{{\it DEQUASARS: Part B1 }}
\lhead{{\it Ross}}
\chead{{\it MIQSOs}}
\rhead{Part B1}
\setcounter{page}{1}
\lfoot{{\it ERC-2018-CoG}}
\rfoot{{\it Extended Synopsis}}
\cfoot{{\it Page \thepage\ of 5}}
%\rfoot{{\it FP7-PEOPLE-2013-IIF}}

\newenvironment{itemize*}%
  {\begin{itemize}%
    \setlength{\itemsep}{0pt}%
    \setlength{\parskip}{0pt}}%
  {\end{itemize}}


\begin{document}

\begin{center}
 {\Large \bf \textcolor{Cerulean}{Data Science at the Edge of the Universe: Using Quasars to kickstart  \\}}
\vspace{4pt} 
  {\Large \bf \textcolor{Cerulean}{the new field of Extragalactic Variable Astrophysics} }
\end{center}

\begin{quotation}
\noindent
{\it 
Black holes are omnipresent in our Universe, and black holes that are
millions to billions of times the mass of our Sun, are ubiquitously
found at the centers of galaxies, including our own Milky Way.
Initially consider physical oddities, we now strongly suspect that
these central, ``supermassive'' black holes have a profound
affect on the galaxies that they live in. This is not surprising since
the potential energy associated with mass accretion onto a
supermassive black hole is comparable to that generated via the
nuclear fusion in the galaxy's stars.
%%
However, the interaction and the physical processes involved in how
this energy escapes the inner most regions of the galaxy and then
interacts with the gas, dust, stars and dark matter, is currently very
poorly understood theoretically, with observational data giving little 
insight on how to make key progress.

The field is poised for a fundamental and rapid change. The first data
are now in hand that show changes on human timescales in external
galaxies, with these new field defining studies including projects led
by the P.I.  Moreover, a fleet of telescopes, detectors and missions
are about to come online over the next few years that will once again
leap-frog the quality and quantity of data we have available
today. Over the course of the next 5-6 years, surveys including
SDSS-V, LSST, DESI, 4MOST and Euclid will see first light. Even more
imminent is the launch of the James Webb Space Telescope.

This proposal has two broad but well-posed goals. 
First, we aim to elucidate, for the first time, how the energy directly associated with a 
supermassive black holes impacts the universal galaxy population.  
%%
Second, will open up and explore the Variable Extragalactic Universe, bringing to bear 
the slew of... CLQs, TDEs, Binary BHs, binary SMBHs... 
Things will go ``bang in the middle of the night''; we just don't know what 
they are yet. 
%
%% NS-NS merger (EM signatures in the blue...) ...
%% Unify the four fundamental forces of nature...
%
%
We will achieve this by leveraging 
several of the new, large-scale surveys that are coming online in the next few years. 
The scope and remit of an ERC Consolidator grant will allow us to combine these 
data products in a manner that will 
% This programme will 
not only establish the new state-of-the-art in extragalactic variable science, 
{\rm it will establish and kickstart the new field of extragalactic variable science itself}. 
%%
The goal is to connect the physics invoked here
from sub-parsec to cosmological scales, and to investigate the
physical processes that link luminous AGN activity and the formation
and evolution of massive galaxies. These critical observations are
made by exploiting the large imaging and spectroscopic datasets that
we will have available from the 
SDSS-V, DESI,  4MOST, LSST and ESA Euclid. 
%%
We ask for the personnel to accomplish these vey ambituous, 
but achievable goals, along with the `buy-in' to the facilites we need access to. 
We also ask for the computing and travel support that will directly enable this research. 

Our ERC Consolidator grant proposal will radically improve our understanding of 
one of the two fundamental energy sources available to galaxies; that of accretion 
onto the compact object in the central engine. 
%%
The P.I. is a world-leader in observational quasar astrophysics, both in terms of 
survey work and individual object study. 
%%
Our proposal takes astrophysics into the 2020s, going from single objects samples, 
to surveys and samples of millions of objects leveraging these multi-billion Euro/dollar/pound  
next generation missions, telescopes and their subsequent datasets. 
%%
%\noindent%
%% 
%Here we ask for 3 postdocs, and `buy-in' to two of the surveys. 
}
\noindent
\end{quotation}

%\smallskip
%\smallskip
%\noindent
%Astrophysics is humankind's scientific endeavour to understand the
%universe and our place in it. \\
%-- How did the Universe begin and evolve? \\
%-- How did galaxies, stars, and planets come to be?\\
%-- Are we alone? \\


\section*{The Future is Now}
%%  You need to answer 5 key questions: 
%%
%    Why bother? , 
%    What problem are you trying to solve?
\smallskip
\smallskip
\noindent
\textbf{\textsc{Global Motivation:}}
At it's heart, there are two major motivations for our project. 
The first is to gain a deep understanding into the physical mechanisms 
related to central engine black holes; their accretion disk physics, their 
dynamics on both human and galactic timescales and the role they might 
play in forming, and regulating the galaxy population. These are among the 
most prescient astrophysical questions of our time, and in an area where 
major breakthroughs are imminent. 

\smallskip
\smallskip
\noindent
The importance of this branch of astrophysics is already well establish in 
Europe and is a priority for the next two decades. This is demonstrated by noting that
one of the two primary mission goals for the Advanced Telescope for High-ENergy Astrophysics (ATHENA) is 
answering the querstion ``How do black holes grow and shape the Universe?''. 
ATHENA is ESA's second L-class flagship mission, due for launch in 2028.
% Spatially-resolved X-ray spectroscopy and deep, wide-field X-ray spectral imaging

\smallskip
\smallskip
\noindent
The second motivation is the massive, untapped and raw discovery space 
that the new experiments will open up, and the likely outcome of discovering 
something ``brand new''. It is somewhat tricky to say specifically what to 
expect, but the fact that e.g. LSST will deliver a dataset {\it so spectaularly} 
difference both in sky coverage and time-sampling coverage, means the 
Universe would have to be an exceptionally boring (and unkind!) place to 
not have a brand new astronomoical object and astrophysical phenomena 
be discovered. 


%% Is this a European priority? , 
%% Could it be solved at a national level? 
\smallskip
\smallskip
\noindent
\textbf{\textsc{Maximising Science Returns from European priorities:}}
Contemporary astronomy is a multi-national endevour with many leading facilites being 
international collaborations. Although a project, with similar but much less ambitious science 
goals and return could be envisaged at the national level, the full discovery and break-through nature being described herein only comes to the fore
when the data from the various international collaborations are combined intelligently. 
Critically data from leading  European Southern Observatory (ESO) and 
Eeuropean Space Agency (ESA) facilites will play a pivital role here. 

\smallskip
\smallskip
\noindent
Although many of the ``building blocks'' for the the solution are already available
(e.g. open source codes, database infrastructures, the methodology of catalog creation and combintion) 
no one has yet to combine the data in the way we envisage. Moreover, the datasets we desire
to deliver our paradigm changing science are only coming online over the next 5-10 years.


\smallskip
\smallskip
\noindent
%% Why now? 
%% What would happen if you did not do this now?, 
\textbf{\textsc{The TIming and the Team:}}
{\it The timing for this proposal could not be better or more imperative.} 
The first of the data data ``firehose'' turns on in late 2019, with
the full datastream from our key sources fully online around 2021. 

\smallskip
\smallskip
\noindent
The model indicates that the first group of people to use a new
product is called ``innovators'', followed by ``early adopters''. 

\smallskip
\smallskip
\noindent
%% Why you?. 
%% Do you have the best consortium to do this work? 
The P.I. has unmistably become a world-leader in the field of 
extragalactic quasar observational astrophysics. 
Moreover, however, the University of Edinburgh is now poised to be an 
astronomical data centre nexus, with 

\smallskip
\smallskip
\noindent
%There are very few people in the world that 
The P.I. has bulit their career on this science case, and has already been a P.I. 
of a Working group team (as part of a collaboration) with prodigious scientific
output (400 published, peer-reviewed papers and counting). 


%%
Given the science goals, the P.I.'s track record and ambition and 
the nature of this project, this proposal satisfies all the aims and goals
of the ERC Con. 
%%
%%
%%
You should aim to have these addressed in the first paragraph of your proposal! 
%%
The “WHY NOW” is one of the most important ones for ERC.

%% The ERC panel will evaluate the PI’s “intellectual capacity, creativity and commitment”. 
%% This includes:
%% -- ability to propose and conduct ground-breaking research and
%% -- achievements going beyond the state-of-the-art
%% -- abundant evidence of creative independent thinking
%% -- the ERC grant would contribute significantly to the establishment and/or further consolidation of the PI's independence
%% -- commitment to the project (minimum 50\% of the PI’s total working time)


%%%%%%%%%%%%%%%%%%%%%%%%%%%%%%%%%%%%%%%%%%%%%%%%%%%%%%%%%%%%%%%%%%%%%%%%%%%%%%%%
%%
%%  https://tex.stackexchange.com/questions/337820/mcq-long-table-using-tikz-tcolorbox-or-tabular
%%  https://tex.stackexchange.com/questions/283419/color-in-a-multirow-cell-with-extra-vertical-space/283454
%%  https://tex.stackexchange.com/questions/406033/how-to-fit-a-cell-of-a-table-to-a-figure-and-arrange-multiple-tables/406042
%% 
%% THIS (??)::
%%     https://texblog.org/2014/05/19/coloring-multi-row-tables-in-latex/
%%
%%
%%   https://www.inf.ethz.ch/personal/markusp/teaching/guides/guide-tables.pdf
%%
%%%%%%%%%%%%%%%%%%%%%%%%%%%%%%%%%%%%%%%%%%%%%%%%%%%%%%%%%%%%%%%%%%%%%%%%%%%%%%%%


\begin{tcolorbox}[tab1, tabularx={X  X }, title=Outstanding Issues in Extragalactic Astrophysics, boxrule=1.25pt] 
Key issue                                                                            &  Novel investigation       \\ 
\hline \hline
\multicolumn{2}{c}{{\sc The physics of accretion}} \\ 
Investigating ``hot'' and ``cold'' mode accretion in the quasar population; 
determining the rates and timescales, and characterising the Changing Look Quasar (CLQ) population.   &     
Identifying and characterizing  all the CLQs in DESI and SDSS-V.  \\ 
\hline
Probing the inner parsec of the quasar central engine & 
Rapid analysis and response on LSST quasar light curves. \\ 
\hline
%%
\multicolumn{2}{c}{{\sc Obscured accretion and galaxy formation}} \\
Establish the relative importance of major mergers, minor mergers, cold streams and secular evolution 
have towards the growth of SMBHs across cosmic time. & 
Deep imaging data from LSST combined with searching for post-starburst signatures 
in DESI, SDSS-V, 4MOST spectra. NIRcam and MIRI imaging from JWST. \\ \hline
Establishing the bolometric output and origin of IR emission, and  
determine presence of extreme outflows in the $z\sim2-3$ quasar population. & 
MIRI MRS spectroscopy with JWST.\\ \hline
Establishing the range of SED parameter space the quasars occupy by a multi-wavelength multi-epoch ``truth table dataset'' & 
Building ``The Stripe 82 Rosetta Stone'' (SpIES, SHELA, VICS82, S82X, HSC; repeat optical observations from SDSS, DES) \\ \hline
%%
Find the physical conditions under which SMBH grew at the epoch when most of the accretion and star formation in the Universe occurred ($z\sim1-4$) & Perform a complete census of AGN across $z\sim0-7$, focussing on $z=1-4$ using medium-deep multiwavelength datasets \\ \hline
\multicolumn{2}{c}{{\sc Galaxy-scale feedback}}\\
Establishing the theoretical impact of extreme outflows in the $z\sim2-3$ quasar population & 
Hydro simulation modelling.  \\
\hline
Understand how the accretion disks around black holes launch winds and outflows and determine how much energy these carry. 
Quantify the amount of ``Maintenance/Jet/Kinetic'' mode and ``Transition/Radiative/Wind'' mode feedback.
& 
Identifying and characterizing  all the CLQs in DESI and SDSS-V.  \\ 
    %\end{tcbitemize}
\end{tcolorbox}





\smallskip
\smallskip
\noindent
If we are to understand galaxy formation and evolution, we have to
understand the two major power sources available to a galaxy; nuclear
fusion in stars and accretion onto compact objects. Our proposal
directly addresses the latter process. This is a worldwide research
priority, but is also a European priority, due to the investment in
the ESA {\it Euclid} and {\it Gaia} missions, and the developements of
new ESO telescopes (ELT) and instrumentation (4MOST).  (e.g.
{\bf Also connection to gravitational waves!!!} 



\noindent
Why (the ERCs)??\\
World class science is the foundation of tomorrow’s technologies, jobs and wellbeing\\
Europe needs to develop, attract and retain research talent\\
Researchers need access to the best infrastructures\\


\subsection{Open Innovation, Open Science, Open to the World}
Every part of the scientific method is becoming an open, collaborative and participative process \\




\noindent
{\bf \underline {Background:}}
%{\bfseries \large \textcolor{Cerulean}{Background:}}
%{\bfseries \underline{\textcolor{Cerulean}{Background:}}}
Where do galaxies come from? How do black holes form and grow? And
what is the history, and fate, of the Universe?  These are the
deepest, most fundamental questions in astrophysics and cosmology, and
sets the scene for my research.

\smallskip 
\smallskip
\noindent
We now know that in the local Universe, there is a link between the
key properties of massive galaxies, such as bulge mass, and their
central supermassive black holes (SMBHs; e.g., [1], [2]). This has led
to the proposal that the supermassive black hole, when accreting, has
an influence on its host galaxy by the means of some regulatory
``feedback'' mechanism(s) (e.g., [3], [4]). However, the details of
the physical processes involved in AGN feedback are still disputed
and, moreover, direct observational evidence for AGN feedback in the
early universe is conspicuous by its absence (e.g., [5], [6]). Hence,
a major source of uncertainty in our current understanding of galaxy
evolution is how supermassive black holes influence, and potentially
regulate, their host galaxies.
%%
What is the main AGN triggering mechanism at the height of quasar
activity? What direct observational evidence in individual objects
links AGN activity to star formation?  Can we observe ``AGN feedback''
in action, in situ, for the most luminous sources?  Such unknowns
about the co-evolution of black holes and their host galaxies remain
among the most fundamental unanswered questions in extragalactic
astronomy.


\smallskip 
\smallskip
\noindent
Furthermore, the details of the physical processes involved in the AGN
activity including how the SMBH directly couples and affects its most
local environment, i.e., the accretion disk, broad line region and
dusty torus, are still unknown at this point (e.g., [7], [8]). Noting
- and for the most part, currently ignoring this issue - quasars have
become key cosmological probes; by being high-$z$ tracers of the
underlying matter distribution ([9]) and as ``backlights'' to observe
the IGM (e.g., [10-15]).



%\smallskip \smallskip
\medskip 
\medskip
\noindent
{\bfseries \large \textsc{\textcolor{Cerulean}{Current Research Highlights}}}

\smallskip
\smallskip
\noindent
My current research has involved discovering new types of quasars,
which are proving to be the key laboratories in (a) understanding the
physical processes linking accretion disk physics to the host galaxy
and (b) understanding the physical processes linking quasar activity
to the host galaxy and wider galactic environment.

\smallskip
\smallskip
\noindent
Quasars are ideal laboratories and tools for
three main lines of investigation: {\rm (i)} to learn about the
physical processes in accretion disks observed on $\lesssim$year
timescales; {\rm (ii)} connecting accreting active galactic nuclei
(AGN) with galaxy evolution and {\rm (iii)} to use quasars as
cosmological probes. 







\smallskip \smallskip
\smallskip
\smallskip
\noindent
%\textbf{\textsc{Changing-Look Quasars:}}
\textbf{\textsc{A microscope for rapid Central Engines:}}
Recently ``Changing-look'' quasars (CLQs; [22-25]) have been
identified, and are defined to be luminous AGN which have a dramatic
appearance, or disappearance, of their broad emission-line component
on observed-frame month-to-year timescales.  CLQs are important since
they offer a direct observational probe into the physical processes
dictating the structure of the broad-line region (BLR). These
timescales can potentially be associated with the viscous timescale
(the drift time through the accretion disk), the light crossing
timescale (critical for reverberation mapping and disk reprocessing)
and the dynamical timescale of the BLR.  CLQs are thus an ideal
laboratory for studying accretion physics, as the entire system
responds to a large change in ionizing flux on a human timescale.

\smallskip \smallskip
\noindent 
In [25] I co-led the first systematic search for CLQs based on
photometry from SDSS and Pan-STARRS1, along with repeat spectra from
the SDSS/BOSS, and reported the discovery of 10 CLQs. This is a
startling result since we now estimate $\approx$10-15\% of bona fide
quasars may exhibit `changing look' behaviour on $\sim$10 year 
(rest-frame) timescales. However, plausible time-scales for variable
dust extinction are factors of $2-10$ too long to explain the dimming
and brightening in these sources.  Changes in accretion rate are the
currently favored explanation for CLQs, but then the question of how
the inner accretion disk couples to the BLR immediately
arises. Further investigation is thus warranted.

\begin{figure}[h]
  \begin{center}
   \hspace{-0.5cm}
%   trim=l b r t
    \includegraphics[height=5.5cm,width=18.0cm] %, trim={0.05cm 0 0.05cm 0},clip]
    {figures/WISE_SDSSzoomHSC_ERQ-image_v3.pdf}
    \vspace{-10pt}
   \caption{}
  \vspace{-12pt}
 \label{fig:Keynote_facilites}
\end{center}
\end{figure}



%\smallskip \smallskip
\medskip\medskip
\noindent
%{\bf \large \underline{Future Research}}
{\bfseries \large \textsc{\textcolor{Cerulean}{Future Research}}}

\smallskip
\smallskip
\noindent
My future research builds on, and expands my current research program;
I have a bold research vision that is designed to be addressed by a
research group, and the environment, current research areas and
telescope access in the Department of Physics and Astronomy at Dartmouth College
are ideal to
carry out these investigations.
%%
The science questions we seek to address are well-posed, yet strike at
the heart of major and still open extragalactic astrophysical
questions: Do we have a full accounting for the accretion history in
the Universe?  How does the energy `escape' from the central engine to
the host galaxy?  Are the modes of AGN ``feedback'' that regulate a
galaxy the same that regulate the AGN itself?  What are the
star-formation properties of mid-infrared luminous quasars at the peak
of quasar activity?  What are the evolutionary properties, if any, of
dark energy?


\smallskip \smallskip
\smallskip
\smallskip
\noindent
\textbf{\textsc{New IR investigations into the CLQ Population:}}
Taking advantage of new optical imaging data from the Dark Energy
Camera Legacy Survey \href{http://legacysurvey.org/decamls/}{(DECaLS)}
and new IR light-curves from NEOWISE ([26, 27]), we have made further
in-roads into understanding the CLQ population. This includes
identifying objects with rapidly changing IR light-curves and also
accretion disk changes, e.g. the $z=0.378$ quasar SDSS
J110057-005304.4, see Figure~\ref{fig:J110057}. From J110057, my new
model ([28]) suggests a dramatic new picture of the physics of the
CLQs governed by processes at the innermost stable circular orbit
(ISCO) and the structure of the innermost disk. {\it We have embarked
on a new observation campaign, gaining optical light-curves (from the
Liverpool Telescope) and spectra (from WHT and Palomar) to test this
startling new hypothesis.}


\smallskip \smallskip
\smallskip
\smallskip
\noindent
\textbf{\textsc{Quasars, Dark Energy and Very Wide Field Surveys: }}
In the community's future is the prospect of very wide-field
ground-based surveys using both imaging and spectroscopy, in the North
and Southern Hemisphere. Building on {\it (i)} my leadership
experience and heritage from being an integral part of BOSS and {\it
(ii)} my world-leading expertise in quasar target selection,
demographics, and physical properties, I will continue to lead the
scientific development of using quasars as large-scale structure (LSS)
tracers in these new surveys.


\smallskip \smallskip
\noindent 
%{\bf \underline {Outline of Future Research:}}
Prior to SDSS-III BOSS, quasars lagged behind massive galaxies as good
tracers of LSS. However, with the evolution of Baryon Acoustic
Oscillation (BAO) and Redshift-Space Distortion (RSD) studies using
BOSS, and in particular the Lyman-$\alpha$ forest (Ly$\alpha$F),
quasars are now seen as key objects in accessing the high-$z$
Universe. Moreover, in the DESI era, there is huge potential to extend
this reach further, first for BAO/RSD and also as Standard Candles. 


\begin{itemize}
\item{{\bf Quasars as Cosmological Probes {\sc I:} Baryon Acoustic
      Oscillations.} I co-led the first investigation that successfully used
    quasars as point test particles to measure the BAO signature
    [15]. This measurement was a cross-correlation with the Ly$\alpha$F,
    but demonstrated that quasars themselves, despite their lower number
    density can trace LSS sufficiently well for BAO studies. This
    provides access to geometry measurements at $z>1$, further
    constraining the Hubble Parameter at high-$z$ and testing the current
    $\Lambda$CDM model.  {\it DESI will be ``the ultimate quasar survey'',
      with sub-per cent measurements of the distance scale (from BAO alone)
      to redshifts of $z\sim3$ using `tracer' and Ly$\alpha$F quasars.}}
  
\item{{\bf Quasars as Cosmological Probes {\sc II:} RSDs with
      Quasars.}  As I proved with the original SDSS sample ([31]), quasars
    can also be used to measure RSDs.  Measurements of the normalized
    growth rate, $f\sigma_{8}$, from RSD using quasars at $z\sim1.5$ is
    crucial since model predictions diverge at these redshifts for
    $\Lambda$CDM+G.R. compared to $f(R)$ gravity, DGP braneworld, and varying
    Gravitational Constant models ([32, 33]). {\it Even with the local
      GW170817 measurements, a range of gravity theories persist and
      testing General Relativity at cosmological scales remains imperative.}}
  
\item{{\bf Quasars as Cosmological Probes {\sc III:} Reverberation
      Mapping Campaigns.}  Recently, [34-37] demonstrated that by using the
    tight relationship between the luminosity of a quasar and the radius
    of its broad-line region, established via reverberation mapping, one
    is able to determine the luminosity distances to quasars.  This means
    that quasars can now be used as standard, or more accurately,
    ``standardizable'' candles (in a very similar way to Type Ia
    supernovae). {\it Thus with light-curve data from LSST and a baseline
      for repeat spectroscopy from SDSS, DESI is ideally placed to exploit
      this new method, with very different systematics to the BAO, and
      access a redshift regime $z>4$ where even the Ly$\alpha$F will not be
      able to offer cosmology constraints.}}

\end{itemize}




\smallskip \smallskip
\smallskip
\smallskip
\noindent
%\textbf{\textsc{Extremely Red Quasars: Feedback in action at high-$z$:}}
\textbf{\textsc{Extremely Red Quasars:}}
In [16] I discovered a new class of object, the ``extremely red
quasars'', that have optical spectroscopy from SDSS/BOSS, and
$r-[22\mu{\rm m}]>14$ colors (i.e., $F_{\nu,\, {\rm MIR}} / F_{\nu,\,
{\rm opt}} \gtrsim 1000$) from the Wide-field Infrared Survey Explorer
(WISE; [17]) satellite, see Figure~\ref{fig:ERQ}.  The ERQs are a
unique obscured quasar population with extreme physical conditions
related to powerful outflows across the line-forming regions. These
sources are the signposts of the most dramatic form of quasar feedback
at the peak epoch of galaxy formation, and may represent an active
``blow-out'' phase of quasar evolution ([18], [19]).  However, due to
the current lack of access to mid-infrared spectroscopy, it is still
unknown whether the large IR luminosities observed in these quasars is
from star formation, which would produce strong polycyclic aromatic
hydrocarbon (PAH) spectral features, or, if it is from the hot dust
near the central quasar, which should produce much weaker/no PAH
emission.




\smallskip \smallskip
\smallskip
\smallskip
\noindent
\textbf{\textsc{SpIES, Telling us about AGN Feedback:}}
My finishing graduate student, John Timlin, is leading the analysis of the {\it
Spitzer}-IRAC Equatorial Survey (SpIES; [20]), which is a new deep
3.6$\mu$m and 4.5$\mu$m imaging survey designed to discover obscured
and faint unobscured quasars in the SDSS Stripe 82 field. The
scientific goals of SpIES are to measure the quasar luminosity
function and clustering to $z\sim4$, and potentially discover a suite
of very high redshift quasars, meanwhile constraining AGN ``feedback''
models.  In [21] we made the first measurements of clustering of
IR-selected $z > 3$ AGNs.  Connecting to current lower-$z$
measurements, we find an ``inefficient feedback'' model is favored,
where all $z>2$ BHs grow to their peak luminosity at $z\sim2$ and then
fully ``shut down'' ($\dot{m}_{\rm accr} \leq 10^{-5}~M_{\odot}$~yr$^{-1}$).

\smallskip \smallskip
\smallskip \smallskip
\noindent
\textbf{\textsc{Early Science with the James Webb Space Telescope: }}
The {\it James Webb Space Telescope} ({\it JWST}) is a 6.5-meter
infrared telescope that will initiate and enable transformative
science. My discovery of the extremely red quasars provides a key
observational clue to the ``major merger'' evolutionary theory for
quasar activity ([29]). I am the P.I. of a large, multi-proposal Cycle 1
programme, that will take maximal advantage of {\it JWST's} new and
uniquely powerful capabilities immediately upon commencing science
operations. The natural science case for {\it JWST} is the detailed
investigation of the incredible richness of the near- and mid-IR spectral
features in the ERQs using the NIRSpec and MIRI spectrographs, and by
utilizing my experience with mid-infrared datasets, surveys and object
discovery, I am leading a team of postdocs and graduate students, to
carry out these investigations. {\it The next JWST Call for Proposals
is in December 2017, and my team is ramping up for a full suite of ERQ
related proposals for this truly revolutionary mission}.


\section{The Future}
Cosmic Vision Themes	The Hot and Energetic Universe
Mapping hot gas structures and determining their physical properties
Searching for supermassive black holes
Athena – Advanced Telescope for High-ENergy Astrophysics – will be an X-ray telescope designed to address the Cosmic Vision science theme 'The Hot and Energetic Universe'. The theme poses two key astrophysical questions:

How does ordinary matter assemble into the large-scale structures we see today? and
How do black holes grow and shape the Universe?

Goes towards ESA {\it Athena}... ; 





\newpage
\begin{center}
\medskip
 \medskip
 {\large \bf References}
    \vspace{-10pt}
\end{center}
\begin{multicols}{2}[]
\noindent
%\footnotesize
%\scriptsize
%\tiny
\lbrack 1\rbrack Kormendy \& Ho, 2013, ARAA, 51, 511\\
\lbrack 2\rbrack Kormendy,  2016, ASSL, 418, 431\\
\lbrack 3\rbrack Alexander et al., 2012, NewAR, 56, 93\\
\lbrack 4\rbrack King \& Pounds, 2015, ARAA, 53, 115 \\
\lbrack 5\rbrack Heckman \& Best, 2014, ARAA, 52, 589\\
\lbrack 6\rbrack Naab \& Ostriker, 2017, ARAA, 55, 59 \\
\lbrack 7\rbrack Netzer, 2015, ARAA, 53,  365\\
\lbrack 8\rbrack Padovani, 2017, A\&ARv, 25, 2\\
%%
\lbrack  9\rbrack Ata et al., 2017, arXiv1705.06373v2\\
\lbrack10\rbrack Solsar et al., 2013, JCAP, 04, 026 \\
\lbrack11\rbrack Busca et al.,  2013, A\&A, 552, 96 \\
\lbrack12\rbrack Delubac et al.,  2015, A\&A, 574, 59 \\
\lbrack13\rbrack Bautista et al., 2017, A\&A, 603, 12 \\
\lbrack14\rbrack du Mas des Bourboux et al., 2017, arXiv1708.02225v3\\
\lbrack15\rbrack Font-Riber et al., 2014, JCAP, 05, 027\\
%%
\lbrack16\rbrack Ross et al. 2015, MNRAS, 453, 3932\\
\lbrack17\rbrack Wright et al., 2010, AJ, 140, 1868\\
\lbrack18\rbrack Zakamska et al., 2016, MNRAS, 459, 3144\\
\lbrack19\rbrack Hamann et al., 2017, MNRAS, 464, 3431\\
%%
\lbrack20\rbrack Timlin, Ross et al., 2016, ApJS, 225, 1\\
\lbrack21\rbrack Timlin, Ross et al., 2017, ApJ, {\it in prep.}\\
%%
\lbrack22\rbrack LaMassa et al., 2015, ApJ, 800, 144\\
\lbrack23\rbrack Runnoe et al., 2016, MNRAS, 455, 1691\\
\lbrack24\rbrack Ruan et al, 2016, ApJ, 826, 188\\
\lbrack25\rbrack MacLeod, Ross et al., 2016, MNRAS, 457, 389\\
%%
\lbrack26\rbrack Meisner et al., 2017, AJ, 153, 38 \\
\lbrack27\rbrack Meisner et al., 2017, AJ, 154, 161 \\
% Meisner  et al., 2017, arXiv1710.02526v1 \\
\lbrack28\rbrack Ross et al., 2017, Nat.As., {\it in prep.} \\
%%
\lbrack29\rbrack Hopkins et al., 2006, ApJS, 163, 1\\
%%
\lbrack30\rbrack Schlegel et al., 2011,  arXiv:1106.1706v2 \\
%
\lbrack31\rbrack Ross et al., 2009, ApJ, 697, 1634 \\
%%
\lbrack32\rbrack Lombriser \& Taylor, 2016, JCAP, 03, 031 \\
\lbrack33\rbrack Baker et al, 2017,  arXiv1710.06394v1 \\
%%
\lbrack34\rbrack Watson et al.,  2011, ApJ, 740, L49\\
\lbrack35\rbrack King et al., 2014, MNRAS, 441, 3454\\
\lbrack36\rbrack King et al., 2015, MNRAS, 453, 1701\\
\lbrack37\rbrack Shen et al., 2015, ApJS, 216, 4\\
%\lbrack37\rbrack The Pierre Auger Collaboration, 2017, Science, 357, 6357 \\
\lbrack38\rbrack Hviding et al.,  2017, arXiv1711.01269v1 



\end{multicols}



\end{document}

\documentclass[oneside, a4paper, onecolumn, 11pt]{article}

\input{format}

\usepackage[left=2.05cm,top=2.05cm,right=2.05cm]{geometry}
\setlength{\bibsep}{0.0pt}

%% To fix list things: 
\setitemize{noitemsep,topsep=0pt,parsep=0pt,partopsep=0pt,leftmargin=*}

\pagestyle{fancy}
%\renewcommand{\headrulewidth}{0pt}      %% Remove line at top

%\pagestyle{empty}
\fancyhf{}
%\lhead{{\it ERC-2018-CoG}}
%\lhead{{\it DEQUASARS: Part B1 }}
\lhead{{\it Ross}}
\chead{Part B1}
\rhead{Q4D}
\setcounter{page}{1}
\lfoot{{\it ERC-COG-2018}}
\rfoot{{\it Extended Synopsis}}
\cfoot{{\it Page \thepage\ of 5}}

\newenvironment{itemize*}%
  {\begin{itemize}%
    \setlength{\itemsep}{0pt}%
    \setlength{\parskip}{0pt}}%
  {\end{itemize}}


\begin{document}


\smallskip
\smallskip
\noindent
{\bf{\textcolor{Cerulean}{a. Extended Synopsis}}} 
\vspace{6pt}

\noindent
\large
{\bf{\textcolor{Cerulean}{Overview and Objectives}}}
\normalsize

\smallskip
\noindent
Current theories of galaxy formation and evolution strongly suggest
that central, supermassive black holes (SMBHs) have a profound effect
on the galaxies that they live in \citep{Vogelsberger2014Nature,
Schaye2015, SomervilleDave2015,Dave2017}.  This is not surprising
since the potential energy associated with mass accretion onto a
supermassive black hole is comparable to that generated via the
nuclear fusion in the galaxy's stars \citep[see
e.g. ][]{Fabian2012}. Thus when a galaxy goes through a ``quasar''
phase (where gas is supplied and accreted by the SMBH) there is ample
energy to potentially impact the host galaxy and the surrounding
intergalactic medium.
%\todo{Just an example of a ToDo} 

\smallskip
\smallskip
\noindent
However, the details of the physical processes involved in how this
energy escapes the inner most regions of the galaxy and then interacts
with the gas, dust, stars and dark matter, is currently poorly
understood, with current observational data giving more puzzles than
clues on how to make progress. Significant further issues arise since
startling new observations from my (Nicholas P. Ross; NPR)
research team \citep{MacLeod2016, Ross2018} show that {\it quasars
vary significantly on timescales of weeks to months}, whereas the
accretion disks (that supply `fuel' for the quasar) should take
thousands of years to change their optical emission; this has recently
been called the ``Quasar Viscosity Crisis'' \citep[e.g.,
][]{Lawrence2018}. Thus, it is unclear if we have an understanding of
a physical phenomena prevalent in many astrophysical systems: the
accretion disk.

\smallskip
\smallskip
\noindent
The field of observational extragalactic astrophysics is poised for a
fundamental and rapid change. Starting in late 2019, a fleet of new
telescopes, instruments and missions will be commissioned, start data
taking, and will leap-frog the quality and quantity of data we have
available today. These surveys and missions include: the fifth
incarnation of the Sloan Digital Sky Survey
(SDSS-V\footnote{\href{www.sdss.org/future/}{{\tt
www.sdss.org/future/}}}); the Large Synoptic Survey Telescope
(LSST\footnote{\href{lsst.org}{{\tt lsst.org}}}); the Dark Energy
Spectroscopic Instrument (DESI\footnote{\href{desi.lbl.gov}{{\tt
desi.lbl.gov}}}) survey; the 4-metre Multi-Object Spectroscopic
Telescope (4MOST\footnote{\href{4most.eu}{{\tt 4most.eu}}}) survey,
and the ESA {\it Euclid}
mission\footnote{\href{sci.esa.int/euclid/}{{\tt
sci.esa.int/euclid/}}}. Even more imminent is the launch of the {\it
James Webb Space Telescope} (JWST\footnote{\href{jwst.stsci.edu}{{\tt
jwst.stsci.edu}}}).

\smallskip
\smallskip
\noindent
This proposal has two broad and well-posed goals. First, we aim to
elucidate in detail {\bf how the energy directly associated with a
supermassive black holes impacts the universal galaxy population.} We
will gain a deep understanding into the physical mechanisms related to
central engine black holes; their accretion disk physics, their
dynamics on both human and galactic timescales and the role they might
play in forming and regulating the galaxy population.
%%
Second, we anticipate {\bf the discovery of brand new extragalactic
phenomena.}  By tapping into the massive and raw discovery space that
the new experiments will open up, there is the highly likely outcome
of discovering something ``brand new'' \citep{Ivezic2008,
LSST_ScienceBook}, 
e.g. the EM counterparts to mergers of Binary SMBHs 
(with their associated gravitational wave chirp and ringdown), 
extragalactic microlensing events, 
objects similar to repeating fast radio bursts \citet{Spitler2016}
or more objects akin 
to the still unexplained `SCP 06F6' \citep{Barbary2009}. 
%%
Our major science objectives are:
\begin{enumerate}
\item Characterize the variable extragalactic universe and quasar population. 
\item Establish the energy transport mechanisms associated with the``quasar phase'', and explain the relation between accretion rate, black hole mass build-up with observed light curve and spectral properties. 
\item Develop and then link theoretical accretion and galaxy formation models for a fully holistic theory of active galaxies. 
\item Discover new extragalactic variable objects. 
\end{enumerate}

\smallskip
\smallskip
\noindent
We will achieve this by leveraging several of the new, large-scale
surveys that are coming online in the next few years. These critical
observations are made by exploiting the large imaging and
spectroscopic datasets that will be available from the SDSS-V, DESI,
4MOST, LSST and ESA {\it Euclid}. {\it Crucially, although these
projects individually will deliver new state-of-the-art datasets, it
is our project that will be the first to break down the associated
data silos and combine these data in order to go beyond the
state-of-the-art.}


\medskip
\medskip
\noindent
\large
{\bf{\textcolor{Cerulean}{1. Current State of the Art.}}}
\normalsize

\smallskip
\noindent
The current state-of-the-art data samples have either
$\approx$10$^{6}$ quasars with one spectral epoch, or only a few
objects with repeat photometric data, i.e. light-curve information and
the accompanying repeat spectra (see Figure~\ref{fig:J110057}).  NPR
has been involved in the production of both of these two types of
samples \citep{MacLeod2016, Paris2017}. We plan to collate datasets so
that the 10$^{6}$ sample have high-fidelity light-curves {\it and}
ample repeat spectroscopy, and in doing so will kick start the new
field of Variable Extragalactic Astrophysics.


\begin{figure}[h]
  \begin{center}
    \hspace{-0.5cm}
    \includegraphics[height=6.25cm,width=17.2cm]
    {figures/J110057_LC_Spectra_20171024.pdf}
    \vspace{-10pt}
    \caption{%\small      
      \footnotesize 
      % \scriptsize
      % \tiny
      {\it (Left:)} The optical and infrared light-curve for the redshift $z=0.378$ quasar 
      J1100-0053 (Ross et al. 2018). 
      Note the fall in the infrared, whereas there is a decrease, but 
      then recovery in the optical. 
      {\it (Right:)} 
      Three epochs of spectra for J1100-0053. 
      The spectacular downturn in the blue for the 2010 spectrum 
      indicates a dramatic change in the accretion disk.
    }
  \vspace{-16pt}
 \label{fig:J110057}
\end{center}
\end{figure}

\smallskip
\smallskip
\noindent
During its initial phases of operation the Sloan Digital Sky Survey
(SDSS) obtained spectra of 1 million galaxies in the local
Universe. This dataset has become the {\it de facto} standard for
understanding the present day galaxy population, and sets the boundary
conditions for all theoretical comparisons. The paradigm changing
success of the SDSS was due to having 1,000,000 objects {\it with very
high signal-to-noise photometry and spectra}, enabling multivariate
analysis that is required for galaxy astrophysics investigations.
{\it We desire the sample size and revolutionary understanding with
new temporal dimension of the quasar population, as the SDSS had with
the low-redshift $z\sim0.1$ galaxy population.}  Our proposal takes
quasar astrophysics into the 2020s, going from single objects samples,
to surveys and samples of millions of objects, with massive
spectroscopic monitoring giving access to the time-domain and
leveraging these very large scale next generation missions, telescopes
and their datasets.


\begin{figure}[h]
  \begin{center}
   \hspace{-0.5cm}
%   trim=l b r t
    \includegraphics[width=16.0cm] %, trim={0.05cm 0 0.05cm 0},clip]
    {figures/Timelines_and_Facilites.pdf}
    \vspace{-10pt}
   \caption{Facilites, Timelines and Priorities. With SDSS-V and DESI in the Northern Hemisphere and 
4MOST, LSST in the South, we have full celestial sphere coverage.}
  \vspace{-12pt}
 \label{fig:Keynote_facilites}
\end{center}
\end{figure}

\smallskip
\smallskip
\noindent
{\it The timing for this proposal could not be better.}  The first of
the data ``firehoses'' turns on in late 2019, with the full datastream
from our key sources fully online by mid-2022.  As such, we have the
time to mature our analysis techniques, and then be in the ideal
position to take advantage of the initial data releases of all these
new projects.
%%
Prompt ERC Consolidator level-support is also imperative since final
survey design and optimization trade-off studies are being made
e.g. for DESI, SDSS-V and LSST over the next $\sim$12-18
months. Having the ability to influence these decisions to our science
goals would be very powerful. Also, having the science teams and
various collaborations know that the PI is embarking on this program
will help attract the best personel for the PDRAs, {\it who would be
guaranteed ``First Light'' data and science.}

\smallskip
\smallskip
\noindent
The importance of this branch of astrophysics is already well
established in Europe and is a priority for the next two decades. This
is demonstrated by noting that one of the two primary mission goals
%for the Advanced Telescope for High-ENergy Astrophysics (ATHENA) mission 
for the ATHENA mission 
is answering the question ``How do black holes grow and shape the
Universe?''.  ATHENA is ESA's second L-class flagship mission, due for
launch in 2028.

\smallskip
\smallskip
\noindent
{\it The scope and remit of an ERC Consolidator grant will allow us to
combine these data products in a manner that will not only establish
the new state-of-the-art in variable extragalactic astrophysics, but it 
will also establish and kickstart the new field of variable extragalactic
astrophysics itself.}




\medskip
\medskip
\noindent
\large
{\bf{\textcolor{Cerulean}{2. Methodology}}}
\normalsize

\noindent
Our proposal contains six work packages that fall into three broad
and complementary categories: observational studies of large numbers
(millons) of objects; high-risk, very high-reward observational
studies of a small number (10s) of objects; theoretical modeling
investigations. Table 1 summarises our overall WP plan. Risks and
mitigation strategies are present for each WP as are Key Deliverables.

\smallskip
\smallskip
\noindent
We define three PDRAs, ``PDRA1'', ``PDRA 2'', ``PDRA 3'', and one PhD
student, ``PhD1''.  The skill set of PDRA1 would include development
of the underlying tools and techniques necessary to extract meaning
from large and/or complex data sets.  The skill sets of PDRA2 would
include expertise in time series analysis, primarily with optical
data but potentially also in other wavebands.  The skill set of PDRA3
would include experience with fluid mechanics modelling and/or large
computer simulations.  PhD1 would have a Masters or a strong 4-year
undergraduate degree in Physics or Mathematics with evidence of
research-level project work.


\smallskip
\smallskip
\noindent
\textbf{\textsc{WP1: Build QuasarSieve:}} 
Raw events come from LSST. The UK LSST Data Access Center (DAC, based
here at the University of Edinburgh) ingests this datastream and
re-emits a filtered stream. In order to utilize this filtered
datastream for our science goals we will build a ``Stage 2 filter'',
which we name {\it QuasarSieve}.  This second stage filter will
identify the quasars, add context, perform outburst forecasting etc.
Our light-curve algorithm will sit on top of {\it QuasarSieve} and
will trigger other telescopes to get e.g. timely spectrum or infrared
data.
%% 
The heavy-industry computing infrastructure is being supplied by the
LSST DAC and our task will be to build software in a timely and robust
manner.  This is a novel enterprise and a rate-limiting step in our
overal programme, with the associated high-risk.  We mitigate this
risk with the data science and machine learning experience from PDRA1
and the P.I. (NPR).  We will also mitigate risk by taking advantage of
the algorithm resources and LSST DAC staff, here at the Royal
Observatory, Edinburgh.  We thus classify {\bf WP1 as medium-risk,
high-reward.}  {\bf Key Deliverables:} An open-source, well-documented
software package that can interact with and return data from the LSST
Data Access Center.


\smallskip
\smallskip
\noindent
\textbf{\textsc{WP2: Quasar Catalogue Generation and Demographic studies:}}  
Building the quasar corpus and cataloguing the observational data will
be a vital step in beginning to pursue our science goals. This
catalogue will be the glue that binds the observational projects
together and will have not only the data, but also the metadata to
enable the other WPs.  Following on from the quasar catalogue
generation, a key science output will be the study of the quasar
demographics.  Luminosity function, clustering and higher-order
statistics will be made in order to precisely determine the census of
quasars, their environments, their host galaxy preferences and their
evolution. All these are vital observational tests for galaxy
formation models and theory (see WP4 below). The goal of this WP is to
construct a quasar catalogue and make key observational tests.
Given the P.I.s experience at these specific tasks, plus the effort
level of PDRA1, PDRA2 and PhD this WP is deemed medium-risk.
{\bf WP2 is medium-risk, high-reward.}  {\bf Key Deliverables:} A
science-enabling compendium that will be the state-of-the-art quasar
dataset for the 2020s.  A suite of new, beyond-the-state-of-the-art
quasar demographic measurements which are the boundary conditions for
theoretical models.


\smallskip
\smallskip
\noindent
\textbf{\textsc{WP3: Light-Curve and Spectral Analyses:}} 
Another major scientific output that will originate from the quasar
corpus catalogue generation will be the full and detailed light-curve
and spectral analyses of the said catalogue. This will result in the
discovery of light-curve trends with quasar type, new methods to
measure black hole mass and the key science goal to see which quasars
are ``changing-look'' objects. This WP will have a data
science/machine learning aspect.  The goal of this WP is to elucidate
the physical processes that drive quasar variability.  The full
Light-Curve and Spectral Analyses that we envisaged will be a
significant amount of work, leading to significant high-reward
science.  
%
{\bf WP3 is medium-risk, high-reward.}  
%
This level of investigation is highly novel, though we envisage no
major barriers outside of our control to achieving our science goals
and PDRA1, PDRA2, as well as the P.I. (NPR) and PhD1 effort will be
directed towards this.
%
As such, we deem this medium-risk.  {\bf Key Deliverables:}
Measurements, for the first time of how the light-curves and spectra
of quasars depend on key physical quasar properties e.g. $M_{\rm
SMBH}$, luminosity, $\lambda = \log (L / L_{\rm Edd})$, spin etc.
These measurements will allow us to make direct comparisons to
accretion disk models.


\smallskip
\smallskip
\noindent
\textbf{\textsc{WP4: Accretion Disk and Quasar Feedback Simulations:}} 
New accretion models are needed to fully explain the observational
data of ``changing look'' quasars that we have examples of today and
the ``Quasar Viscosity Crisis''. New radiation MHD codes begin to
explain the observations here, but further development is needed to
gain the desired deep understanding. Cosmological-scale hydrodynamic
simulations with stellar and quasar feedbasck are now also online. The
exceedingly ambitious goal of WP5 is to develop new holistic accretion
disk-to-cosmological scale simulations that explain our observational
results and link them to ``quasar feedback''.  WP4 is thus high-risk
due to its novel nature and algorithmic complexity.  We also envisage
ramp-up time to get our theoretical simulations to the level that will
be required by our beyond-the-state-of-the-art dataset.  However, we
mitigate this risk first by noting this will be the lead WP and top
priority for PDRA3.  We further mitigate this risk by invoking
collaboration with accretion disk theorist Prof. Ken Rice (WKMR; Chair
of Computational Astrophysics at the IfA, University of Edinburgh) and
Prof. Romeel Dave (RSD; Chair of Physics in the IfA, University of
Edinburgh).
%%
Thus PDRA3, NPR, potentially PDRA2, with guidance where nesceassy from
WKMR and RSD would collaborate on this WP.  We thus classify {\bf WP4
as medium-to-high risk, very high-reward.}  {\bf Key Deliverables:}
New accretion disk models and theory that explain the light curve data
of our beyond-the-state-of-the-art dataset.  New galaxy evolution
models, describing the hydrodynamics involved on galactic scales, but
related to the quasar central engine.


\smallskip
\smallskip
\noindent
\textbf{\textsc{WP5: Observations of Quasars by the James Webb Space Telescope:}} 
What are the star-formation properties of luminous quasars at the peak
of quasar activity?  We aim to answer this by looking for the presence
of polycyclic aromatic hydrocarbon (PAH) spectral features in infrared
bright quasars with the {\it James Webb Space Telescope} (JWST).  {\bf
WP5 is high risk, high-reward.}  This is an ideal investigation for
the JWST, but we classify this as high-risk since we have to apply for
the telescope time and are not guaranteed the data.  We note this will
be the single WP NPR would lead and does not impact in any direct way
the other WPs. This would lead to very-high gain science.  {\bf Key
Deliverables:} State-of-the-art data products from the JWST,
with the observational evidence and physical interpretation of how
``quasar feedback'' regulates galaxy formation in high-redshift
quasars.


\smallskip
\smallskip
\noindent
\textbf{\textsc{WP6: New Object Discovery:}} 
The LSST will scan the sky repeatedly, enabling it, and us, to both
discover new, distant transient events and to study variable objects
throughout our universe. The most interesting science to come may well
be the discovery of new classes of objects. Suffice to say, this would be exceptionally high-reward. 
%%
{\bf WP6 is high risk, exceptionally high-reward.}
We class this as high risk, since it is tricky to class a WP
with essentially unknown discovery potential as `low-risk'.
However, we {\it nota bene} that a lack of any novel discovery here would 
be a startling null result.  
{\bf Key Deliverables:} Potential discovery of new classes of astronomical objects. 


\begin{figure}[h]
  \begin{center}
   \hspace{-0.5cm}
%   trim=l b r t
    \includegraphics[width=16.0cm] %, trim={0.05cm 0 0.05cm 0},clip]
    {figures/workplan.pdf}
    \vspace{-10pt}
  \caption{An overview of our WPs:  the personel attached to each WP and a guide to 
their start and duration is shown. 
%The connection between the WPs is also shown, be it generally one-way  (square starting points) or an iteration (both ends pointed). As such, the flow arrows are guides and not specifying exact timescales. 
As given be the shadings, WP1, 2 and 3 are 
observational studies of large numbers of objects; WP4 are theoretical modeling
investigations and WP5 and 6 are high-risk, very high-reward observational
studies of a small number of objects.}
  \vspace{-12pt}
 \label{fig:Keynote_facilites}
\end{center}
\end{figure}


\smallskip
\smallskip
\noindent
\large
{\bf{\textcolor{Cerulean}{3. Resources,  Survey `buy-in' and Budget}}}
\normalsize

\smallskip
\smallskip
\noindent
\textbf{\textsc{Personnel:}} 
We request the resources and support for 100\% of the time and effort
for the P.I. We request the resources and support for 3 Postdoctoral
Research Associates (PDRAs), for a total of 10 PDRA year equivalents
(3+3+4). We request the resources and support for 1 PhD studentship.


\smallskip
\smallskip
\noindent
\textbf{\textsc{Survey Buy-in:}} 
We request support for the ``buy-in'' to two of the new surveys,
SDSS-V and DESI. The costs here are \euro184,100 for SDSS-V and
\euro200,100 for DESI.  We ask this support to come from the
``additional funds that can be made available to cover access to large
facilities.''  We request access to these funds as it
gives our project access to telescopes and data in the North and
South Hemispheres for complete coverage of the celestial sphere 
and delivers the crucial early spectroscopy that will be vital to
train, test and build our data science and machine learning codes and
algorithms.  We emphasise that the science return is `exponentially'
dependent on the breadth of data available and heralds a brand new regime of ``several-survey'' or
``multi-mission'' astronomy.  {\it Buy-in here would place the
P.I. and the University of Edinburgh as the only group and institute
in the world to be involved in SDSS-V, DESI, 4MOST, LSST and ESA {\it
Euclid} and JWST}.

\smallskip
\smallskip
\noindent
\textbf{\textsc{Computing Requirements:}} 
With the availability of the facilities at an institute (e.g. IfA
Cullen), university
(e.g. \href{https://www.ed.ac.uk/information-services/research-support/research-computing/ecdf}{Edinburgh
Compute and Data Facility}) and at a national
{\href{https://www.hartree.stfc.ac.uk/Pages/home.aspx}{(The Hartree
Centre)} level, the rate limiting factor will be how quickly and
efficiently we can deploy our codes and analysis. 
%The rate limiting factor, in the vast majority of endevours, is (a) data access and (b) development.


\smallskip
\smallskip
\noindent
\textbf{\textsc{Travel:}} 
We request support for travel for all 5 members of the group,
including repeat medium-term (i.e., few weeks) travel to the US and
ESO Chile to work with key collaborators at critical timings of the
First Light for the new telescopes.




\newpage
%%\thispagestyle{empty}
\fancyhf{}
%\lhead{{\it ERC-2018-CoG}}
%\lhead{{\it DEQUASARS: Part B1 }}
\lhead{{\it Ross}}
\chead{Part B1}
\rhead{Q4D}
\setcounter{page}{1}
\lfoot{{\it ERC-2018-CoG}}
\rfoot{{\it Extended Synopsis}}
%\cfoot{{\it Page \thepage\ of 5}}

\bibliographystyle{plainnat}
\bibliography{/cos_pc19a_npr/LaTeX/tester_mnras}

%\input{references}


\end{document}

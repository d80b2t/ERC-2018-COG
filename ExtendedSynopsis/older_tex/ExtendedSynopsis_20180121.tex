%\documentclass[11pt,epsf]{article}
\documentclass[oneside, a4paper, onecolumn, 11pt]{article}

\usepackage{graphicx, amssymb, multicol, amsmath}
\usepackage{fancyhdr, hyperref, sidecap}
%\usepackage[left=2.05cm,top=2.05cm,bottom=1.55cm,right=2.05cm]{geometry}
\usepackage[left=2.05cm,top=2.05cm,right=2.05cm]{geometry}
\usepackage[utf8]{inputenc}
\usepackage{natbib}	        %%  bibliography style
\setlength{\bibsep}{0.0pt}
\usepackage{eurosym}
\usepackage{enumitem}
\usepackage{nopageno}
\usepackage{fancyhdr}
\usepackage[usenames,dvipsnames,svgnames,table]{xcolor}

\usepackage{amsmath, amssymb}
\usepackage{booktabs, bm}           %%  bold math
\usepackage{cancel}
\usepackage{dcolumn}  %%  Align table columns on decimal point
\usepackage{epsfig, epsf, eurosym, enumitem}
\usepackage{fancyhdr}
\usepackage[T1]{fontenc}
\usepackage[para]{footmisc}
\usepackage{graphicx }
%\usepackage{lscape}
\usepackage{hyperref,ifthen}
\usepackage{mathptmx, multicol}
\usepackage[authoryear, round]{natbib}
\usepackage{nopageno}
\usepackage{subfigure}
\usepackage{verbatim}
\usepackage{threeparttable}
\usepackage[usenames,dvipsnames]{xcolor}
\usepackage{tcolorbox}
\usepackage{tabularx}
\usepackage{array}
\usepackage{colortbl}
\usepackage{framed}
\usepackage{todonotes}



%%%%%%%%%%%%%%%%%%%%%%%%%%%%%%%%%%%%%%%%%%%
%       define Journal abbreviations      %
%%%%%%%%%%%%%%%%%%%%%%%%%%%%%%%%%%%%%%%%%%%
\def\nat{Nat} \def\apjl{ApJ~Lett.} \def\apj{ApJ}
\def\apjs{ApJS} \def\aj{AJ} \def\mnras{MNRAS}
\def\prd{Phys.~Rev.~D} \def\prl{Phys.~Rev.~Lett.}
\def\plb{Phys.~Lett.~B} \def\jhep{JHEP}
\def\npbps{NUC.~Phys.~B~Proc.~Suppl.} \def\prep{Phys.~Rep.}
\def\pasp{PASP} \def\aap{Astron.~\&~Astrophys.} \def\araa{ARA\&A}
\def\jcap{\ref@jnl{J. Cosmology Astropart. Phys.}} 
\def\nar{New~A.R.} \def\aapr{A\&ARv}

\newcommand{\preep}[1]{{\tt #1} }

%%%%%%%%%%%%%%%%%%%%%%%%%%%%%%%%%%%%%%%%%%%%%%%%%%%%%
%              define symbols                       %
%%%%%%%%%%%%%%%%%%%%%%%%%%%%%%%%%%%%%%%%%%%%%%%%%%%%%
\def \Mpc {~{\rm Mpc} }
\def \Om {\Omega_0}
\def \Omb {\Omega_{\rm b}}
\def \Omcdm {\Omega_{\rm CDM}}
\def \Omlam {\Omega_{\Lambda}}
\def \Omm {\Omega_{\rm m}}
\def \ho {H_0}
\def \qo {q_0}
\def \lo {\lambda_0}
\def \kms {{\rm ~km~s}^{-1}}
\def \kmsmpc {{\rm ~km~s}^{-1}~{\rm Mpc}^{-1}}
\def \hmpc{~\;h^{-1}~{\rm Mpc}} 
\def \hkpc{\;h^{-1}{\rm kpc}} 
\def \hmpcb{h^{-1}{\rm Mpc}}
\def \dif {{\rm d}}
\def \mlim {m_{\rm l}}
\def \bj {b_{\rm J}}
\def \mb {M_{\rm b_{\rm J}}}
\def \mg {M_{\rm g}}
\def \mi {M_{\rm i}}
\def \qso {_{\rm QSO}}
\def \lrg {_{\rm LRG}}
\def \gal {_{\rm gal}}
\def \xibar {\bar{\xi}}
\def \xis{\xi(s)}
\def \xisp{\xi(\sigma, \pi)}
\def \Xisig{\Xi(\sigma)}
\def \xir{\xi(r)}
\def \max {_{\rm max}}
\def \gsim { \lower .75ex \hbox{$\sim$} \llap{\raise .27ex \hbox{$>$}} }
\def \lsim { \lower .75ex \hbox{$\sim$} \llap{\raise .27ex \hbox{$<$}} }
\def \deg {^{\circ}}
%\def \sqdeg {\rm deg^{-2}}
\def \deltac {\delta_{\rm c}}
\def \mmin {M_{\rm min}}
\def \mbh  {M_{\rm BH}}
\def \mdh  {M_{\rm DH}}
\def \msun {M_{\odot}}
\def \z {_{\rm z}}
\def \edd {_{\rm Edd}}
\def \lin {_{\rm lin}}
\def \nonlin {_{\rm non-lin}}
\def \wrms {\langle w_{\rm z}^2\rangle^{1/2}}
\def \dc {\delta_{\rm c}}
\def \wp {w_{p}(\sigma)}
\def \PwrSp {\mathcal{P}(k)}
\def \DelSq {$\Delta^{2}(k)$}
\def \WMAP {{\it WMAP \,}}
\def \cobe {{\it COBE }}
\def \COBE {{\it COBE \;}}
\def \HST  {{\it HST \,\,}}
\def \Spitzer  {{\it Spitzer \,}}
\def \ATLAS {VST-AA$\Omega$ {\it ATLAS} }
\def \BEST   {{\tt best} }
\def \TARGET {{\tt target} }
\def \TQSO   {{\tt TARGET\_QSO}}
\def \HIZ    {{\tt TARGET\_HIZ}}
\def \FIRST  {{\tt TARGET\_FIRST}}
\def \zc {z_{\rm c}}
\def \zcz {z_{\rm c,0}}


\newcommand{\sqdeg}{deg$^{-2}$}
\newcommand{\lya}{Ly$\alpha$\ }
%\newcommand{\lya}{Ly\,$\alpha$\ }
\newcommand{\lyaf}{Ly\,$\alpha$\ forest}
%\newcommand{\eg}{e.g.~}
%\newcommand{\etal}{et~al.~}
\newcommand{\cii}{C\,{\sc ii}\ }
\newcommand{\ciii}{C\,{\sc iii}]\ }
\newcommand{\civ}{C\,{\sc iv}\ }
\newcommand{\SiIV}{Si\,{\sc iv}\ }
\newcommand{\mgii}{Mg\,{\sc ii}\ }
\newcommand{\feii}{Fe\,{\sc ii}\ }
\newcommand{\feiii}{Fe\,{\sc iii}\ }
\newcommand{\caii}{Ca\,{\sc ii}\ }
\newcommand{\halpha}{H\,$\alpha$\ }
\newcommand{\hbeta}{H\,$\beta$\ }
\newcommand{\oi}{[O\,{\sc i}]\ }
\newcommand{\oii}{[O\,{\sc ii}]\ }
\newcommand{\oiii}{[O\,{\sc iii}]\ }
\newcommand{\heii}{[He\,{\sc ii}]\ }
\newcommand{\nii}{N\,{\sc ii}\ }
\newcommand{\nv}{N\,{\sc v}\ }

%% From:: /cos_pc19a_npr/LaTeX/proposals/JWST/JWST_ERS/Proposal/lines.tex
%%  
\newcommand{\imw}{$i$--$W3$}
\newcommand{\imwf}{$i$--$W4$}
\newcommand{\rmwf}{$r$--$W4$}
\newcommand{\imwt}{$i$--$W2$}
\newcommand{\wtmwf}{$W3$--$W4$}
%\newcommand{\kms}{km s$^{-1}$}
\newcommand{\cmN}{cm$^{-2}$}
\newcommand{\cmn}{cm$^{-3}$}
%\newcommand{\msun}{M$_{\odot}$}
\newcommand{\lsun}{L$_{\odot}$}
\newcommand{\lam}{$\lambda$}
\newcommand{\mum}{$\mu$m}
\newcommand{\ebv}{$E(B$$-$$V)$}
%\newcommand{\heii}{\mbox{He\,{\sc ii}}}
\newcommand{\cv}{\mbox{C\,{\sc v}}}
%\newcommand{\civ}{\mbox{C\,{\sc iv}}}
%\newcommand{\ciii}{\mbox{C\,{\sc iii}}}
%\newcommand{\cii}{\mbox{C\,{\sc ii}}}
%\newcommand{\nv}{\mbox{N\,{\sc v}}}
\newcommand{\niv}{\mbox{N\,{\sc iv}}}
\newcommand{\niii}{\mbox{N\,{\sc iii}}}
%\newcommand{\oi}{\mbox{O\,{\sc i}}}
%\newcommand{\oii}{\mbox{O\,{\sc ii}}}
%\newcommand{\oiii}{\mbox{[O\,{\sc iii}]}}
\newcommand{\oiv}{\mbox{O\,{\sc iv}}}
\newcommand{\ov}{\mbox{O\,{\sc v}}}
\newcommand{\ovi}{\mbox{O\,{\sc vi}}}
\newcommand{\ovii}{\mbox{O\,{\sc vii}}}

%\newcommand{\feii}{\mbox{Fe\,{\sc ii}}}
%\newcommand{\feiii}{\mbox{Fe\,{\sc iii}}}
%\newcommand{\mgii}{\mbox{Mg\,{\sc ii}}}
\newcommand{\neii}{[Ne\,{\sc ii}]\ }
\newcommand{\neiii}{[Ne\,{\sc ii}]\ }
\newcommand{\nev}{Ne\,{\sc v}\ }
\newcommand{\nevi}{[Ne\,{\sc vi}]\ }
\newcommand{\neviii}{\mbox{Ne\,{\sc viii}}}
\newcommand{\aliii}{\mbox{Al\,{\sc iii}}}
\newcommand{\siii}{\mbox{Si\,{\sc ii}}}
\newcommand{\siiii}{\mbox{Si\,{\sc iii}}}
\newcommand{\siiv}{\mbox{Si\,{\sc iv}}}
%\newcommand{\lya}{\mbox{Ly$\alpha$}}
%\newcommand{\lyb}{\mbox{Ly$\beta$}}
\newcommand{\hi}{\mbox{H\,{\sc i}}}
\newcommand{\snine}{\mbox{[S\,{\sc ix}]}}
\newcommand{\sivi}{\mbox{[Si\,{\sc vi}]}}
\newcommand{\sivii}{\mbox[{Si\,{\sc vii}]}}
\newcommand{\siix}{\mbox{[Si\,{\sc ix}]}}
\newcommand{\six}{\mbox{[Si\,{\sc x}]}}
\newcommand{\sixi}{\mbox{[Si\,{\sc xi}]}}
\newcommand{\caviii}{\mbox{[Ca\,{\sc viii}]}}
\newcommand{\arii}{\mbox{[Ar\,{\sc ii}]}}

%%[Ar II] 6.97
%% [S IX] 1.252 μm 328 
% [Si X] 1.430 μm 351 
% [Si XI] 1.932 μm 401 
% [Si VI] 1.962 μm 167 
% [Ca VIII] 2.321 μm 128 
% [Si VII] 2.483 μm 205 
% [Si IX] 3.935 μm 303
% [Ar II] 6.97


%\snine\ at 1.252$\mu$m, \six\ at 1.430$\mu$m, \sixi\ at 1.932$\mu$m, \sivi\ at
%1.962$\mu$m, \caviii\ at 2.321$\mu$m, \sivi\ at 2.483$\mu$m \siix\ at
%3.935$\mu$m and \arii\ at 6.97$\mu$m. 
%%
%% such as [Ne ii]12.8 μm, [Ne v]14.3 μm, [Ne iii]15.5 μm, [S iii]18.7 μm and 33.48 μm, [O iv]25.89 μm and [Si ii]34.8 μm (e.g
%%
%% MIR emission lines like [NeII] and [NeV] are ..
%%
%% Also,  arXiv:astro-ph/0003457v1 
%% [NeV] 14.32um & 24.32um and [NeVI] 7.65um imply an A(V)>160 towards the NLR...
%% [NeIII]15.56um/[NeII]12.81um
%%
%% [Ne V] 14.3, 24.2 μm 97.
%% [Ne II] 12.8 μm
%% [OIV] 26μm
%%


\tcbuselibrary{skins}
\newcolumntype{Y}{>{\raggedleft\arraybackslash}X}

\tcbset{tab1/.style={enhanced, fonttitle=\bfseries, fontupper=\normalsize\sffamily,
colback=yellow!10!white,
colframe=red!50!black,
colbacktitle=Cerulean!40!white,
coltitle=black,center title}
%subtitle style={boxrule=0.4pt, colback=yellow!50!red!25!white} 
}

%% To fix list things: 
\setitemize{noitemsep,topsep=0pt,parsep=0pt,partopsep=0pt,leftmargin=*}
\renewcommand{\labelitemi}{\tiny$\blacksquare$}

\pagestyle{fancy}
\renewcommand{\headrulewidth}{0pt}  %% Remove line at top

%\pagestyle{empty}
\fancyhf{}
%\lhead{{\it ERC-2018-CoG}}
%\lhead{{\it DEQUASARS: Part B1 }}
\lhead{{\it Ross}}
\chead{{\it MIQSOs}}
\rhead{Part B1}
\setcounter{page}{1}
\lfoot{{\it ERC-2018-CoG}}
\rfoot{{\it Extended Synopsis}}
\cfoot{{\it Page \thepage\ of 5}}
%\rfoot{{\it FP7-PEOPLE-2013-IIF}}

\newenvironment{itemize*}%
  {\begin{itemize}%
    \setlength{\itemsep}{0pt}%
    \setlength{\parskip}{0pt}}%
  {\end{itemize}}


\begin{document}

%\begin{center}
% {\Large \bf \textcolor{Cerulean}{Data Science at the Edge of the Universe: Using Quasars to  \\}}
%\vspace{4pt} 
%%  {\Large \bf \textcolor{Cerulean}{the new field of Extragalactic Variable Astrophysics} }
 % {\Large \bf \textcolor{Cerulean}{kickstart  the new field of Extragalactic Time-Domain Astrophysics} }
%\end{center}

%\begin{quotation}
%\noindent
%{\it 
%We will kickstart the new field of Extragalactic Time-Domain Astrophysics by building on our team's experience of using novel observational techniques to study accreting black holes in the early Universe. We will do this by utilizing and combining the firehose of data from several new large surveys that will start to come online from late 2019 onwards.  In doing so, we will learn about one of the two fundamental energy sources available to galaxies (accretion onto the central supermassive black hole) and perform the observational tests to gather the evidence to distinguish between galaxy evolution models and theory. We will also be best positioned to discover totally new extragalactic phenomena. 
%}
%\end{quotation}

\smallskip
\smallskip
\noindent
%{\bfseries \large \textcolor{Cerulean}{Overview}}
\section{\textcolor{Cerulean}{Overview}}
Where do galaxies come from? How do black holes form and grow? And
what is the history, and fate, of the Universe?  These are the
deepest, most fundamental questions in astrophysics and cosmology, and
sets the scene for this proposal.

\smallskip
\smallskip
\noindent
{\it Black holes} are omnipresent in our Universe, and black holes that are
millions to billions of times the mass of our Sun, are ubiquitously
found at the centers of galaxies, including our own Milky Way.
Initially consider physical oddities, we now strongly suspect that
these central, ``supermassive'' black holes have a profound
affect on the galaxies that they live in. This is not surprising since
the potential energy associated with mass accretion onto a
supermassive black hole is comparable to that generated via the
nuclear fusion in the galaxy's stars.
%%
However, the interaction and the physical processes involved in how
this energy escapes the inner most regions of the galaxy and then
interacts with the gas, dust, stars and dark matter, is currently very
poorly understood theoretically, with observational data giving little 
insight on how to make key progress.

\smallskip
\smallskip
\noindent
\underline{{\it The field is poised for a fundamental and rapid change.}} The first data
are now in hand that show changes on human timescales in external
galaxies, with these new field-defining studies including projects led
by the P.I.  Moreover, a fleet of telescopes, detectors and missions
are about to come online over the next few years that will 
leap-frog the quality and quantity of data we have available
today. Over the course of the next 5-6 years, surveys and missions including
SDSS-V, LSST, DESI, 4MOST and ESA {\it Euclid} will see first light. Even more
imminent is the launch of the {\it James Webb Space Telescope}. 


\smallskip
\smallskip
\noindent
{\bf This proposal has two broad but well-posed goals.}
First, we aim to elucidate, for the first time, how the energy directly associated with a 
supermassive black holes impacts the universal galaxy population.  
%%
Second, will open up and explore the Variable Extragalactic Universe, bringing to bear 
the slew of new larget format ``synoptic'' telescopes. 
%CLQs, TDEs, Binary BHs, binary SMBHs... 
Things will go ``bang in the middle of the night''; we just don't know what 
they are yet. 
%
%% NS-NS merger (EM signatures in the blue...) ...
%% Unify the four fundamental forces of nature...

\smallskip
\smallskip
\noindent
We will achieve this by leveraging several of the new, large-scale
surveys that are coming online in the next few years.  
%%
The goal is to connect the physical mechanisms from sub-parsec to
cosmological scales, and to investigate the physical processes that
link luminous AGN activity and the formation and evolution of massive
galaxies. These critical observations are made by exploiting the large
imaging and spectroscopic datasets that we will have available from
the SDSS-V, DESI, 4MOST, LSST and ESA Euclid.
%%
We ask for the personnel to accomplish these vey ambituous, but
achievable goals, along with the `buy-in' to the facilites we need
access to.  We also ask for the computing and travel support that will
directly enable this research.

\smallskip
\smallskip
\noindent
{\it The scope and
remit of an ERC Consolidator grant will allow us to combine these data
products in a manner that will not only establish the new
state-of-the-art in extragalactic time-domain science, it will
establish and kickstart the new field of extragalactic  time-domain 
science itself.}

\smallskip
\smallskip
\noindent
Funding this ERC Consolidator grant proposal will radically improve our understanding of 
one of the two fundamental energy sources available to galaxies; that of accretion 
onto the compact object in the central engine. 
%%
The P.I. is a world-leader in observational quasar astrophysics, both in terms of 
survey work and individual object study. 
%%
Our proposal takes astrophysics into the 2020s, going from single objects samples, 
to surveys and samples of millions of objects leveraging these multi-billion Euro/dollar/pound  
next generation missions, telescopes and their subsequent datasets. 

\smallskip
\smallskip
\noindent
\section{\textcolor{Cerulean}{Scientific Background and Motivation}}
%{\bf \underline {Background:}}
%{\bfseries \large \textcolor{Cerulean}{Scientific Background:}}
%{\bfseries \underline{\textcolor{Cerulean}{Background:}}}
%\textbf{\textsc{Global Motivation:}}
Quasars are extremely bright extragalactic objects powered by gas falling onto supermassive black holes. 
They are detected in optical surveys, both via their photometric and spectra properties. 
Accretion onto a supermassive black hole is one of two major energy sources available to a galaxy. 
However, we do not know how this energy espaces the central regions where it is generated, and 
how it interacts with the galaxy at large. 

\smallskip
\smallskip
\noindent
In the Table below, we summarise the outstanding issues, and our novel investigations. 

\smallskip
\smallskip
%%%%%%%%%%%%%%%%%%%%%%%%%%%%%%%%%%%%%%%%%%%%%%%%%%%%%%%%%%%%%%%%%%%%%%%%%%%%%%%%
%%
%%  https://tex.stackexchange.com/questions/337820/mcq-long-table-using-tikz-tcolorbox-or-tabular
%%  https://tex.stackexchange.com/questions/283419/color-in-a-multirow-cell-with-extra-vertical-space/283454
%%  https://tex.stackexchange.com/questions/406033/how-to-fit-a-cell-of-a-table-to-a-figure-and-arrange-multiple-tables/406042
%% 
%% THIS (??)::
%%     https://texblog.org/2014/05/19/coloring-multi-row-tables-in-latex/
%%
%%
%%   https://www.inf.ethz.ch/personal/markusp/teaching/guides/guide-tables.pdf
%%
%%%%%%%%%%%%%%%%%%%%%%%%%%%%%%%%%%%%%%%%%%%%%%%%%%%%%%%%%%%%%%%%%%%%%%%%%%%%%%%%


\begin{tcolorbox}[tab1, tabularx={X  X }, title=Outstanding Issues in Extragalactic Astrophysics, boxrule=1.25pt] 
Key issue                                                                            &  Novel investigation       \\ 
\hline \hline
\multicolumn{2}{c}{{\sc The physics of accretion}} \\ 
Investigating ``hot'' and ``cold'' mode accretion in the quasar population; 
determining the rates and timescales, and characterising the Changing Look Quasar (CLQ) population.   &     
Identifying and characterizing  all the CLQs in DESI and SDSS-V.  \\ 
\hline
Probing the inner parsec of the quasar central engine & 
Rapid analysis and response on LSST quasar light curves. \\ 
\hline
%%
\multicolumn{2}{c}{{\sc Obscured accretion and galaxy formation}} \\
Establish the relative importance of major mergers, minor mergers, cold streams and secular evolution 
have towards the growth of SMBHs across cosmic time. & 
Deep imaging data from LSST combined with searching for post-starburst signatures 
in DESI, SDSS-V, 4MOST spectra. NIRcam and MIRI imaging from JWST. \\ \hline
Establishing the bolometric output and origin of IR emission, and  
determine presence of extreme outflows in the $z\sim2-3$ quasar population. & 
MIRI MRS spectroscopy with JWST.\\ \hline
Establishing the range of SED parameter space the quasars occupy by a multi-wavelength multi-epoch ``truth table dataset'' & 
Building ``The Stripe 82 Rosetta Stone'' (SpIES, SHELA, VICS82, S82X, HSC; repeat optical observations from SDSS, DES) \\ \hline
%%
Find the physical conditions under which SMBH grew at the epoch when most of the accretion and star formation in the Universe occurred ($z\sim1-4$) & Perform a complete census of AGN across $z\sim0-7$, focussing on $z=1-4$ using medium-deep multiwavelength datasets \\ \hline
\multicolumn{2}{c}{{\sc Galaxy-scale feedback}}\\
Establishing the theoretical impact of extreme outflows in the $z\sim2-3$ quasar population & 
Hydro simulation modelling.  \\
\hline
Understand how the accretion disks around black holes launch winds and outflows and determine how much energy these carry. 
Quantify the amount of ``Maintenance/Jet/Kinetic'' mode and ``Transition/Radiative/Wind'' mode feedback.
& 
Identifying and characterizing  all the CLQs in DESI and SDSS-V.  \\ 
    %\end{tcbitemize}
\end{tcolorbox}


\smallskip
\smallskip
\noindent
Within the next 5 - 6 years a fleet of new surveys will come online. 

\smallskip
\smallskip
\noindent
%%% European Team Advice:  ERC_CoG_2018_B1_tips.pdf;  page 2.
In ``MIDAS'' we propose to develop a novel $X$ in order to address 
$<$give the fundamental/key questions you are addressing$>$. 
%If you have preliminary results or building on previous work briefly mention here.
The P.I. has new, to be published imminently, results on one object that employs this method. 
We want to make a mega-improvement, literally taking our field from single, to million object 
statistics. Only in that way... 

\smallskip
\smallskip
\noindent
During its first phase of operation from 2000–2005, the SDSS obtained spectra 
of 1 million galaxies in local Universe. This dataset has become the {\it de facto} standard 
for understanding the present day galaxy population, and setting a boundary 
condition for all theoretical comparisons. 
%%
{\it The paradigm changing success of the SDSS was due to it having 1e6 objects. 
We desire the same sample size and revolutionary understanding of the quasar 
population as the SDSS had with the $z\sim0.1$ galaxy population.}

\smallskip
\smallskip
\noindent
The current state-of-the-art has a sample of 1e6 sample of objects with one spectral epoch. 
or only has single objects with the repeat spectra (Ross et al.).
We plan to make sure the  1e6 sample have repeat spectra and in doing so, will 
kickstart the new field of Extragalactic Time-Domain Astrophysics.

\smallskip
\smallskip
\noindent
Challenges and the novel approach you are suggesting: Novel approaches
need to be developed and they are better than because.... Our approach
is based on X and we will be able to do Y. Also give the present state
of the art and where the proposed work will lead.
Goal: our ultimate goal is to...

\smallskip
\smallskip
\noindent
Roadmap: The project will utilise cutting edge.... with .....Together
these experiments will be used to test..... which is a critical unmet
need.... OR if successful the project will..... OR answering these
questions requires an ambitious research programme covering... I now
seek to.....


\section{\textcolor{Cerulean}{The Future is Now: Project in larger context}}
%%  You need to answer 5 key questions: 
%%
%    Why bother? , 
%    What problem are you trying to solve?
%\smallskip
%\smallskip
%\noindent
\textbf{\textsc{\textcolor{Cerulean}{Global Motivation:}}}
At its heart, there are two major motivations for our project. 
The first is to gain a deep understanding into the physical mechanisms 
related to central engine black holes; their accretion disk physics, their 
dynamics on both human and galactic timescales and the role they might 
play in forming, and regulating the galaxy population. These are among the 
most prescient astrophysical questions of our time, and in an area where 
major breakthroughs are imminent. 

\smallskip
\smallskip
\noindent
The importance of this branch of astrophysics is already well establish in 
Europe and is a priority for the next two decades. This is demonstrated by noting that
one of the two primary mission goals for the Advanced Telescope for High-ENergy Astrophysics (ATHENA) is 
answering the question ``How do black holes grow and shape the Universe?''. 
ATHENA is ESA's second L-class flagship mission, due for launch in 2028.

\smallskip
\smallskip
\noindent
The second motivation is the massive, untapped and raw discovery space 
that the new experiments will open up, and the likely outcome of discovering 
something ``brand new''. It is somewhat tricky to say specifically what to 
expect, but the fact that e.g. LSST will deliver a dataset {\it so spectacularly} 
difference both in sky coverage and time-sampling coverage, means the 
Universe would have to be an exceptionally boring (and unkind!) place to 
not have a brand new astronomical object and astrophysical phenomena 
be discovered. 


%% Is this a European priority? , 
%% Could it be solved at a national level? 
\smallskip
\smallskip
\noindent
\textbf{\textsc{\textcolor{Cerulean}{Maximising Science Returns from European priorities:}}}
Contemporary astronomy is a multi-national endeavor with many leading
facilites being international collaborations. Although a project, with
similar but much less ambitious science goals and return could be
envisaged at the national level, the full discovery and break-through
nature being described herein only comes to the fore when the data
from the various international collaborations are combined
intelligently.  Critically data from leading European Southern
Observatory (ESO) and European Space Agency (ESA) facilites will play
a pivotal role here.

\smallskip
\smallskip
\noindent
Although many of the ``building blocks'' for the the solution are
already available (e.g. open source codes, database infrastructures,
the methodology of catalog creation and combination) no one has yet to
combine the data in the way we envisage. Moreover, the datasets we
desire to deliver our paradigm changing science are only coming online
over the next 5-10 years.


\smallskip
\smallskip
\noindent
%% Why now? 
%% What would happen if you did not do this now?, 
\textbf{\textsc{\textcolor{Cerulean}{The Timing and the Team:}}}
{\it The timing for this proposal could not be better or more imperative.} 
The first of the data data ``firehose'' turns on in late 2019, with
the full datastream from our key sources fully online around 2021. 

%% LSST:: Each patch of sky it images will be visited 1000 times during the survey,
%%
%% https://www.youtube.com/watch?v=kpdLDJXEmys
%% https://www.youtube.com/watch?v=ScKuACRkGnM
%%
%% https://pbs.twimg.com/media/DTL8RM3X0AE9by3.jpg

\begin{figure}[h]
  \begin{center}
   \hspace{-0.5cm}
%   trim=l b r t
    \includegraphics[width=16.0cm] %, trim={0.05cm 0 0.05cm 0},clip]
    {figures/Timelines_and_Facilites.pdf}
    \vspace{-10pt}
   \caption{}
  \vspace{-12pt}
 \label{fig:Keynote_facilites}
\end{center}
\end{figure}

\smallskip
\smallskip
\noindent
The model indicates that the first group of people to use a new
product is called ``innovators'', followed by ``early adopters''. 
%%
Interdisciplinary field of scientific methods, processes, and systems
to extract knowledge or insights from data in various forms, either
structured or unstructured.  Data science "concept to unify
statistics, data analysis and their related methods" in order to
"understand and analyze actual phenomena" with data.[3] It employs
techniques and theories drawn from many fields within the broad areas
of mathematics, statistics, information science, and computer science,
in particular from the subdomains of machine learning, classification,
cluster analysis, data mining, databases, and visualization.  {\it
Modern day observational astrophysicists are in all but the term data
scientists, and as such, this proposal is inherently
interdisciplinary.}

\begin{figure}[h]
  \begin{center}
   \hspace{-0.5cm}
%   trim=l b r t
    \includegraphics[width=16.0cm] %, trim={0.05cm 0 0.05cm 0},clip]
    {figures/workplan.pdf}
    \vspace{-10pt}
   \caption{}
  \vspace{-12pt}
 \label{fig:Keynote_facilites}
\end{center}
\end{figure}


\smallskip
\smallskip
\noindent
%% Why you?. 
%% Do you have the best consortium to do this work? 
The P.I. has unmistakably become a world-leader in the field of 
extragalactic quasar observational astrophysics. 
Moreover, however, the University of Edinburgh is now poised to be an 
astronomical data centre nexus, with 

\smallskip
\smallskip
\noindent
%There are very few people in the world that 
The P.I. has built their career on this science case, and has already been a P.I. 
of a Working group team (as part of a collaboration) with prodigious scientific
output (400 published, peer-reviewed papers and counting). 

%%
Given the science goals, the P.I.'s track record and ambition and 
the nature of this project, this proposal satisfies all the aims and goals
of the ERC Con. 
%%
%%
%%
%You should aim to have these addressed in the first paragraph of your proposal! 
%%
%The “WHY NOW” is one of the most important ones for ERC.
%%
%% The ERC panel will evaluate the PI’s “intellectual capacity, creativity and commitment”. 
%% This includes:
%% -- ability to propose and conduct ground-breaking research and
%% -- achievements going beyond the state-of-the-art
%% -- abundant evidence of creative independent thinking
%% -- the ERC grant would contribute significantly to the establishment and/or further consolidation of the PI's independence
%% -- commitment to the project (minimum 50\% of the PI’s total working time)


\smallskip
\smallskip
\noindent
%\section*{\textcolor{Cerulean}{Scientific Background and Motivation}}
\section{\textcolor{Cerulean}{Feasibility, Projects and Methodology}}
%We set out a range of projects in Table~1.

\subsection*{Feasibility}
The extended synopsis should provide a concise presentation for the
scientific proposal, including the feasibility of the project and
paying particular attention to the ground breaking nature of the
proposal and how it may open new horizon or opportunities for
research. Give some background to the project and make sure you
explain the proposed work in the context of the state of the art.

\subsection*{P.I.'s Expereince and Track Record}
The P.I. has an established track record of managing science teams and groups:: 
eg. 

\begin{itemize}
\item I was Chair of the SDSS-III BOSS Quasar Target
Selection team. Once BOSS was successfully running, I became the Chair
of the Quasar Science Working Group. In both roles, I was managing a
group of senior professors, other postdocs and graduate students.

\noindent
The scientific yield from the BOSS Quasar Survey was extremely high,
with highlights including: the first detection of baryon acoustic
oscillations at $z>1$; the first detection of baryon acoustic
oscillations using quasars; the first measurements of ``the knee'' of
the $z\sim2.5-3$ and $z=5$ optical quasar luminosity function, and the
discovery of the first ``Changing Look'' Quasar.  BOSS also allowed
the first identification of ``extremely red'' quasars, which are a
unique obscured quasar population with extreme physical conditions
related to powerful outflows. {\it I either led, or was heavily
involved in, all these projects, and
\href{https://tinyurl.com/ycxd8lb6}{over 400 publications} have used
the BOSS Quasar catalogs}.

\item

\item STFC ERF budget of \euro615,000 on award. 

\end{itemize}

\subsection*{Breaking Down The Data Silos}
The biggest obstacle to using advanced data analysis is not skill base or technology; it is simply access to the data.
%%
A data silo is a repository of fixed data that remains under the
control of one department and is isolated from the rest of the
organization, much like grain in a farm silo is closed off from
outside elements. Data silos can have technical or cultural roots.

%https://hbr.org/2016/12/breaking-down-data-silos
\noindent
There is a cost to using data. Behind the glamor of powerful
analytical insights is a backlog of tedious data preparation. Since
the popular emergence of data science as a field, its practitioners
have asserted that 80\% of the work involved is acquiring and
preparing data. Despite efforts among software vendors to create
self-service tools for data preparation, this proportion of work is
likely to stay the same for the foreseeable future, for a couple of
reasons.

\noindent
But there is a bigger and costlier demon that lurks in enterprises. A
demon that can drive up that 80\% and often makes initiatives
impossible: data silos. These silos are isolated islands of data, and
they make it prohibitively costly to extract data and put it to other
uses. They can arise for multiple reasons.
%%
In commerical enterprises, data remained siloed for many reasons, a
major one being of course for monetary gain.  However, in research
environments, and {\it especially in contemporary observational
astrophysics}, the data siloes are open, but due to the lack of raw
person-power. 


\subsection*{Algorimths and Methodology}
Our algorithms and methodology is based on the latest machine-learning and data science techniques. 
The Extreme deconvolution `XDQSO' technique\href{xDQSO}{http://www.sdss.org/dr14/data\_access/value-added-catalogs/?vac\_id=xdqso/}.\\
GitHub: https://github.com/xdqso/xdqso

\noindent
\href{http://ogrisel.github.io/scikit-learn.org/sklearn-tutorial/index.html}{\tt scikit-learn} is a Python module integrating classic machine learning algorithms in the tightly-knit scientific Python world (numpy, scipy, matplotlib). It aims to provide simple and efficient solutions to learning problems, accessible to everybody and reusable in various contexts: machine-learning as a versatile tool for science and engineering. 

\href{https://github.com/astroML/sklearn\_tutorial}{{\tt https://github.com/astroML/sklearn\_tutorial}}\\
\href{https://github.com/jakevdp/PythonDataScienceHandbook}{{\tt https://github.com/astroML/sklearn\_tutorial}}\\
\href{https://github.com/jakevdp/sklearn\_tutorial}{{\tt https://github.com/jakevdp/sklearn\_tutorial}}\\

\subsection*{Computing Requirements}
As any half-decent data scientist will tell you, with the availability of 
contemporary cloud based services (both commercial and non-for-profit), 
the infrastructure is essentially already in place for all but the most demanding
of compute task. The rate limiting factor, in the vast majority of endevours, is 
the actual data access itself. 

\smallskip
\smallskip
\noindent
The facilites available to me at a institute (e.g. Cullen), university
(e.g. \href{https://www.ed.ac.uk/information-services/research-support/research-computing/ecdf}{``Edinburgh
Compute and Data Facility''} and at a national
{\href{https://www.hartree.stfc.ac.uk/Pages/home.aspx}{The Hartree Centre} level will all be more than sufficient and utilized.
The rate limiting factor will be how quickly and efficiently we can deploy our codes, 
and analysis, i.e. person-power. 


\smallskip
\smallskip
\noindent
If we are to understand galaxy formation and evolution, we have to
understand the two major power sources available to a galaxy; nuclear
fusion in stars and accretion onto compact objects. Our proposal
directly addresses the latter process. This is a worldwide research
priority, but is also a European priority, due to the investment in
the ESA {\it Euclid} and {\it Gaia} missions, and the developments of
new ESO telescopes (ELT) and instrumentation (4MOST).  (e.g.
{\bf Also connection to gravitational waves!!!} 

%\smallskip
%\smallskip
%\noindent
%Why (the ERCs)??\\
%World class science is the foundation of tomorrow’s technologies, jobs and wellbeing\\
%Europe needs to develop, attract and retain research talent\\
%Researchers need access to the best infrastructures\\

\section*{\textcolor{Cerulean}{Open Innovation, Open Science, Open to the World}}
%\section*{Open Innovation, Open Science, Open to the World}
Every part of the scientific method is becoming an open, collaborative and participative process. 
The P.I. is an exceptionally strong, longtime and vocal supporter of ``Open Access''. 
All my codes, data\footnote{where I am not breaking current data access agreements}, papers 
and proposals can be found at \href{github.com/d80b2t}{{\tt github.com/d80b2t}}. 
Indeed, this proposal itself is now at that location. 

One of the major research outputs 
of this ERC will be computer code. 
As such, we are already working with the
\href{The Software Sustainability Institute}{\tt https://www.software.ac.uk/}
which was founded to support the UK’s research software community. 
Our software well be developed using the FAIR ideology (Findable, Accessible, Interoperable, Reusable
[10]
%Wilkinson, MD, Sci Data. 2016 Mar 15;3:160018. doi: 10.1038/sdata.2016.18.
) 
and will be delivered in a manner which is fully inline 
with ``Open Innovation, Open Science, Open to the World''. 





\section{The Future}
Cosmic Vision Themes:: The Hot and Energetic Universe \\
Mapping hot gas structures and determining their physical properties \\
Searching for supermassive black holes \\ 
Athena – Advanced Telescope for High-ENergy Astrophysics – will be an X-ray telescope designed to address the Cosmic Vision science theme 'The Hot and Energetic Universe'. The theme poses two key astrophysical questions:
%%
How does ordinary matter assemble into the large-scale structures we see today? and
How do black holes grow and shape the Universe?
%%
Goes towards ESA {\it Athena}. ; 




\newpage
\begin{center}
\medskip
 \medskip
 {\large \bf References}
    \vspace{-10pt}
\end{center}
\begin{multicols}{2}[]
\noindent
%\footnotesize
%\scriptsize
%\tiny
\lbrack 1\rbrack Kormendy \& Ho, 2013, ARAA, 51, 511\\
\lbrack 2\rbrack Kormendy,  2016, ASSL, 418, 431\\
\lbrack 3\rbrack Alexander et al., 2012, NewAR, 56, 93\\
\lbrack 4\rbrack King \& Pounds, 2015, ARAA, 53, 115 \\
\lbrack 5\rbrack Heckman \& Best, 2014, ARAA, 52, 589\\
\lbrack 6\rbrack Naab \& Ostriker, 2017, ARAA, 55, 59 \\
\lbrack 7\rbrack Netzer, 2015, ARAA, 53,  365\\
\lbrack 8\rbrack Padovani, 2017, A\&ARv, 25, 2\\
%%
\lbrack  9\rbrack Ata et al., 2017, arXiv1705.06373v2\\
\lbrack10\rbrack Solsar et al., 2013, JCAP, 04, 026 \\
\lbrack11\rbrack Busca et al.,  2013, A\&A, 552, 96 \\
\lbrack12\rbrack Delubac et al.,  2015, A\&A, 574, 59 \\
\lbrack13\rbrack Bautista et al., 2017, A\&A, 603, 12 \\
\lbrack14\rbrack du Mas des Bourboux et al., 2017, arXiv1708.02225v3\\
\lbrack15\rbrack Font-Riber et al., 2014, JCAP, 05, 027\\
%%
\lbrack16\rbrack Ross et al. 2015, MNRAS, 453, 3932\\
\lbrack17\rbrack Wright et al., 2010, AJ, 140, 1868\\
\lbrack18\rbrack Zakamska et al., 2016, MNRAS, 459, 3144\\
\lbrack19\rbrack Hamann et al., 2017, MNRAS, 464, 3431\\
%%
\lbrack20\rbrack Timlin, Ross et al., 2016, ApJS, 225, 1\\
\lbrack21\rbrack Timlin, Ross et al., 2017, ApJ, {\it in prep.}\\
%%
\lbrack22\rbrack LaMassa et al., 2015, ApJ, 800, 144\\
\lbrack23\rbrack Runnoe et al., 2016, MNRAS, 455, 1691\\
\lbrack24\rbrack Ruan et al, 2016, ApJ, 826, 188\\
\lbrack25\rbrack MacLeod, Ross et al., 2016, MNRAS, 457, 389\\
%%
\lbrack26\rbrack Meisner et al., 2017, AJ, 153, 38 \\
\lbrack27\rbrack Meisner et al., 2017, AJ, 154, 161 \\
% Meisner  et al., 2017, arXiv1710.02526v1 \\
\lbrack28\rbrack Ross et al., 2017, Nat.As., {\it in prep.} \\
%%
\lbrack29\rbrack Hopkins et al., 2006, ApJS, 163, 1\\
%%
\lbrack30\rbrack Schlegel et al., 2011,  arXiv:1106.1706v2 \\
%
\lbrack31\rbrack Ross et al., 2009, ApJ, 697, 1634 \\
%%
\lbrack32\rbrack Lombriser \& Taylor, 2016, JCAP, 03, 031 \\
\lbrack33\rbrack Baker et al, 2017,  arXiv1710.06394v1 \\
%%
\lbrack34\rbrack Watson et al.,  2011, ApJ, 740, L49\\
\lbrack35\rbrack King et al., 2014, MNRAS, 441, 3454\\
\lbrack36\rbrack King et al., 2015, MNRAS, 453, 1701\\
\lbrack37\rbrack Shen et al., 2015, ApJS, 216, 4\\
%\lbrack37\rbrack The Pierre Auger Collaboration, 2017, Science, 357, 6357 \\
\lbrack38\rbrack Hviding et al.,  2017, arXiv1711.01269v1 



\end{multicols}



\end{document}

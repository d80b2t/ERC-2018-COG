%\documentclass[11pt,epsf]{article}
\documentclass[oneside, a4paper, onecolumn, 11pt]{article}

\usepackage{graphicx, amssymb, multicol, amsmath}
\usepackage{fancyhdr, hyperref, sidecap}
%\usepackage[left=2.05cm,top=2.05cm,bottom=1.55cm,right=2.05cm]{geometry}
\usepackage[left=2.05cm,top=2.05cm,right=2.05cm]{geometry}
\usepackage[utf8]{inputenc}
\usepackage{natbib}	        %%  bibliography style
\setlength{\bibsep}{0.0pt}
\usepackage{eurosym}
\usepackage{enumitem}
\usepackage{nopageno}
\usepackage{fancyhdr}


\input{format}

\tcbuselibrary{skins}
\newcolumntype{Y}{>{\raggedleft\arraybackslash}X}

\tcbset{tab1/.style={enhanced, fonttitle=\bfseries, fontupper=\normalsize\sffamily,
colback=yellow!10!white,
colframe=red!50!black,
colbacktitle=Cerulean!40!white,
coltitle=black,center title}
%subtitle style={boxrule=0.4pt, colback=yellow!50!red!25!white} 
}

%% To fix list things: 
\setitemize{noitemsep,topsep=0pt,parsep=0pt,partopsep=0pt,leftmargin=*}
\renewcommand{\labelitemi}{\tiny$\blacksquare$}

\pagestyle{fancy}
\renewcommand{\headrulewidth}{0pt}  %% Remove line at top

%\pagestyle{empty}
\fancyhf{}
%\lhead{{\it ERC-2018-CoG}}
%\lhead{{\it DEQUASARS: Part B1 }}
\lhead{{\it Ross}}
\chead{{\it }}
\rhead{Part B1}
\setcounter{page}{1}
\lfoot{{\it ERC-2018-CoG}}
\rfoot{{\it Extended Synopsis}}
\cfoot{{\it Page \thepage\ of 5}}
%\rfoot{{\it FP7-PEOPLE-2013-IIF}}

\newenvironment{itemize*}%
  {\begin{itemize}%
    \setlength{\itemsep}{0pt}%
    \setlength{\parskip}{0pt}}%
  {\end{itemize}}


\begin{document}


\smallskip
\smallskip
\noindent
{\bf{\textcolor{Cerulean}{a. Extended Synopsis}}} 
\vspace{6pt}

\noindent
%\Huge \huge \LARGE \Large \large \normalsize (default) \small \footnotesize \scriptsize \tiny
\large
{\bf{\textcolor{Cerulean}{Overview and Objectives}}}
\normalsize


\smallskip
\smallskip
\noindent
Black holes are omnipresent in our Universe, and black holes that are
millions to billions of times the mass of our Sun, are ubiquitously
found at the centers of galaxies, including our own Milky Way.
Current theories of galaxy formation and evoution now strongly suggest
that these central, ``supermassive'' black holes have a profound
affect on the galaxies that they live in. This is not surprising since
the potential energy associated with mass accretion onto a
supermassive black hole is comparable to that generated via the
nuclear fusion in the galaxy's stars. However, the interaction and the
physical processes involved in how this energy escapes the inner most
regions of the galaxy and then interacts with the gas, dust, stars and
dark matter, is currently very poorly understood theoretically, with
observational data giving little insight on how to make key progress.

\smallskip
\smallskip
\noindent
The field of observational extragalactic astrophysics is poised for a
fundamental and rapid change. The first data are now in hand that show
changes on {\it human timescales} in external galaxies, with these new
field-defining studies including projects led by the P.I.  Moreover, a
fleet of new telescopes, instruments and missions are about to come
online over the next few years that will leap-frog the quality and
quantity of data we have available today. Over the course of the next
5-6 years, surveys and missions including the fifth incranation of the
Sloan Digitial Sky Survey
(SDSS-V\footnote{\href{www.sdss.org/future/}{{\tt
www.sdss.org/future/}}}), the Large Synoptic Survey Telescope
(LSST\footnote{\href{lsst.org}{{\tt lsst.org}}}), the Dark Energy
Spectroscopit Instrument (DESI\footnote{\href{desi.lbl.gov}{{\tt
desi.lbl.gov}}}) survey, the 4-meter Multi-Object Spectroscopic
Telescope (4MOST\footnote{\href{4most.eu}{{\tt 4most.eu}}}) survey,
and the ESA {\it Euclid}
mission\footnote{\href{sci.esa.int/euclid/}{{\tt
sci.esa.int/euclid/}}}, will see first light. Even more imminent is
the launch of the {\it James Webb Space Telescope}
(JWST\footnote{\href{jwst.stsci.edu}{{\tt jwst.stsci.edu}}}).

\smallskip
\smallskip
\noindent
This proposal has two broad and well-posed goals.  First, we aim to
elucidate, for the first time, how the energy directly associated with
a supermassive black holes impacts the universal galaxy population.
This will open up and explore the Variable Extragalactic Universe,
bringing to bear the slew of new larget format ``synoptic''
telescopes.  The goal is to connect the physical mechanisms from
sub-parsec to cosmological scales, and to investigate the physical
processes that link luminous AGN activity and the formation and
evolution of massive galaxies.  Second, we will {\it discover brand new
extragalactic phenomena.}  

\smallskip
\smallskip
\noindent
We will achieve this by leveraging several of the new, large-scale
surveys that are coming online in the next few years. These critical
observations are made by exploiting the large imaging and
spectroscopic datasets that we will have available from the SDSS-V,
DESI, 4MOST, LSST and ESA {\it Euclid}. 




\medskip
\medskip
\noindent
\large
{\bf{\textcolor{Cerulean}{1. State of the Art.}}}
\normalsize

\noindent
The current state-of-the-art data samples have $\approx$10$^{6}$
objects with one spectral epoch, or only $\sim$a few objects with
repeat spectra (e.g, MacLeod, Ross et al. 2016; Ross et al. 2018).  We
plan to collate datasets so that the $\approx$10$^{6}$ sample have
light-curves and repeat spectra and in doing so, will kickstart the
new field of Extragalactic Time-Domain Astrophysics.

\smallskip
\smallskip
\noindent
During its first phase of operation (2000–05), the Sloan Digital Sky
Survey (SDSS) obtained spectra of 1 million galaxies in the local
Universe. This dataset has become the {\it de facto} standard for
understanding the present day galaxy population, and setting a
boundary condition for all theoretical comparisons.  {\it The paradigm
changing success of the SDSS was due to it having 1,000,000 objects.
We desire the same sample size and revolutionary understanding of the
quasar population as the SDSS had with the $z\sim0.1$ galaxy
population.}  Our proposal takes astrophysics into the 2020s, going
from single objects samples, to surveys and samples of millions of
objects leveraging these very large scale next generation missions,
telescopes and their subsequent datasets.

\smallskip
\smallskip
\noindent
At its heart, there are two major motivations for our project.  The
first is to gain a deep understanding into the physical mechanisms
related to central engine black holes; their accretion disk physics,
their dynamics on both human and galactic timescales and the role they
might play in forming, and regulating the galaxy population. These are
among the most prescient astrophysical questions of our time, and in
an area where major breakthroughs are imminent.

\smallskip
\smallskip
\noindent
\textbf{\textsc{\textcolor{Cerulean}{Timing:}}}
{\it The timing for this proposal could not be better or more imperative.} 
The first of the data ``firehoses'' turns on in late 2019, with
the full datastream from our key sources fully online around 2022. 
As such, with two years to use existing datasets as testbeds, we 
have the time to ramp-up our efforts, while also being able to 
take advantage of the initial data releases of all these new projects. 
%% LSST:: Each patch of sky it images will be visited 1000 times during the survey,
%%
%% https://www.youtube.com/watch?v=kpdLDJXEmys
%% https://www.youtube.com/watch?v=ScKuACRkGnM
%%
%% https://pbs.twimg.com/media/DTL8RM3X0AE9by3.jpg

\begin{figure}[h]
  \begin{center}
   \hspace{-0.5cm}
%   trim=l b r t
    \includegraphics[width=16.0cm] %, trim={0.05cm 0 0.05cm 0},clip]
    {figures/Timelines_and_Facilites.pdf}
    \vspace{-10pt}
   \caption{}
  \vspace{-12pt}
 \label{fig:Keynote_facilites}
\end{center}
\end{figure}


\smallskip
\smallskip
\noindent
The importance of this branch of astrophysics is already well
establish in Europe and is a priority for the next two decades. This
is demonstrated by noting that one of the two primary mission goals
for the Advanced Telescope for High-ENergy Astrophysics (ATHENA) is
answering the question ``How do black holes grow and shape the
Universe?''.  ATHENA is ESA's second L-class flagship mission, due for
launch in 2028.

\smallskip
\smallskip
\noindent
The second motivation is the massive, untapped and raw discovery space
that the new experiments will open up, and the highly likely outcome
of discovering something ``brand new''. It is somewhat tricky to say
specifically what to expect, but the fact that e.g. LSST will deliver
a dataset {\it so spectacularly} different both in sky coverage and
time-sampling coverage, means the Universe would have to be an
exceptionally boring place to not have brand new astronomical objects
and astrophysical phenomena be discovered.

\smallskip
\smallskip
\noindent
{\it The scope and
remit of an ERC Consolidator grant will allow us to combine these data
products in a manner that will not only establish the new
state-of-the-art in extragalactic time-domain science, it will
establish and kickstart the new field of extragalactic time-domain 
science itself.}





\medskip
\medskip
\noindent
\large
{\bf{\textcolor{Cerulean}{2. Methodology}}}
\normalsize

\noindent
Our proposal contains eight work packages that fall into three broad,
complemenraty catagories: observational studies of large numbers
(millons) of objects; high-risk, very high-reward observational
studies of a small number (10s) of objects; theoretical modeling
investigations. Risks and mitigation strategies are present for each
WP.


\smallskip
\smallskip
\noindent
\textbf{\textsc{WP1: Build an Event Broker:}} 
The LSST will deliver three levels of data products and
services. ``Level 1'' are the nightly data products and their primary
purpose is to enable rapid follow-up of time-domain events. ``Level
2'' data products are annual and will include well calibrated images
and catalogues. ``Level 3'' are the user-created data product services
that will enable science cases that greatly benefit from co-location
of user processing and/or data within the LSST Archive Center. In
order to access the LSST data for our science needs we will need to
build an {\it event broker}, an intermediary program module that
interacts with primarily the ``Level 3'' data products from the LSST.

\noindent
{\bf WP1 is low-risk, high-reward.} 
The goal of this WP is to build an Event Broker.  The heavy-industry
computing infrastruture is already being supplied by the LSST Data
Access Center and LSST Corporation. Our task will be to build the
event broker in a timely and robust manner. At the effort level of one
PDRA (``PDRA 1'') and a substatioal committment from the P.I., (NPR)
along with the key personel and algorthim resources at the Royal
Observatory, Edinburgh, there is no element of this which can be
deemed high-risk.


\smallskip
\smallskip
\noindent
\textbf{\textsc{WP2: Quasar Catalogue Generation:}} 
Building the quasar corpus and cataloguing the observational data will
be a large, but vital step in beginning to pursue our science
goals. This catalogue will be the glue that binds the observational
projects together and will have not only the data, but moreover the
metadata to enable the other WPs.

\noindent
{\bf WP2 is low-risk, high-reward.}
The goal of this WP is to construct a quasar catalogue.
This is a full WP, and with the P.I.s (NPR) experience at this
specific task, plus the effort level of one PDRA (``PDRA 2'') there is
no element of this which can be deemed high-risk.


\smallskip
\smallskip
\noindent
\textbf{\textsc{WP3: Light-Curve and Spectral Analyses:}} 
Following on from the quasar corpus catalogue generation, one key
science output will be the full and detailed light-curve and spectral
analyses of the said catalogue. This will result in the discovery of
light-curve trends with quasar type, new methods to measure black hole
mass and the relatively easy check to see which quasars have become
``changing-look'' objects. This WP will also have a data science/machine learning 
aspect.

\noindent
{\bf WP3 is low-risk, high-reward.} 
The goal of this WP is to elucidate the physical processes that drive quasar variability.
The full Light-Curve and Spectral
Analyses that we envisaged will be a significant amount of work,
leading to a signifcant high-reward science. This particular work
package will be broken down into small projects and the level of two
PDRAs (``PDRA 1'' and ``PDRA 2''), as well as the P.I. (NPR) and a PhD
student (``PhD 1'') will be directed towards this. There is no element
of this which can be deemed high-risk.


\smallskip
\smallskip
\noindent
\textbf{\textsc{WP4: Quasar Demographic studies:}} 
Another major scientific output that will originate from the quasar
corpus catalogue generation will be the study of the Quasar
Demographics from our datastreams. This is different from WP3 in that
these investigations wont necessarily be tied to the time-domain
aspect of our catalogue, but will be the crucial baseline that we, and
the field in general, will use to compare to the time-depedent
discoveries. Luminosity function, clustering and higher-order
statistics will be made in order to precisely determine the census of
AGN, their environemts, their host galaxy preferences and their
evolution. All these are vital observational tests for galaxy
formation models and theory (see WP6).

\noindent
{\bf WP4 is low-risk, high-reward.}
The goal of this WP is to make the key observational tests that have to be explained by any 
viable galaxy formation theory. 
Similar to WP3, the analyses that we envisaged will be broken down
into small projects and the level of two PDRAs (``PDRA 1'' and ``PDRA
2''), as well as the P.I. (NPR) and a PhD student (``PhD 1'') will be
directed towards this. There is no element of this which can be deemed
high-risk.


\smallskip
\smallskip
\noindent
\textbf{\textsc{WP5: Accretion Disk Simulation:}} 
New accretion models are needed to fully explain the observational
data of ``changing look'' quasars that we have examples of today (see
e.g. Ross et al. 2018). New radiation MHD codes begin to explain the
observations here, but further developement is needed to gain the
desired deep understanding. 

\noindent
{\bf WP5 is lower-risk, high-reward.}
The goal of WP5 is to develop new accretion disk simulations that
explain our observational results.  This will be the lead WP for one
PDRA (``PDRA 3'') and a low level of the P.I.'s (NPR) time. We
classify it not as fully `low-risk', since we envisage some ramp-up
time to get our theortical simulations to the level that will match
our beyond-the-state-of-the-art dataset. However, we mitgate this risk
by invoking the collaboration with an accretion disk theorist
Prof. Ken Rice (WKMR) who is the Personal Chair of Computational
Astrophysics in the School of Physics and Astronomy here at the
University of Edinburgh.


\smallskip
\smallskip
\noindent
\textbf{\textsc{WP6: AGN Feedback Simulations:}} 
Cosmological-scale hydrodynamic simulations are now coming online. 
While we do not week to lead or generate new versions of these, we do 
envisaged using their outputs in order to `benchmark' our observational 
demographic work. 

\noindent
{\bf WP6 is low-risk, high-reward.}
All the data from these simulations is already in place today, though no one 
has embarked on doing any of the `heavy-lifting' and comparisons we will 
have the observational results for. Professor RS Dave (RSD) who is a Chair of Physics 
in the Institute for Astronomy will be a key collaborator here. 


\smallskip
\smallskip
\noindent
\textbf{\textsc{WP7: Observations of Qausars by the James Webb Space Telescope:}} 
What are the star-formation properties of mid-infrared luminous quasars at the peak of quasar activity? 
We aim to answer this by looking for the presence of polycyclic aromatic hydrocarbon (PAH) spectral features 
in $z \approx 2.5$ infrared bright quasars. 

\noindent
{\bf WP4 is medium-to-high risk, high-reward.}
This is an ideal investigation for the James Webb Space Telescope, but we classify this as `high-risk' since this is the one telescope/survey/mission where we would have to bid/apply for the telescope time and are not guaranteed the data. We mitigate the risk here by saying that this will be the one project the P.I. (NPR) would directly lead, and would lead to very-high gain science, but does not impact in any direct way any of the other WPs. 


\smallskip
\smallskip
\noindent
\textbf{\textsc{WP8: New Object Discovery:}} 
The LSST will scan the sky repeatedly, enabling it, and us, to both
discover new, distant transient events and to study variable objects
throughout our universe. The LSST will extend our view of the
changeable universe a thousand times over current surveys.  The most
interesting science to come may well be the discovery of new classes
of objects.

\noindent
{\bf WP8 is medium-risk, exceptionally high-reward.}
We class this as medium-risk, since it is tricky to class a WP with essentially unknown discovery potential as fully `low-risk'. However, we do not classify this as `high-risk' since if there was a paucity of discovery of novel classes of objects, this would be the first time in the hitorsy of observational astrophysics that a new facility such as LSST has come online and found nothing new. 


\smallskip
\smallskip
\noindent
Although many of the ``building blocks'' for our science are already
available (e.g. open source codes, database infrastructures, the
methodology of catalog creation and combination) no one has yet to
combine the data in the way we envisage. Moreover, the new datasets we
desire to deliver our paradigm changing science are only coming online
over the next 5 or so years.


\smallskip
\smallskip
\noindent
\textbf{\textsc{{Data Science and Observational Astrophysics:}}}
Data science is a new interdisciplinary field of scientific methods to
extract knowledge or insights from data in various forms, either
structured or unstructured. It employs techniques and theories drawn
from many fields within the broad areas of mathematics, statistics,
information science, and computer science, in particular from the
subdomains of machine learning, classification, cluster analysis, data
mining, databases, and visualization.  {\it Modern day observational
astrophysicists are in all but name data scientists, and as such, this
proposal is inherently interdisciplinary.}

\begin{figure}[h]
  \begin{center}
   \hspace{-0.5cm}
%   trim=l b r t
    \includegraphics[width=16.0cm] %, trim={0.05cm 0 0.05cm 0},clip]
    {figures/workplan.pdf}
    \vspace{-10pt}
 %  \caption{}
  \vspace{-12pt}
 \label{fig:Keynote_facilites}
\end{center}
\end{figure}

\smallskip
\smallskip
\noindent
The P.I. has become a world-leader in the field of extragalactic
quasar observational astrophysics.  Moreover, the University of
Edinburgh is now poised to be an astronomical data centre nexus, with
access to the two largest datasets in our proposal; LSST and {\it
Euclid}.  The P.I. has built their career on this science case, and
has already been a P.I.  of a science group (as part of a
collaboration) with prodigious scientific output (400 published,
peer-reviewed papers and counting).


\smallskip
\smallskip
\noindent
%\section*{\textcolor{Cerulean}{4. Feasibility, Projects and Methodology}}
\textbf{\textsc{P.I.'s Experience and Track Record:}}
The P.I. has an established track record of managing science teams and groups, e.g., 

\begin{itemize}
\item The P.I. was the Chair
of the SDSS-III:BOSS Quasar Science Working Group, managing a
group of senior professors, other postdocs and graduate students.
The scientific yield from the BOSS Quasar Survey was extremely high 
%with highlights including: the first detection of baryon acoustic
%oscillations at $z>1$; the first detection of baryon acoustic
%oscillations using quasars; the first measurements of ``the knee'' of
%the $z\sim2.5-3$ and $z=5$ optical quasar luminosity function, and the
%discovery of the first ``Changing Look'' Quasar.  BOSS also allowed
%the first identification of ``extremely red'' quasars, which are a
%unique obscured quasar population with extreme physical conditions
%related to powerful outflows. {\it The P.I. either led, or was heavily
%involved in, all these projects, and 
with \href{https://tinyurl.com/ycxd8lb6}{over 400 journal publications} having 
used the BOSS Quasar catalogs and datasets.

\item The P.I. spent a considerable amount of time in 2017 working as the
Chief Data Scientist for a San Francisco Bay Area 
tech start-up\footnote{Due to legal immigration issues, this venture is no longer being pursued.}. 

\item The P.I. is an STFC Ernest Rutherford Fellow, with a budget of \euro615,000 on award. 
\end{itemize}


\smallskip
\smallskip
\noindent
\textbf{\textsc{Breaking Down The Data Silos}}
to using advanced data analysis is not skill base or technology; it is
simply access to the data.  A data silo is a repository of fixed data
that remains under the control of one department/collaboration and is
isolated from the rest of the world, much like grain in a farm silo is
closed off from outside elements. These silos are isolated islands of
data, and they make it prohibitive to extract data and put it to other
uses. They can arise for multiple reasons. In commercial enterprises,
data remained siloed for monetary gain.  However, in research
environments, and {\it especially in contemporary observational
astrophysics}, the data silos are open, but due to the lack of raw
person-power, still remain uncombined. {\it The combination 
of P.I. and host institute means we are uniquely positioned to 
break down these astro-data silos for massively significant 
science gain.}


\smallskip
\smallskip
\noindent
\textbf{\textsc{Algorithms}}
Our algorithms and methodology is based on the latest machine-learning and data science techniques. 
This includes the ``extreme deconvolution'' \href{http://www.sdss.org/dr14/data\_access/value-added-catalogs/?vac\_id=xdqso/}{`XDQSO' technique}\footnote{\href{https://github.com/xdqso/xdqso}{\tt github.com/xdqso/xdqso}}.
%%
\href{http://ogrisel.github.io/scikit-learn.org/sklearn-tutorial/index.html}{\tt
scikit-learn} is a Python module integrating classic machine learning
algorithms in the scientific Python world (numpy, scipy,
matplotlib). It aims to provide simple and efficient solutions to
learning problems, accessible to everybody and reusable in various
contexts.  \href{https://github.com/astroML/sklearn\_tutorial}{{\tt
github.com/astroML/sklearn\_tutorial}} and \href{https://github.com/jakevdp/PythonDataScienceHandbook}{{\tt github.com/jakevdp/PythonDataScienceHandbook}} have full details.

%\href{https://github.com/astroML/sklearn\_tutorial}{{\tt https://github.com/astroML/sklearn\_tutorial}}\\
%\href{https://github.com/jakevdp/PythonDataScienceHandbook}{{\tt https://github.com/astroML/sklearn\_tutorial}}\\
%\href{https://github.com/jakevdp/sklearn\_tutorial}{{\tt https://github.com/jakevdp/sklearn\_tutorial}}\\


\smallskip
\smallskip
\noindent
\large
{\bf{\textcolor{Cerulean}{3. Resources,  Survey 'buy-in' and Budget}}}
\normalsize

\smallskip
\smallskip
\noindent
\textbf{\textsc{Personnel:}} 
 Lorem ipsum dolor sit amet, consectetur adipiscing elit. Aliquam porta sodales est, vel cursus risus porta non. Vivamus vel pretium velit. Sed fringilla suscipit felis, nec iaculis lacus convallis ac. Fusce pellentesque condimentum dolor, quis vehicula tortor hendrerit sed. Class aptent taciti sociosqu ad litora torquent per conubia nostra, per inceptos himenaeos. Etiam interdum tristique diam eu blandit. Donec in lacinia libero.

\noindent
Sed elit massa, eleifend non sodales a, commodo ut felis. Sed id pretium felis. Vestibulum et turpis vitae quam aliquam convallis. Sed id ligula eu nulla ultrices tempus. Phasellus mattis erat quis metus dignissim malesuada. Nulla tincidunt quam volutpat nibh facilisis euismod. Cras vel auctor neque. Nam quis diam risus.

\smallskip
\smallskip
\noindent
\textbf{\textsc{Survery Buy-in:}} 
Nunc semper quam et leo interdum vulputate eu quis magna. Sed nec arcu at orci egestas convallis. Aenean quam velit, aliquam vitae viverra in, elementum vel elit. Nunc suscipit aliquet sapien a suscipit. Cras nulla ipsum, posuere eu fringilla sit amet, dapibus ultricies nulla. Nullam eu augue id purus mollis dignissim sed et libero. Phasellus eget justo sed neque pellentesque egestas nec id arcu. Donec facilisis pulvinar sapien et fringilla. Suspendisse vestibulum rhoncus sapien id laoreet. Morbi et orci vitae tortor imperdiet imperdiet. In hac habitasse platea dictumst. Vivamus vel neque id mi ultrices tristique. Integer quam libero, ornare vel gravida in, feugiat a ante. Nam dapibus, tellus vitae pellentesque cursus, dui nisl egestas augue, non fermentum nisl est nec nisi. Vestibulum nec mi justo, eget dapibus velit.

\smallskip
\smallskip
\noindent
\textbf{\textsc{Computing Requirements:}} 
With the availability of contemporary cloud based services (both
commercial and non-for-profit), the infrastructure is essentially
already in place for all but the most demanding of compute tasks. The
rate limiting factor, in the vast majority of endevours, is (a) data
access and (b) development.

\smallskip
\smallskip
\noindent
The facilites available to me at an institute (e.g. IfA Cullen),
university
(e.g. \href{https://www.ed.ac.uk/information-services/research-support/research-computing/ecdf}{``Edinburgh
Compute and Data Facility''} and at a national
{\href{https://www.hartree.stfc.ac.uk/Pages/home.aspx}{(The Hartree
Centre)} level will all be sufficient and utilized.  The rate limiting
factor will be how quickly and efficiently we can deploy our codes,
and analysis, i.e. person-power.

\smallskip
\smallskip
\noindent
%\subsection*{Open Innovation, Open Science, Open to the World}
\textbf{\textsc{Open Innovation, Open Science, Open to the World:}} 
The P.I. is an exceptionally strong, longtime and vocal supporter of ``Open Access''. 
All my codes, data\footnote{where I am not breaking current data access agreements}, papers 
and proposals can be found at \href{github.com/d80b2t}{{\tt github.com/d80b2t}}. 
Indeed, this proposal itself is now at that location. 

\iffalse
One of the major research outputs 
of this ERC will be computer code. 
As such, we are already working with the
\href{The Software Sustainability Institute}{\tt https://www.software.ac.uk/}
which was founded to support the UK’s research software community. 
Our software well be developed using the FAIR ideology (Findable, Accessible, Interoperable, Reusable
\footnote{Wilkinson, MD, Sci Data. 2016 Mar 15;3:160018. doi: 10.1038/sdata.2016.18.}
) 
and will be delivered in a manner which is fully inline 
with ``Open Innovation, Open Science, Open to the World''. 
\fi


\smallskip
\smallskip
\noindent
\large
{\bf{\textcolor{Cerulean}{4. Project Management:}}}
\normalsize

\smallskip
\smallskip
\noindent
Lorem ipsum dolor sit amet, consectetur adipiscing elit. Aliquam porta
sodales est, vel cursus risus porta non. Vivamus vel pretium
velit. Sed fringilla suscipit felis, nec iaculis lacus convallis
ac. Fusce pellentesque condimentum dolor, quis vehicula tortor
hendrerit sed. Class aptent taciti sociosqu ad litora torquent per
conubia nostra, per inceptos himenaeos. Etiam interdum tristique diam
eu blandit. Donec in lacinia libero.

\smallskip
\smallskip
\noindent
Sed elit massa, eleifend non sodales a, commodo ut felis. Sed id
pretium felis. Vestibulum et turpis vitae quam aliquam convallis. Sed
id ligula eu nulla ultrices tempus. Phasellus mattis erat quis metus
dignissim malesuada. Nulla tincidunt quam volutpat nibh facilisis
euismod. Cras vel auctor neque. Nam quis diam risus.




\input{references}


\end{document}

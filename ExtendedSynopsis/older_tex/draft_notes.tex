\documentclass[11pt,a4paper]{article}
\input{format}
\usepackage{amsmath}
\usepackage{tcolorbox}
\usepackage{framed}


\begin{document}


Our ERC Consolidator grant proposal will radically improve our understanding of 
one of the two fundamental energy sources available to galaxies; that of accretion 
onto the compact object in the central engine. We will achieve this by leveraging 
several of the new, large-scale surveys that are coming online in the next few years. 
The scope and remit of an ERC Consolidator grant will allow us to combine these 
data products in a manner that will 
% This programme will 
not only establish the new state-of-the-art in extragalactic variable science, 
{\it it will establish and kickstart the new field of extragalactic variable science itself}. 
%%
The P.I. is a world-leader in observational quasar astrophysics, both in terms of 
survey work and individual object study. 
%%
Our proposal takes astrophysics into the 2020s, going from single objects samples, 
to surveys and samples of millions of objects leveraging these multi-billion Euro/dollar/pound  
next generation missions, telescopes and their subsequent datasets. 
%%

Quasars are ideal probes... etc. 

\begin{tcolorbox}
\begin{center}
  Overview of Surveys related to this propopsal
\end{center}
\end{tcolorbox}

\begin{framed}
The {\bf Sloan Digital Sky Survey (SDSS):} An ongoing project, currently in its fourth phase, SDSS-IV.  
The P.I. was a leading member of the SDSS-III: Baryon Oscillation Spectroscopic Survey (BOSS). 
The fifth generation of Sloan Digital Sky Surveys: SDSS-V will be an all-sky, multi-epoch spectroscopic 
survey, yielding spectra of over 6 million objects during its lifetime. Data taking is due to start in 
2020. Access would be through a USD \$230,000 'buy-in' (which allows access for the P.I. and one PDRAs).  . 
{\it Data Products: Repeat spectra in the North and Southern Hemisphere for 500,000 bright QSOs.} \\

The {\bf Dark Energy Spectroscopic Instrument (DESI) Survey:} is a 5 year cosmology survey 
that will be conducted on the Mayall 4-meter telescope at Kitt Peak National Observatory starting 
in 2019. It uses the 5,000 fiber Dark Energy Spectroscopic Instrument and will obtain optical 
spectra for $\approx$20 million galaxies and quasars. The DESI Survey is starts in late 2019 
and data access is through a USD \$250,000 `buy-in' (which allows access for the P.I. and two PDRAs).  \\
{\it Data Products: Spectra of 1e6 quasars across 14,000 deg$^{2}$ of the Northern Sky.} \\

\textit{\textbf{Euclid}} is an ESA Medium Class mission to map the geometry of the dark Universe.
It aims to understand why the expansion of the Universe is accelerating and what the nature 
of the source responsible for this acceleration (``dark energy'') is. 
%%
The
mission will investigate the distance-redshift relationship and the
evolution of cosmic structures by measuring shapes and redshifts of
galaxies and clusters of galaxies out to redshifts $\sim$2, or equivalently
to a look-back time of 10 billion years. In this way, Euclid will
cover the entire period over which dark energy played a significant
role in accelerating the expansion.
%%
{\it Euclid} is planned for launch in mid-2021. 
%%
{\it Data Products: Very broadband optical and 3 filter near-infrared space-based imaging for 15,000 deg$^2$.} \\

The {\bf Large Synoptic Survey Telescope (LSST)} project will conduct
a 10-year survey of the sky, imaging the full Southern Sky every 3
nights. The LSST survey is designed to address four science areas
(Understanding the Mysterious Dark Matter and Dark Energy; Hazardous
Asteroids and the Remote Solar System; The Transient Optical Sky; The
Formation and Structure of the Milky Way) and is an absolutely unique
facility as far as areal, temporal and wavelength coverage.\\
{\it Data Products: $ugrizY$ broadband optical and near-infrared imaging for 20,000 deg$^2$. 
Images the full Southern Sky every 3 days. } \\

The {\bf 4-metre Multi-Object Spectroscopic Telescope  (4MOST):} 
is a fibre-fed spectroscopic survey facility on the VISTA telescope with a large enough field-of-view to survey a large fraction of the southern sky. The facility will be able to simultaneously obtain spectra of 2,400 objects distributed over a field-of-view of 4 square degrees. 
The initial Galactic and Extragalactic surveys will operate over a five-year period delivering spectra for $\geq$25 million objects over 
$\gtrsim$15,000 deg. 4MOST will commence science operations in mid-2021 ({\bf TO BE CHECKED!!!}. 
{\it Data Products: } \\

{\underline Notes::} 4MOST has full access to the full LSST footprint. LSST will overlap half (7,500 deg$^2$) of the {\it Euclid} footprint. \\

The {\it Extended Roentgen Survey with an Imaging Telescope Array (eROSITA)} is the main instrument on the 
Spektr-RG (Spectrum-X-Gamma; SRG, SXG), an international high-energy astrophysics observatory. 


\hrulefill 

The {\bf Wide-field Infrared Survey Explorer (WISE)} is a NASA
infrared-wavelength astronomical space telescope launched in December
2009 and is still operation (in its ``NEOWISE-R'' mission phase as at
the time of writing). WISE performed an all-sky astronomical survey
with images at 3.4, 4.6, 12 and 22$\mu$m using a 40cm (16 in) diameter
infrared telescope in Earth orbit. 

The {\bf James Webb Space Telescope (JWST)} is a space telescope
developed in coordination among NASA, the European Space Agency, and
the Canadian Space Agency. It is scheduled to be launched in June
2019. The telescope will offer unprecedented resolution and
sensitivity from 0.6 to 27$\mu$m.
%%
JWST is a partnership between NASA, ESA and the Canadian Space Agency.
In particular, ESA's contributions to JWST include (but are not
limited to) the NIRSpec instrument and the Optical Bench Assembly of
the MIRI instrument.  In return for these contributions, ESA gains
full partnership in JWST and secures full access to the JWST
observatory for astronomers from ESA Member States on identical terms
to those of today on the Hubble Space Telescope. European scientists
will be represented on all advisory bodies of the project and will be
expected to win observing time on JWST through a joint peer review
process, backed by an expectation of a minimum ESA share of 15\% of
the total JWST observing time.


The ESA {\it Gaia} mission is an ongoing mission to chart a
three-dimensional map of our Galaxy, the Milky Way, in the process
revealing the composition, formation and evolution of the Galaxy. Gaia
is providing unprecedented positional and radial velocity measurements
with the accuracies needed to produce a stereoscopic and kinematic
census of about $\sim$one billion stars in our Galaxy and throughout
the Local Group. This amounts to about 1 per cent of the Galactic
stellar population.


%\hline 
%\hline

\end{framed}



sizes and timescales:: \\
10$^{8}$ MBH to Jupiter, \\
speed at which we send a probe there...\\




\newpage


Black holes are omnipresent in our Universe, and black holes that are
millions to billions of times the mass of our Sun, are ubiquitously
found at the centers of galaxies, including our own Milky Way.
Initially consider physical oddities, we now strongly suspect that
these (central, ``supermassive'') galactic black holes have a profound
affect on the galaxies that they live in. This is not surprising since
the potential energy associated with mass accretion onto a
supermassive black hole is comparable to that generated via the
nuclear fusion in the galaxy's stars.

However, the interaction and the physical processes involved in how
this energy escapes the inner most regions of the galaxy and then
interacts with the gas, dust, stars and dark matter, is currently very
poorly understood theoretical, with very few observational data giving
insight on how to make key progress.

The field is poised for a fundamental and rapid change. The first data are now in hand 
that show {\it changes on human timescales} in external galaxies, with these new 
field defining studies including lead projects by the P.I. 

This proposal has two broad but well-posed goals. 
First, we aim to elucidate, for the first time, how the energy directly associated with a 
supermassive black holes impacts the universal galaxy population.  

Second, will open up and explore the Variable Extragalactic Universe, bringing to bear 
the slew of 
CLQs, TDEs, Binary BHs, binary SMBHs... 
Things will go ``bang in the middle of the night''; we just don't know what 
they are yet. 

NS-NS merger (EM signatures in the blue...) ...
Unify the four fundamental forces of nature...


We ask for the personnel to accomplish these vey ambituous, 
but achievable goals, along with the `buy-in' to the facilites we need access to. 






\subsection{Contemporary Galaxy formation theory}
























\end{document}

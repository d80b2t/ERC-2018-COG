\documentclass[oneside, a4paper, onecolumn, 11pt]{article}

\usepackage{amsmath, amssymb}
\usepackage{booktabs, bm}           %%  bold math
\usepackage{cancel}
\usepackage{dcolumn}  %%  Align table columns on decimal point
\usepackage{epsfig, epsf, eurosym, enumitem}
\usepackage{fancyhdr}
\usepackage[T1]{fontenc}
\usepackage[para]{footmisc}
\usepackage{graphicx }
%\usepackage{lscape}
\usepackage{hyperref,ifthen}
\usepackage{mathptmx, multicol}
\usepackage[authoryear, round]{natbib}
\usepackage{nopageno}
\usepackage{subfigure}
\usepackage{verbatim}
\usepackage{threeparttable}
\usepackage[usenames,dvipsnames]{xcolor}
\usepackage{tcolorbox}
\usepackage{tabularx}
\usepackage{array}
\usepackage{colortbl}
\usepackage{framed}
\usepackage{todonotes}



%%%%%%%%%%%%%%%%%%%%%%%%%%%%%%%%%%%%%%%%%%%
%       define Journal abbreviations      %
%%%%%%%%%%%%%%%%%%%%%%%%%%%%%%%%%%%%%%%%%%%
\def\nat{Nat} \def\apjl{ApJ~Lett.} \def\apj{ApJ}
\def\apjs{ApJS} \def\aj{AJ} \def\mnras{MNRAS}
\def\prd{Phys.~Rev.~D} \def\prl{Phys.~Rev.~Lett.}
\def\plb{Phys.~Lett.~B} \def\jhep{JHEP}
\def\npbps{NUC.~Phys.~B~Proc.~Suppl.} \def\prep{Phys.~Rep.}
\def\pasp{PASP} \def\aap{Astron.~\&~Astrophys.} \def\araa{ARA\&A}
\def\jcap{\ref@jnl{J. Cosmology Astropart. Phys.}} 
\def\nar{New~A.R.} \def\aapr{A\&ARv}

\newcommand{\preep}[1]{{\tt #1} }

%%%%%%%%%%%%%%%%%%%%%%%%%%%%%%%%%%%%%%%%%%%%%%%%%%%%%
%              define symbols                       %
%%%%%%%%%%%%%%%%%%%%%%%%%%%%%%%%%%%%%%%%%%%%%%%%%%%%%
\def \Mpc {~{\rm Mpc} }
\def \Om {\Omega_0}
\def \Omb {\Omega_{\rm b}}
\def \Omcdm {\Omega_{\rm CDM}}
\def \Omlam {\Omega_{\Lambda}}
\def \Omm {\Omega_{\rm m}}
\def \ho {H_0}
\def \qo {q_0}
\def \lo {\lambda_0}
\def \kms {{\rm ~km~s}^{-1}}
\def \kmsmpc {{\rm ~km~s}^{-1}~{\rm Mpc}^{-1}}
\def \hmpc{~\;h^{-1}~{\rm Mpc}} 
\def \hkpc{\;h^{-1}{\rm kpc}} 
\def \hmpcb{h^{-1}{\rm Mpc}}
\def \dif {{\rm d}}
\def \mlim {m_{\rm l}}
\def \bj {b_{\rm J}}
\def \mb {M_{\rm b_{\rm J}}}
\def \mg {M_{\rm g}}
\def \mi {M_{\rm i}}
\def \qso {_{\rm QSO}}
\def \lrg {_{\rm LRG}}
\def \gal {_{\rm gal}}
\def \xibar {\bar{\xi}}
\def \xis{\xi(s)}
\def \xisp{\xi(\sigma, \pi)}
\def \Xisig{\Xi(\sigma)}
\def \xir{\xi(r)}
\def \max {_{\rm max}}
\def \gsim { \lower .75ex \hbox{$\sim$} \llap{\raise .27ex \hbox{$>$}} }
\def \lsim { \lower .75ex \hbox{$\sim$} \llap{\raise .27ex \hbox{$<$}} }
\def \deg {^{\circ}}
%\def \sqdeg {\rm deg^{-2}}
\def \deltac {\delta_{\rm c}}
\def \mmin {M_{\rm min}}
\def \mbh  {M_{\rm BH}}
\def \mdh  {M_{\rm DH}}
\def \msun {M_{\odot}}
\def \z {_{\rm z}}
\def \edd {_{\rm Edd}}
\def \lin {_{\rm lin}}
\def \nonlin {_{\rm non-lin}}
\def \wrms {\langle w_{\rm z}^2\rangle^{1/2}}
\def \dc {\delta_{\rm c}}
\def \wp {w_{p}(\sigma)}
\def \PwrSp {\mathcal{P}(k)}
\def \DelSq {$\Delta^{2}(k)$}
\def \WMAP {{\it WMAP \,}}
\def \cobe {{\it COBE }}
\def \COBE {{\it COBE \;}}
\def \HST  {{\it HST \,\,}}
\def \Spitzer  {{\it Spitzer \,}}
\def \ATLAS {VST-AA$\Omega$ {\it ATLAS} }
\def \BEST   {{\tt best} }
\def \TARGET {{\tt target} }
\def \TQSO   {{\tt TARGET\_QSO}}
\def \HIZ    {{\tt TARGET\_HIZ}}
\def \FIRST  {{\tt TARGET\_FIRST}}
\def \zc {z_{\rm c}}
\def \zcz {z_{\rm c,0}}


\newcommand{\sqdeg}{deg$^{-2}$}
\newcommand{\lya}{Ly$\alpha$\ }
%\newcommand{\lya}{Ly\,$\alpha$\ }
\newcommand{\lyaf}{Ly\,$\alpha$\ forest}
%\newcommand{\eg}{e.g.~}
%\newcommand{\etal}{et~al.~}
\newcommand{\cii}{C\,{\sc ii}\ }
\newcommand{\ciii}{C\,{\sc iii}]\ }
\newcommand{\civ}{C\,{\sc iv}\ }
\newcommand{\SiIV}{Si\,{\sc iv}\ }
\newcommand{\mgii}{Mg\,{\sc ii}\ }
\newcommand{\feii}{Fe\,{\sc ii}\ }
\newcommand{\feiii}{Fe\,{\sc iii}\ }
\newcommand{\caii}{Ca\,{\sc ii}\ }
\newcommand{\halpha}{H\,$\alpha$\ }
\newcommand{\hbeta}{H\,$\beta$\ }
\newcommand{\oi}{[O\,{\sc i}]\ }
\newcommand{\oii}{[O\,{\sc ii}]\ }
\newcommand{\oiii}{[O\,{\sc iii}]\ }
\newcommand{\heii}{[He\,{\sc ii}]\ }
\newcommand{\nii}{N\,{\sc ii}\ }
\newcommand{\nv}{N\,{\sc v}\ }

%% From:: /cos_pc19a_npr/LaTeX/proposals/JWST/JWST_ERS/Proposal/lines.tex
%%  
\newcommand{\imw}{$i$--$W3$}
\newcommand{\imwf}{$i$--$W4$}
\newcommand{\rmwf}{$r$--$W4$}
\newcommand{\imwt}{$i$--$W2$}
\newcommand{\wtmwf}{$W3$--$W4$}
%\newcommand{\kms}{km s$^{-1}$}
\newcommand{\cmN}{cm$^{-2}$}
\newcommand{\cmn}{cm$^{-3}$}
%\newcommand{\msun}{M$_{\odot}$}
\newcommand{\lsun}{L$_{\odot}$}
\newcommand{\lam}{$\lambda$}
\newcommand{\mum}{$\mu$m}
\newcommand{\ebv}{$E(B$$-$$V)$}
%\newcommand{\heii}{\mbox{He\,{\sc ii}}}
\newcommand{\cv}{\mbox{C\,{\sc v}}}
%\newcommand{\civ}{\mbox{C\,{\sc iv}}}
%\newcommand{\ciii}{\mbox{C\,{\sc iii}}}
%\newcommand{\cii}{\mbox{C\,{\sc ii}}}
%\newcommand{\nv}{\mbox{N\,{\sc v}}}
\newcommand{\niv}{\mbox{N\,{\sc iv}}}
\newcommand{\niii}{\mbox{N\,{\sc iii}}}
%\newcommand{\oi}{\mbox{O\,{\sc i}}}
%\newcommand{\oii}{\mbox{O\,{\sc ii}}}
%\newcommand{\oiii}{\mbox{[O\,{\sc iii}]}}
\newcommand{\oiv}{\mbox{O\,{\sc iv}}}
\newcommand{\ov}{\mbox{O\,{\sc v}}}
\newcommand{\ovi}{\mbox{O\,{\sc vi}}}
\newcommand{\ovii}{\mbox{O\,{\sc vii}}}

%\newcommand{\feii}{\mbox{Fe\,{\sc ii}}}
%\newcommand{\feiii}{\mbox{Fe\,{\sc iii}}}
%\newcommand{\mgii}{\mbox{Mg\,{\sc ii}}}
\newcommand{\neii}{[Ne\,{\sc ii}]\ }
\newcommand{\neiii}{[Ne\,{\sc ii}]\ }
\newcommand{\nev}{Ne\,{\sc v}\ }
\newcommand{\nevi}{[Ne\,{\sc vi}]\ }
\newcommand{\neviii}{\mbox{Ne\,{\sc viii}}}
\newcommand{\aliii}{\mbox{Al\,{\sc iii}}}
\newcommand{\siii}{\mbox{Si\,{\sc ii}}}
\newcommand{\siiii}{\mbox{Si\,{\sc iii}}}
\newcommand{\siiv}{\mbox{Si\,{\sc iv}}}
%\newcommand{\lya}{\mbox{Ly$\alpha$}}
%\newcommand{\lyb}{\mbox{Ly$\beta$}}
\newcommand{\hi}{\mbox{H\,{\sc i}}}
\newcommand{\snine}{\mbox{[S\,{\sc ix}]}}
\newcommand{\sivi}{\mbox{[Si\,{\sc vi}]}}
\newcommand{\sivii}{\mbox[{Si\,{\sc vii}]}}
\newcommand{\siix}{\mbox{[Si\,{\sc ix}]}}
\newcommand{\six}{\mbox{[Si\,{\sc x}]}}
\newcommand{\sixi}{\mbox{[Si\,{\sc xi}]}}
\newcommand{\caviii}{\mbox{[Ca\,{\sc viii}]}}
\newcommand{\arii}{\mbox{[Ar\,{\sc ii}]}}

%%[Ar II] 6.97
%% [S IX] 1.252 μm 328 
% [Si X] 1.430 μm 351 
% [Si XI] 1.932 μm 401 
% [Si VI] 1.962 μm 167 
% [Ca VIII] 2.321 μm 128 
% [Si VII] 2.483 μm 205 
% [Si IX] 3.935 μm 303
% [Ar II] 6.97


%\snine\ at 1.252$\mu$m, \six\ at 1.430$\mu$m, \sixi\ at 1.932$\mu$m, \sivi\ at
%1.962$\mu$m, \caviii\ at 2.321$\mu$m, \sivi\ at 2.483$\mu$m \siix\ at
%3.935$\mu$m and \arii\ at 6.97$\mu$m. 
%%
%% such as [Ne ii]12.8 μm, [Ne v]14.3 μm, [Ne iii]15.5 μm, [S iii]18.7 μm and 33.48 μm, [O iv]25.89 μm and [Si ii]34.8 μm (e.g
%%
%% MIR emission lines like [NeII] and [NeV] are ..
%%
%% Also,  arXiv:astro-ph/0003457v1 
%% [NeV] 14.32um & 24.32um and [NeVI] 7.65um imply an A(V)>160 towards the NLR...
%% [NeIII]15.56um/[NeII]12.81um
%%
%% [Ne V] 14.3, 24.2 μm 97.
%% [Ne II] 12.8 μm
%% [OIV] 26μm
%%

\usepackage{amsmath}
\usepackage{tcolorbox}
\usepackage{framed}


\begin{document}


\section{Section A. State-of-the-art and Objectives}
The outstanding challenge in contempory galaxy formation theories is...

In your first sentence start by giving some background information to the problem (to set the scene), you can also provide some statistics or financial information i.e. current cost of the disease to the health system, purification of water etc. Follow this by what is the current situation and what your ground breaking solution is to this problem. Does this need a coordinated effort across a number of different disciplines? Also stress here why you are uniquely placed to answer this problem.

``State of the art is 43 objects, 11 of which I've discovered.''

The objectives of the proposal are:
\begin{itemize}
\item To elucidate on how the energetics of the galaxy central engine impact the host galaxy;
\item To produce the state-of-the-art (and quite possibly the unique) multi-wavelength, multi-epoch 
quasar dataset from the combination of 5 of the worlds leading surveys in 2020; 
\item To connect, for the first time, the physical mechanisms acting on sub-parsec to mega-parsec scales. 
\end{itemize}

This will be achieved by:
\begin{itemize}
\item Characterizing one million changing look quasars;
\item Cataloging {\it Euclid}, 4MOST, SDSS-V, DESI and LSST quasars; 
\item Theoretically connecting new simulation codes at the accretion disk and galaxy-scales. 
\end{itemize}

Some points to keep in mind when include::
\begin{itemize}
\item clearly state why and how the proposed work is novel and important in your field
\item what are your objectives,
\item What are the key challenges/open questions in your field that need to be answered 
\item how will you go about it, clearly explain how you propose to address these questions. 
\item what are the expected outcomes?
\item  explain the impact of your work- if you are successful-, on the research area and beyond and your long term vision.
\end{itemize}


\section{Why Now?}
Figure~\ref{} has the timelines for the major projects that are coming online in the next decade. 


Applicant Principal Investigators 
\subsection{must demonstrate the ground-breaking nature, }
\subsection{ ambition}

\subsection{Feasibility of their scientific proposal.}
%% 
%% Explain the risks of your project and how you plan to address those risks should they materialise.  

WhilteNot risky.  Current data SDSS
I-IV = 520,000 spectra published; 1,000,000 by 2020 (survey ongoing).
PanSTARRS, CRTS, ZTF;

Future:: DESI, ground-based in full construction mode; survey is
starting late 2019) SDSS-V; ground-based; telescope, pipeline,
instrumentation already in place. new surveys start ing 2020 LSST;
ground-based in full construction mode; science survey is starting
2021..)  4MOST; ground-based, in pre construction mode) Euclid, very
muich the most risky programme, since it is space-based, and currently
most likely to slip in schedule. However, although a key ESA facility,
Euclid is arguably the least important of the 5 data streams for our
science case.


\section{Question and Anwer table}
You could also use a two column table having the questions you have set to answer on one side and the respective objectives for each question on the second column (Just a recommendation of the things you could do to make reading your proposal more pleasant for the evaluator).



\newpage
\section{Methodology}

\smallskip
\smallskip
\noindent
Describe your work plan in detail. You can separate the section in terms of work packages/case studies or describe the work in terms of the aims/objectives you described in the previous section, and how you will accomplish those. If you like you can have a flow chart of the different WP’s and how the whole project will come together. Each WP or aims/objectives can be broken down into subtasks.


\begin{itemize}
\item If you have any preliminary results in relation to the work you are describing make sure you emphasise it. Maybe dedicate a separate subsection/heading to preliminary results?
\item You can also list the milestones and the ground breaking features of the planned work. Also how will this WP/aim advance the state of the art in your field? This way even the non-expert evaluator will be able to see the big picture of what you are proposing to do.
\item Describe the balance of your project between the high risk /high gain experiments and what will be the long term benefits in your area from the results you will generate. How will you be advancing the field? In other words emphasise the impact of your work.
\item Have a dedicated section on feasibility of what you are proposing. Explain which WP’s/tasks present high levels of risk. Provide a contingency plan, particularly if any of the tasks are unconventional, present a great challenge and are high risk (but also high gain). Mention your experience and knowledge to hedge against this risk or alternative approaches or help from collaborators. Be “safely adventurous”.
\item Include a gantt chart or a timeline for the evaluators to visualise the timescale of each component of the work you are proposing.
\item Include a 4-5 line summary to recap and remind the evaluator what the essence of the project is and why it so important to get this funded now.
\end{itemize}


\newpage
\section{Resources (including project costs)}
Resources and timelines should entirely reflect the project and nothing else. Link the budget to the proposed activities as accurately as possible.

Feasibility is key.\\
Remember to use whole Euro integers only, when preparing the budget table.
Make sure the numbers on the table match with the numbers on the A forms.
Use this table as it is and don’t reformat it.
Delete all the superscript numbers in final draft to get rid of the text in footer. You will most likely
need the space for your justification.

\input{survey_buyins}

\input{HPC_resources}


\end{document}


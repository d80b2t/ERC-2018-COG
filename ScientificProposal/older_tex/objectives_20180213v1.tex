\section{State-of-the-art and Objectives}
\noindent
The objectives for this proposal are: to access and combine several
new state-of-the-art large astronomical datasets; to kick start the
new field of variable extragalactic astrophysics; to create a holistic
theory of accretion disk physics and quasar feedback in galaxy
formation theory, and to discover brand new astronomical phenomena.
Details are given in Table 1.  We stress that the novel and
high-risk/high-gain combination of these new state-of-the-art large
datasets will by design {\it go significantly beyond the
state-of-the-art} and will allow substantial advances in the frontiers
of understanding astrophysical phenomena as well as discovering new
objects.


\subsection{Background}
\noindent
Quasars\footnote{ Historically, ``quasars'' and ``Active Galactic
Nuclei (AGN)'' have described different luminosity/classes of objects,
but here we use these terms interchangeably (with a preference for
quasar) in recognition of the fact that they both describe accreting
supermassive black holes \citep[e.g.][]{Haardt2016book}.}  are powered
by accretion of material onto supermassive black holes (SMBHs), via
accretion disks.  In the local Universe, there is a link between the
key properties of massive galaxies, such as bulge mass, and their
central supermassive black holes \citep[e.g., ][]{McLure_Dunlop2002,
HaringRix2004, Salviander2007, Greene2010, KormendyHo2013}. This has
led to the proposal that the supermassive black hole, when accreting,
has an influence on its host galaxy by the means of some regulatory
``feedback'' mechanism(s) \citep[e.g., ][]{Sijacki2007, Hopkins2008a,
AlexanderHickox2012, Fabian2012, KingPounds2015}. However, the details
of the physical processes involved in this `AGN/quasar feedback' are
still disputed and, moreover, direct observational evidence for quasar
feedback in the early universe is conspicuous by its absence
\citep[e.g., ][]{HeckmanBest2014, NaabOstriker2017}. Hence, a major
source of uncertainty in our current understanding of galaxy evolution
is how supermassive black holes influence, and potentially regulate,
their host galaxies \citep{Vogelsberger2013, Vogelsberger2014,
Schaye2015, Angles-Alcazar2013, Angles-Alcazar2017}.

\smallskip
\smallskip
\noindent
What is the main quasar triggering mechanism at the height of quasar
activity? What direct observational evidence in individual objects
links quasar activity to star formation?  Can we observe ``quasar
feedback'' in action, in situ, for the most luminous sources?  Such
unknowns about the co-evolution of black holes and their host galaxies
remain among the most fundamental unanswered questions in
extragalactic astronomy.

\smallskip 
\smallskip
\noindent
Furthermore, the details of the physical processes involved in the
quasar activity including how the SMBH directly couples and affects
its most local environment, i.e., the accretion disk, broad line
region and dusty torus, are still unknown at this point
\citep[e.g.,][]{Netzer2015, Padovani2017}.

\smallskip 
\smallskip
\noindent
Although it has long been established that quasars are powered by
accretion discs surrounding supermassive black holes, there have also
been long-standing issues. For example, the observed spectral energy
distributions (SEDs) of typical quasars
\citep[e.g.,][]{Koratkar_Blaes1999, Sirko_Goodman2003} differ markedly
from classical predictions \citep[][]{SS73, Pringle1981} with a
typical observed quasar SED flat in $\lambda F_{\lambda}$ over several
decades in wavelength \citep{Elvis1994, Richards2006b}.  Also, real
accretion disks seem to be cooler \cite[e.g., ][]{Lawrence2012} and
larger \cite[e.g.,][]{Pooley2007, Morgan2010, Morgan2012,
Mosquera2011} than the standard accretion disk model predictions.

\smallskip 
\smallskip
\noindent
However, even more troubling are new observations of {\it extreme
variability} in some objects (see next section) - factors of several
over a decade or so, including, crucially, at optical wavelengths, and
not just in the extreme UV or in X-rays. This has led to the ``Quasar
Viscosity Crisis'' \citep{Lawrence2018}.

\smallskip 
\smallskip
\noindent
{\bf As such, we are left in the embarrassing current situation of invoking
galaxy-wide ``quasar feedback'' in order to reconcile demographic
observations in cosmological-scale simulations, but where we currently
do not understand the physics of mechanism that is supposed to
initiate this necessary and vital energy transport.}


\subsubsection{Observational State-of-the-Art}
Here we present a concise overview of the observational
state-of-the-art in the brand new field of variable extragalactic
astrophysics, concentrating on quasar studies.

\smallskip
\smallskip
\noindent
\textbf{\textsc{A microscope for rapid Central Engines:}}
``Changing-look'' quasars \citep[CLQs; ][]{LaMassa2015,
Runnoe2016, Ruan2016, Runco2016, MacLeod2016, Yang2017} 
are defined to be luminous quasars which have a
dramatic appearance, or disappearance, of their broad emission-line
component on observed-frame month-to-year timescales.  CLQs are
important since they offer a direct observational probe into the
physical processes dictating the structure of the broad-line region
(BLR). These timescales can potentially be associated with the viscous
timescale (the drift time through the accretion disk), the light
crossing timescale (critical for reverberation mapping and disk
reprocessing) and the dynamical timescale of the BLR.  {\it CLQs are thus
an ideal laboratory for studying accretion physics, as the entire
system responds to a large change in ionizing flux on a human
timescale.}

\smallskip 
\smallskip
\noindent 
In \citet{MacLeod2016} I co-led the first systematic search for CLQs
based on photometry from SDSS and Pan-STARRS1, along with repeat
spectra from the SDSS/BOSS, and reported the discovery of 10
CLQs. This is a startling result since we now estimate
$\approx$10-15\% of bona fide quasars may exhibit `changing look'
behaviour on $\sim$10 year (rest-frame) timescales. However, plausible
time-scales for variable dust extinction are factors of $2-10$ too
long to explain the dimming and brightening in these sources.  Changes
in accretion rate are the currently favored explanation for CLQs, but
then the question of how the inner accretion disk couples to the BLR
immediately arises. Further investigation is thus warranted.

\begin{figure}[h]
  \begin{center}
    \hspace{-0.5cm}
    \includegraphics[height=6.25cm,width=17.2cm]
    {figures/J110057_LC_Spectra_20171024.pdf}
    \vspace{-10pt}
    \caption{\footnotesize 
      {\it (Left:)} The optical and infrared light-curve for
      J1100-0053; Note the fall in the infrared, whereas there is a
      decrease, but then recovery in the optical.  {\it (Right:)} Three
      epochs of spectra for J1100-0053.  The spectacular downturn in the
      blue for the 2010 spectrum indicates a dramatic change in the
      accretion disk.}
    \vspace{-16pt}
    \label{fig:J1100}
  \end{center}
\end{figure}

\smallskip
\smallskip
\noindent
\textbf{\textsc{New IR investigations into the CLQ Population:}}
Taking advantage of new optical imaging data from the Dark Energy
Camera Legacy Survey \href{http://legacysurvey.org/decamls/}{(DECaLS)}
and new IR light-curves from NEOWISE \citep{Meisner2017a,
Meisner2017b}, I have made further in-roads into understanding the CLQ
population. This includes identifying objects with rapidly changing IR
light-curves and also accretion disk changes, e.g. the $z=0.378$
quasar SDSS J110057-005304.4, see Figure~\ref{fig:J1100}. From
J1100-0053, my new model \citep{Ross2018} suggests a dramatic new
picture of the physics of the CLQs governed by processes at the
innermost stable circular orbit (ISCO) and the structure of the
innermost disk. Expanding these new observations in sample and
temporal size, in order to properly inform our theoretical
models is the next big challenge.

\smallskip
\smallskip
\noindent
{\bf In summary, as of the time of writing, the
observational state-of-the-art for extreme variable quasars is 44
objects, 11 of which I have either discovered or co-led the
discovery.}



\subsubsection{Theoretical State-of-the-Art}
Here we present a concise high-level overview of the theoretical
state-of-the-art and in particular focus on issues related to our
quasar studies.

\smallskip 
\smallskip
\noindent 
\textbf{\textsc{Contemporary Accretion Disk theory:}} 
The accretion disk scale is $\lesssim 10^{3}-10^{6}$ r$_{g}$, which is
$\approx$5$\times$$10^{-3}$ to 5 pc for a 10$^{8}$ M$_{\rm
BH}$. \citet{YuanNarayan2014} review, black hole accretion flows can
be divided into two broad classes: cold and hot. Cold accretion flows
consist of cool optically thick gas and are found at relatively high
mass accretion rates.  Hot accretion flows, are virially hot and
optically thin, and occur at lower mass accretion rates.  How a
accretion disk flow transitions between `cold' and `hot', e.g. as the
mass flowrate $\dot{m}$ changes, is not well understood, and is an
area of current activity.


\smallskip 
\smallskip
\noindent 
\textbf{\textsc{Contemporary Galaxy formation theory:}}
Contemporary cosmological magnetohydrodynamical galaxy formation simulations take into account a wide range of physical processes, use state-of-the-art numerical codes and take weeks to months to run on the largest supercomputers.  They are incredibily sophisticated apparatus and allow us to gain deep insight into the physical processes that drive galaxy formation, including the energy connected to an accreting central SMBH. \citet{NaabOstriker2017} present an up to date reivew of the major challenges for galaxy formation theory. 

\smallskip 
\smallskip
\noindent 
Current state-of-the-art cosmological simulations, for example, the EAGLE Project \citep{Schaye2015,
Crain2015} and the IllustrisTNG Project \citep{Pillepich2018} employ
and track 10s of billions resolution elements across 100s of
megaparsec-cubed volumes.  For EAGLE (e.g. their L100N1504
simulation), the fundamental units of dimensions mass (M), length (L)
and time (T, i.e. resolution) are $\sim~2~\times10^{5}$ for initial
baryonic particle mass, ``softening lengths'' of 0.35-0.7 pkpc; and
and time-steps sampling $\sim$1000 years ($\sim$10$^{6}$ time-steps
across the age of the Universe)\footnote{The times are spaced
logarithmically in the expansion factor $a$ such that $\Delta a =
0.005a$.}. For the new IllustrisTNG ``TNG100'' model one has
$1.4\times10^{6}$ for baryonic particle mass, softening lengths $\approx$0.2-1
pkpc, and $8\times10^{5} h^{-1}$ M$_{\odot}$ for the seed black hole
mass.  As such, these are extremely powerful for global galactic
properties, but these simulations cannot, and were never designed to, explicitly
address inner central engine physics.

\smallskip 
\smallskip
\noindent 
Further progress is made with the new high-resolution ``zoom-in''
galaxy simulations, e.g. Feedback In Realistic Environments
\citep[FIRE-2;][]{Wetzel2016, Hopkins2017} or MUFASA
\citep[][]{Dave2016}.  In FIRE-2 for example, \citet{Wetzel2016} run a
cosmological scale dark-matter-only simulation to redshift $z=0$. An
isolated DM halo is then selected, the particles are traced back to
very high, $z=100$ redshift and the `convex hull' is regenerated at
high resolution (embedded within the full lower-resolution volume).
The fiducial baryonic simulation contains dark matter, gas, and stars
within the zoom-in region, comprising 140 million total particles,
with $M_{\rm DM} =3.5\times10^{4} M_{\odot}$ and $M_{\rm gas,initial}
= 7070 M_{\odot}$.  The dark matter and stars have fixed gravitational
softening lengths of 20pc and 4pc, respectively.  In these zoom-ins,
the shortest time step achieved is 180 years.  As such, these
`zoom-in' simulations are impressive, but still not close enough to
resolving the scales, masses and cadences needed in order to
successful model e.g. the ``changing look'' quasars.

\smallskip 
\smallskip
\noindent 
However, what is remains very concerning is that even once the mass,
length and timescales are computationally accessible, {\it we
currently do not know what physical prescriptions should be directed
for the central black hole and quasar engines to follow.}

\smallskip
\smallskip
\noindent
For example and as described in detailed in \citet{Weinberger2017},
modelling AGNs in cosmological simulations poses several fundamental
challenges. The detailed physical mechanisms of both accretion on to
SMBHs, and the AGN-gas interaction are poorly understood
\citep{Hopkins_Quataert2010, Hopkins_Quataert2011,
Huarte-Espinosa2011, Gaibler2012, Angles-Alcazar2013, Gaspari2013,
Cielo2014, Costa2014, Angles-Alcazar2015, Emsellem2015,
CurtisSijacki2015, CurtisSijacki2016a, CurtisSijacki2016b,
Rosas-Guevara2015, Roos2015, Hopkins2016, Bieri2017,
Angles-Alcazar2017}. This makes it at present impossible to formulate
a `correct' treatment for simulations.  The long-time standard
physical mechanism of Bondi-Hoyle-Lyttleton accretion, i.e. that of
spherical accretion onto a compact object traveling through the
interstellar medium \citep{Hoyle_Lyttleton1939, Bondi_Hoyle1944,
Bondi1952} with the accretion rate given by $\dot {M}_{\rm Bondi} =
\pi G^{2 }M_{\rm BH}^{2} \, \rho / c_{s}^{3}$, {\it is known to
be a considerable oversimplification} \citep[e.g.,][]{Edgar2004}.
We need a new theory. And we need new observations to guide us here. 


\subsubsection{Upcoming Surveys, Instruments and Missions}
\noindent
\citep{Lawrence2016_ASPC} emphasize that variability studies hold
information on otherwise unresolvable regions in quasars. And
population studies of large samples likewise have been very productive
for our understanding of quasars. These two themes are coming together
in the idea of systematic variability studies of large samples and
{\it over the next 5 or so years} the field of observational
extragalactic astrophysics is poised for a fundamental and rapid
change.

\smallskip
\smallskip
\noindent
Starting in late 2019, a fleet of new telescopes, instruments and
missions are coming online over the next few years that will leap-frog
the quality and quantity of data we have available today. Over the
course of the next 5-6 years, surveys and missions including the fifth
incarnation of the Sloan Digital Sky Survey
(SDSS-V\footnote{\href{www.sdss.org/future/}{{\tt
www.sdss.org/future/}}}), the Large Synoptic Survey Telescope
(LSST\footnote{\href{lsst.org}{{\tt lsst.org}}}), the Dark Energy
Spectroscopic Instrument (DESI\footnote{\href{desi.lbl.gov}{{\tt
desi.lbl.gov}}}) survey, the 4-meter Multi-Object Spectroscopic
Telescope (4MOST\footnote{\href{4most.eu}{{\tt 4most.eu}}}) survey,
and the ESA {\it Euclid}
mission\footnote{\href{sci.esa.int/euclid/}{{\tt
sci.esa.int/euclid/}}}, will see first light. Even more imminent is
the launch of the {\it James Webb Space Telescope}
(JWST\footnote{\href{jwst.stsci.edu}{{\tt jwst.stsci.edu}}}).


\begin{framed}
 \begin{tcolorbox}
   \begin{center} 
     Overview of Facilities and Surveys related to this proposal
   \end{center}
 \end{tcolorbox}
  
\noindent
\textbf{\textsc{Imminent:}}

The {\bf Sloan Digital Sky Survey (SDSS):} An ongoing project,
currently in its fourth phase, SDSS-IV.  {\bf The P.I. was a leading
member of the SDSS-III: Baryon Oscillation Spectroscopic Survey (BOSS;
see Track Record and C.V.).} The fifth generation of Sloan Digital Sky
Surveys, SDSS-V will be an all-sky, multi-epoch spectroscopic survey,
yielding spectra of over 6 million objects during its lifetime. In
particular, the SDSS-V Black Hole Mapper (BHM) will focus on
long-term, time-domain studies of AGN, including direct measurement of
black hole masses and changing-look quasars, and on the optical
characterization of eROSITA X-ray sources. Data taking is due to start
in 2020. Access would be through a \euro184,100 `buy-in', which allows
access for the P.I. and one PDRA.  {\it Data Products: Repeat spectra
in the North and Southern Hemisphere for 500,000 bright QSOs.} \\

The {\bf Dark Energy Spectroscopic Instrument (DESI) Survey:} is a 5
year cosmology survey that will be conducted on the Mayall 4-meter
telescope at Kitt Peak National Observatory starting in 2019. It uses
the 5,000 fiber Dark Energy Spectroscopic Instrument and will obtain
optical spectra for $\approx$20 million galaxies and quasars.  {\bf
The P.I. helped write the original science case and proposal for DESI
\citep{Schlegel2011} but having left the U.S./LBNL, no longer has data
access rights.}  The DESI Survey starts in late 2019 and data access
is through a \euro200,100 `buy-in', which allows access for the
P.I. and two PDRAs.  {\it Data Products: Spectra of 1e6 quasars across
14,000 deg$^{2}$ of the Northern Sky.} \\

The {\bf Large Synoptic Survey Telescope (LSST)} project will conduct
a 10-year survey of the sky, imaging the full Southern Sky every 3
nights. The LSST survey is designed to address four science areas
(Understanding the Mysterious Dark Matter and Dark Energy; Hazardous
Asteroids and the Remote Solar System; The Transient Optical Sky; The
Formation and Structure of the Milky Way) and is an absolutely unique
facility as far as areal, temporal and wavelength coverage. The U.K.
is a member of LSST.  {\it Data Products: $ugrizY$ broadband optical
and near-infrared imaging for 20,000 deg$^2$.  Images the full
Southern Sky every 3 days.} \\

\textit{\textbf{Euclid}} is an ESA Medium Class mission to map the
geometry of the dark Universe.  It aims to understand why the
expansion of the Universe is accelerating and what the nature of the
source responsible for this acceleration (``dark energy'') is.  The
mission will investigate the distance-redshift relationship and the
evolution of cosmic structures by measuring shapes and redshifts of
galaxies and clusters of galaxies out to redshifts $\sim$2, or
equivalently to a look-back time of 10 billion years. {\it Euclid} will 
also discover a range of near-infrared (NIR) detected quasars,  
 {\it Euclid} is planned for launch in mid-2021.  {\it Data Products: Very broadband
optical and 3 filter near-infrared space-based imaging for 15,000
deg$^2$.} \\

The {\bf 4-metre Multi-Object Spectroscopic Telescope (4MOST):} is a
fibre-fed spectroscopic survey facility on the VISTA telescope with a
large enough field-of-view to survey a large fraction of the southern
sky. The facility will be able to simultaneously obtain spectra of
2,400 objects distributed over a field-of-view of 4 square degrees.
The initial Galactic and Extragalactic surveys will operate over a
five-year period delivering spectra for $\geq$25 million objects over
$\gtrsim$15,000 deg. 4MOST will commence science operations in early
2022. {\it Data Products: } 5MOST will operate continuously for an
initial five-year public survey delivering spectra for $\geq$25
million object over 15,000 deg$^{2}$.\\

The {\bf James Webb Space Telescope (JWST)} is a space telescope
developed in coordination among NASA, the European Space Agency, and
the Canadian Space Agency. It is scheduled to be launched in June
2019. The telescope will offer unprecedented resolution and
sensitivity from 0.6 to 27$\mu$m. JWST is a partnership between NASA,
ESA and the Canadian Space Agency.  In particular, ESA's contributions
to JWST include (but are not limited to) the NIRSpec instrument and
the Optical Bench Assembly of the MIRI instrument.  In return for
these contributions, ESA gains full partnership in JWST and secures
full access to the JWST observatory for astronomers from ESA Member
States on identical terms to those of today on the {\it Hubble Space
Telescope}. {\it Data Products: Revolutionary optical to mid-infrared
deep-field imaging and spectra.  Unique access to wavelengths
$\lambda>2\mu$m, inaccessible from the ground, ideal for high-$z$
quasar studies.} \\


The {\bf Extended Roentgen Survey with an Imaging Telescope Array
(eROSITA)} is the main instrument on the Spektr-RG mission, an
international high-energy astrophysics observatory.  Set to launch in
2019 with both high sensitivity and a large FOV, eROSITA will discover
as many new X-ray sources in its first twelve months as are known
today, after more than 50 years of X-ray astronomy.  SDSS-V will
provide optical spectroscopic measurements including identifications
and redshifts, of $\sim$400,000 eROSITA X-ray sources detected in the
first 1.5 years of the all sky survey.  In addition, SDSS-V's BHM will
characterize numerous serendipitous discoveries, extreme and rare
objects, transients, and other peculiar variables found in the eROSITA
survey \citep{Merloni2012}, and expand an optical+X-ray quasar sample
with implications for observational cosmological constraints
\citep[e.g.][]{Risaliti_Lusso2015}.\\

\underline{Notes:} 4MOST has full access to the full LSST
footprint. LSST will overlap half (7,500 deg$^2$) of the {\it Euclid}
footprint. Data access to eROSITA sources is via an MOU with  
SDSS-V. \\

\hrulefill 

\noindent
\textbf{\textsc{Ongoing:}} 

The {\bf Wide-field Infrared Survey Explorer (WISE)} is a NASA
infrared-wavelength astronomical space telescope launched in December
2009 and is still operation (as at the time of writing, in its
``NEOWISE-R'' mission phase). WISE performed an all-sky astronomical
survey with images at 3.4, 4.6, 12 and 22$\mu$m using a 40cm (16 in)
diameter infrared telescope in Earth orbit.  {\bf The P.I. is a world
expert in quasar identification using WISE \citep[e.g., ][]{Ross2012,
Ross2015, Timlin2016, Timlin2018} and exploiting mid-infrared light
curve data.} \\

The \textbf{ESA {\it Gaia}} mission is an ongoing mission to chart a
three-dimensional map of our Galaxy, the Milky Way, in the process
revealing the composition, formation and evolution of the Galaxy. Gaia
is providing unprecedented positional and radial velocity measurements
with the accuracies needed to produce a stereoscopic and kinematic
census of about $\sim$one billion stars in our Galaxy and throughout
the Local Group. This amounts to about 1 per cent of the Galactic
stellar population.
\end{framed}



\smallskip
\smallskip
\noindent
{\bf \emph{The timing for this proposal could not be better or more imperative.} 
The first of the data ``firehoses'' turns on in late 2019, with
the full datastream from our key sources fully online around 2022. 
As such, with two years to use existing datasets as testbeds, we 
have the time to ramp-up our efforts, while also being able to 
take advantage of the initial data releases of all these new projects.}
%% LSST:: Each patch of sky it images will be visited 1000 times during the survey,


\subsection{Objectives}
\smallskip
\smallskip
\noindent
The outstanding issues and novel investigations 
that are pertinent to this proposal are summarised in the Table below. 


\smallskip
\smallskip
\noindent
Our ERC Consolidator grant proposal will radically improve our
understanding of one of the two fundamental energy sources available
to galaxies; that of accretion onto the compact object in the central
engine. We will achieve this by leveraging several of the new,
large-scale surveys that are coming online in the next few years.  The
scope and remit of an ERC Consolidator grant will allow us to combine
these data products in a manner that will not only establish the new
state-of-the-art in extragalactic variable astrophysics, {\it it will
establish and kick start the new field of extragalactic variable
astrophysics itself}.  The P.I. is a world-leader in observational quasar
astrophysics, both in terms of survey work and individual object
study.  Our proposal takes astrophysics into the 2020s, going from
single objects samples, to surveys and samples of millions of objects
leveraging these multi-billion \euro/\pounds/\$ next generation
missions, telescopes and their subsequent datasets.

\smallskip
\smallskip
\noindent
%%%%%%%%%%%%%%%%%%%%%%%%%%%%%%%%%%%%%%%%%%%%%%%%%%%%%%%%%%%%%%%%%%%%%%%%%%%%%%%%
%%
%%  https://tex.stackexchange.com/questions/337820/mcq-long-table-using-tikz-tcolorbox-or-tabular
%%  https://tex.stackexchange.com/questions/283419/color-in-a-multirow-cell-with-extra-vertical-space/283454
%%  https://tex.stackexchange.com/questions/406033/how-to-fit-a-cell-of-a-table-to-a-figure-and-arrange-multiple-tables/406042
%% 
%% THIS (??)::
%%     https://texblog.org/2014/05/19/coloring-multi-row-tables-in-latex/
%%
%%
%%   https://www.inf.ethz.ch/personal/markusp/teaching/guides/guide-tables.pdf
%%
%%%%%%%%%%%%%%%%%%%%%%%%%%%%%%%%%%%%%%%%%%%%%%%%%%%%%%%%%%%%%%%%%%%%%%%%%%%%%%%%


\begin{tcolorbox}[tab1, tabularx={X  X }, title=Outstanding Issues in Extragalactic Astrophysics, boxrule=1.25pt] 
Key issue                                                                            &  Novel investigation       \\ 
\hline \hline
\multicolumn{2}{c}{{\sc The physics of accretion}} \\ 
Investigating ``hot'' and ``cold'' mode accretion in the quasar population; 
determining the rates and timescales, and characterising the Changing Look Quasar (CLQ) population.   &     
Identifying and characterizing  all the CLQs in DESI and SDSS-V.  \\ 
\hline
Probing the inner parsec of the quasar central engine & 
Rapid analysis and response on LSST quasar light curves. \\ 
\hline
%%
\multicolumn{2}{c}{{\sc Obscured accretion and galaxy formation}} \\
Establish the relative importance of major mergers, minor mergers, cold streams and secular evolution 
have towards the growth of SMBHs across cosmic time. & 
Deep imaging data from LSST combined with searching for post-starburst signatures 
in DESI, SDSS-V, 4MOST spectra. NIRcam and MIRI imaging from JWST. \\ \hline
Establishing the bolometric output and origin of IR emission, and  
determine presence of extreme outflows in the $z\sim2-3$ quasar population. & 
MIRI MRS spectroscopy with JWST.\\ \hline
Establishing the range of SED parameter space the quasars occupy by a multi-wavelength multi-epoch ``truth table dataset'' & 
Building ``The Stripe 82 Rosetta Stone'' (SpIES, SHELA, VICS82, S82X, HSC; repeat optical observations from SDSS, DES) \\ \hline
%%
Find the physical conditions under which SMBH grew at the epoch when most of the accretion and star formation in the Universe occurred ($z\sim1-4$) & Perform a complete census of AGN across $z\sim0-7$, focussing on $z=1-4$ using medium-deep multiwavelength datasets \\ \hline
\multicolumn{2}{c}{{\sc Galaxy-scale feedback}}\\
Establishing the theoretical impact of extreme outflows in the $z\sim2-3$ quasar population & 
Hydro simulation modelling.  \\
\hline
Understand how the accretion disks around black holes launch winds and outflows and determine how much energy these carry. 
Quantify the amount of ``Maintenance/Jet/Kinetic'' mode and ``Transition/Radiative/Wind'' mode feedback.
& 
Identifying and characterizing  all the CLQs in DESI and SDSS-V.  \\ 
    %\end{tcbitemize}
\end{tcolorbox}



\smallskip
\smallskip
\noindent
\textbf{\textsc{Maximising Science Returns from European priorities:}}
Contemporary astronomy is a multi-national endeavor with many leading
facilites being international collaborations. Although a project, with
similar but much less ambitious science goals and return could be
envisaged at the national level, the full discovery and break-through
nature being described herein only comes to the fore when the data
from the various international collaborations are combined
intelligently.  Critically data from leading European Southern
Observatory (ESO) and European Space Agency (ESA) facilites will play
a pivotal role here.

% At the end of this section you can include some sub headings on: \\

%Research Vision and aims
%Justification of why your vision and aims are important
%Where will your field be at the end of the funding period in terms of new knowledge? If you can explain this in some sort of schematic diagramme it would be even better.


\smallskip


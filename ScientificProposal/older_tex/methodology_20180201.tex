%%%%%%%%%%%%%%%%%%%%%%%%%%%%%%%%%%%%%%%%%%%%%%%%%%%%%%%%%%%%%%%%%%%%%%%%%%%%%%%
%%%%%%%%%%%%%%%%%%%%%%%%%%%%%%%%%%%%%%%%%%%%%%%%%%%%%%%%%%%%%%%%%%%%%%%%%%%%%%%
%%
%%
%%             b.   M  E  T  H  O  D  O  L  O  G  Y  
%%
%%
%%%%%%%%%%%%%%%%%%%%%%%%%%%%%%%%%%%%%%%%%%%%%%%%%%%%%%%%%%%%%%%%%%%%%%%%%%%%%%%
%%%%%%%%%%%%%%%%%%%%%%%%%%%%%%%%%%%%%%%%%%%%%%%%%%%%%%%%%%%%%%%%%%%%%%%%%%%%%%%
\section*{b. Methodology}
\noindent
Describe your work plan in detail. You can separate the section in terms of work packages/case studies or describe the work in terms of the aims/objectives you described in the previous section, and how you will accomplish those. If you like you can have a flow chart of the different WP’s and how the whole project will come together. Each WP or aims/objectives can be broken down into subtasks.


\begin{itemize}
\item If you have any preliminary results in relation to the work you are describing make sure you emphasise it. Maybe dedicate a separate subsection/heading to preliminary results?
\item You can also list the milestones and the ground breaking features of the planned work. Also how will this WP/aim advance the state of the art in your field? This way even the non-expert evaluator will be able to see the big picture of what you are proposing to do.
\item Describe the balance of your project between the high risk /high gain experiments and what will be the long term benefits in your area from the results you will generate. How will you be advancing the field? In other words emphasise the impact of your work.
\item Have a dedicated section on feasibility of what you are proposing. Explain which WP’s/tasks present high levels of risk. Provide a contingency plan, particularly if any of the tasks are unconventional, present a great challenge and are high risk (but also high gain). Mention your experience and knowledge to hedge against this risk or alternative approaches or help from collaborators. Be “safely adventurous”.
\item Include a gantt chart or a timeline for the evaluators to visualise the timescale of each component of the work you are proposing.
\item Include a 4-5 line summary to recap and remind the evaluator what the essence of the project is and why it so important to get this funded now.
\end{itemize}

\smallskip
\smallskip
\noindent
My future research builds on, and expands my current research program;
I have a bold research vision that is designed to be addressed by a
research group, and the environment, current research areas and
telescope access in the Department of Physics and Astronomy at Dartmouth College
are ideal to
carry out these investigations.
%%
The science questions we seek to address are well-posed, yet strike at
the heart of major and still open extragalactic astrophysical
questions: Do we have a full accounting for the accretion history in
the Universe?  How does the energy `escape' from the central engine to
the host galaxy?  Are the modes of AGN ``feedback'' that regulate a
galaxy the same that regulate the AGN itself?  What are the
star-formation properties of mid-infrared luminous quasars at the peak
of quasar activity?  What are the evolutionary properties, if any, of
dark energy?

\smallskip
\smallskip
\noindent
Our proposal contains eight work packages that fall into three broad,
complementary categories: observational studies of large numbers
(millons) of objects; high-risk, very high-reward observational
studies of a small number (10s) of objects; theoretical modeling
investigations. Risks and mitigation strategies are present for each
WP.


\smallskip
\smallskip
\noindent
\textbf{\textsc{WP1: Build an Event Broker:}} 
The LSST will deliver three levels of data products and
services. ``Level 1'' are the nightly data products and their primary
purpose is to enable rapid follow-up of time-domain events. ``Level
2'' data products are annual and will include well calibrated images
and catalogues. ``Level 3'' are the user-created data product services
that will enable science cases that greatly benefit from co-location
of user processing and/or data within the LSST Archive Center. In
order to access the LSST data for our science needs we will need to
build an {\it event broker}, an intermediary program module that
interacts with primarily the ``Level 3'' data products from the LSST.

\noindent
{\bf WP1 is low-risk, high-reward.} 
The goal of this WP is to build an Event Broker.  The heavy-industry
computing infrastructure is already being supplied by the LSST Data
Access Center and LSST Corporation. Our task will be to build the
event broker in a timely and robust manner. At the effort level of one
PDRA (``PDRA 1'') and a substatioal commitment from the P.I., (NPR)
along with the key personel and algorithm resources at the Royal
Observatory, Edinburgh, there is no element of this which can be
deemed high-risk.


\smallskip
\smallskip
\noindent
\textbf{\textsc{WP2: Quasar Catalogue Generation:}} 
Building the quasar corpus and cataloguing the observational data will
be a large, but vital step in beginning to pursue our science
goals. This catalogue will be the glue that binds the observational
projects together and will have not only the data, but moreover the
metadata to enable the other WPs.

\noindent
{\bf WP2 is low-risk, high-reward.}
The goal of this WP is to construct a quasar catalogue.
This is a full WP, and with the P.I.s (NPR) experience at this
specific task, plus the effort level of one PDRA (``PDRA 2'') there is
no element of this which can be deemed high-risk.


\smallskip
\smallskip
\noindent
\textbf{\textsc{WP3: Light-Curve and Spectral Analyses:}} 
Following on from the quasar corpus catalogue generation, one key
science output will be the full and detailed light-curve and spectral
analyses of the said catalogue. This will result in the discovery of
light-curve trends with quasar type, new methods to measure black hole
mass and the relatively easy check to see which quasars have become
``changing-look'' objects. This WP will also have a data science/machine learning 
aspect.

\smallskip
\smallskip
\noindent
\textbf{\textsc{New IR investigations into the CLQ Population:}}
Taking advantage of new optical imaging data from the Dark Energy
Camera Legacy Survey \href{http://legacysurvey.org/decamls/}{(DECaLS)}
and new IR light-curves from NEOWISE ([26, 27]), we have made further
in-roads into understanding the CLQ population. This includes
identifying objects with rapidly changing IR light-curves and also
accretion disk changes, e.g. the $z=0.378$ quasar SDSS
J110057-005304.4, see Figure~\ref{fig:J110057}. From J110057, my new
model ([28]) suggests a dramatic new picture of the physics of the
CLQs governed by processes at the innermost stable circular orbit
(ISCO) and the structure of the innermost disk. {\it We have embarked
on a new observation campaign, gaining optical light-curves (from the
Liverpool Telescope) and spectra (from WHT and Palomar) to test this
startling new hypothesis.}


\noindent
{\bf WP3 is low-risk, high-reward.} 
The goal of this WP is to elucidate the physical processes that drive quasar variability.
The full Light-Curve and Spectral
Analyses that we envisaged will be a significant amount of work,
leading to a signifcant high-reward science. This particular work
package will be broken down into small projects and the level of two
PDRAs (``PDRA 1'' and ``PDRA 2''), as well as the P.I. (NPR) and a PhD
student (``PhD 1'') will be directed towards this. There is no element
of this which can be deemed high-risk.


\smallskip
\smallskip
\noindent
\textbf{\textsc{WP4: Quasar Demographic studies:}} 
Another major scientific output that will originate from the quasar
corpus catalogue generation will be the study of the Quasar
Demographics from our datastreams. This is different from WP3 in that
these investigations wont necessarily be tied to the time-domain
aspect of our catalogue, but will be the crucial baseline that we, and
the field in general, will use to compare to the time-depedent
discoveries. Luminosity function, clustering and higher-order
statistics will be made in order to precisely determine the census of
AGN, their environments, their host galaxy preferences and their
evolution. All these are vital observational tests for galaxy
formation models and theory (see WP6).

\noindent
{\bf WP4 is low-risk, high-reward.}
The goal of this WP is to make the key observational tests that have to be explained by any 
viable galaxy formation theory. 
Similar to WP3, the analyses that we envisaged will be broken down
into small projects and the level of two PDRAs (``PDRA 1'' and ``PDRA
2''), as well as the P.I. (NPR) and a PhD student (``PhD 1'') will be
directed towards this. There is no element of this which can be deemed
high-risk.


\smallskip
\smallskip
\noindent
\textbf{\textsc{WP5: Accretion Disk Simulation:}} 
New accretion models are needed to fully explain the observational
data of ``changing look'' quasars that we have examples of today (see
e.g. Ross et al. 2018). New radiation MHD codes begin to explain the
observations here, but further development is needed to gain the
desired deep understanding. 

\noindent
{\bf WP5 is lower-risk, high-reward.}
The goal of WP5 is to develop new accretion disk simulations that
explain our observational results.  This will be the lead WP for one
PDRA (``PDRA 3'') and a low level of the P.I.'s (NPR) time. We
classify it not as fully `low-risk', since we envisage some ramp-up
time to get our theoretical simulations to the level that will match
our beyond-the-state-of-the-art dataset. However, we mitigate this risk
by invoking the collaboration with an accretion disk theorist
Prof. Ken Rice (WKMR) who is the Personal Chair of Computational
Astrophysics in the School of Physics and Astronomy here at the
University of Edinburgh.


\smallskip
\smallskip
\noindent
\textbf{\textsc{WP6: AGN Feedback Simulations:}} 
Cosmological-scale hydrodynamic simulations are now coming online. 
While we do not week to lead or generate new versions of these, we do 
envisaged using their outputs in order to `benchmark' our observational 
demographic work. 

\noindent
{\bf WP6 is low-risk, high-reward.}
All the data from these simulations is already in place today, though no one 
has embarked on doing any of the `heavy-lifting' and comparisons we will 
have the observational results for. Professor RS Dave (RSD) who is a Chair of Physics 
in the Institute for Astronomy will be a key collaborator here. 


\smallskip
\smallskip
\noindent
\textbf{\textsc{WP7: Observations of Quasars by the James Webb Space Telescope:}} 
What are the star-formation properties of mid-infrared luminous quasars at the peak of quasar activity? 
We aim to answer this by looking for the presence of polycyclic aromatic hydrocarbon (PAH) spectral features 
in $z \approx 2.5$ infrared bright quasars. 

\smallskip
\smallskip
\noindent
%\textbf{\textsc{Extremely Red Quasars: Feedback in action at high-$z$:}}
\textbf{\textsc{Extremely Red Quasars:}}
In [16] I discovered a new class of object, the ``extremely red
quasars'', that have optical spectroscopy from SDSS/BOSS, and
$r-[22\mu{\rm m}]>14$ colors (i.e., $F_{\nu,\, {\rm MIR}} / F_{\nu,\,
{\rm opt}} \gtrsim 1000$) from the Wide-field Infrared Survey Explorer
(WISE; [17]) satellite, see Figure~\ref{fig:ERQ}.  The ERQs are a
unique obscured quasar population with extreme physical conditions
related to powerful outflows across the line-forming regions. These
sources are the signposts of the most dramatic form of quasar feedback
at the peak epoch of galaxy formation, and may represent an active
``blow-out'' phase of quasar evolution ([18], [19]).  However, due to
the current lack of access to mid-infrared spectroscopy, it is still
unknown whether the large IR luminosities observed in these quasars is
from star formation, which would produce strong polycyclic aromatic
hydrocarbon (PAH) spectral features, or, if it is from the hot dust
near the central quasar, which should produce much weaker/no PAH
emission.

\noindent
{\bf WP4 is medium-to-high risk, high-reward.}
This is an ideal investigation for the James Webb Space Telescope, but we classify this as `high-risk' since this is the one telescope/survey/mission where we would have to bid/apply for the telescope time and are not guaranteed the data. We mitigate the risk here by saying that this will be the one project the P.I. (NPR) would directly lead, and would lead to very-high gain science, but does not impact in any direct way any of the other WPs. 



\smallskip
\smallskip
\noindent
\textbf{\textsc{WP8: New Object Discovery:}} 
The LSST will scan the sky repeatedly, enabling it, and us, to both
discover new, distant transient events and to study variable objects
throughout our universe. The LSST will extend our view of the
changeable universe a thousand times over current surveys.  The most
interesting science to come may well be the discovery of new classes
of objects.

\noindent
{\bf WP8 is medium-risk, exceptionally high-reward.}
We class this as medium-risk, since it is tricky to class a WP with essentially unknown discovery potential as fully `low-risk'. However, we do not classify this as `high-risk' since if there was a paucity of discovery of novel classes of objects, this would be the first time in the hitorsy of observational astrophysics that a new facility such as LSST has come online and found nothing new. 


\smallskip
\smallskip
\noindent
\textbf{\textsc{{Data Science and Observational Astrophysics:}}}
Data science is a new interdisciplinary field of scientific methods to
extract knowledge or insights from data in various forms, either
structured or unstructured. It employs techniques and theories drawn
from many fields within the broad areas of mathematics, statistics,
information science, and computer science, in particular from the
subdomains of machine learning, classification, cluster analysis, data
mining, databases, and visualization.  {\it Modern day observational
astrophysicists are in all but name data scientists, and as such, this
proposal is inherently interdisciplinary.}




\smallskip
\smallskip
\noindent
\textbf{\textsc{Breaking Down The Data Silos}}
to using advanced data analysis is not skill base or technology; it is
simply access to the data.  A data silo is a repository of fixed data
that remains under the control of one department/collaboration and is
isolated from the rest of the world, much like grain in a farm silo is
closed off from outside elements. These silos are isolated islands of
data, and they make it prohibitive to extract data and put it to other
uses. They can arise for multiple reasons. In commercial enterprises,
data remained siloed for monetary gain.  However, in research
environments, and {\it especially in contemporary observational
astrophysics}, the data silos are open, but due to the lack of raw
person-power, still remain uncombined. {\it The combination 
of P.I. and host institute means we are uniquely positioned to 
break down these astro-data silos for massively significant 
science gain.}


\smallskip
\smallskip
\noindent
\textbf{\textsc{Algorithms}}
Our algorithms and methodology is based on the latest machine-learning and data science techniques. 
This includes the ``extreme deconvolution'' \href{http://www.sdss.org/dr14/data\_access/value-added-catalogs/?vac\_id=xdqso/}{`XDQSO' technique}\footnote{\href{https://github.com/xdqso/xdqso}{\tt github.com/xdqso/xdqso}}.
%%
\href{http://ogrisel.github.io/scikit-learn.org/sklearn-tutorial/index.html}{\tt
scikit-learn} is a Python module integrating classic machine learning
algorithms in the scientific Python world (numpy, scipy,
matplotlib). It aims to provide simple and efficient solutions to
learning problems, accessible to everybody and reusable in various
contexts.  \href{https://github.com/astroML/sklearn\_tutorial}{{\tt
github.com/astroML/sklearn\_tutorial}} and \href{https://github.com/jakevdp/PythonDataScienceHandbook}{{\tt github.com/jakevdp/PythonDataScienceHandbook}} have full details.


\smallskip
\smallskip
\noindent
\textbf{\textsc{Open Innovation, Open Science, Open to the World:}} 
The P.I. is an exceptionally strong, longtime and vocal supporter of ``Open Access''. 
All my codes, data\footnote{where I am not breaking current data access agreements}, papers 
and proposals can be found at \href{github.com/d80b2t}{{\tt github.com/d80b2t}}. 
Indeed, this proposal itself is now at that location. 

\smallskip
\smallskip
\noindent
One of the major research outputs of this ERC will be computer code.
As such, we are already working with the \href{The Software
Sustainability Institute}{\tt https://www.software.ac.uk/} which was
founded to support the UK’s research software community.  Our software
well be developed using the FAIR ideology (Findable, Accessible,
Interoperable, Reusable \footnote{Wilkinson, MD, Sci Data. 2016 Mar
15;3:160018. doi: 10.1038/sdata.2016.18.}  )  and will be delivered in
a manner which is fully inline with ``Open Innovation, Open Science,
Open to the World''.

\smallskip
\smallskip
\noindent
{\it The timing for this proposal could not be better or more imperative.} 
The first of the data ``firehoses'' turns on in late 2019, with
the full datastream from our key sources fully online around 2022. 
As such, with two years to use existing datasets as testbeds, we 
have the time to ramp-up our efforts, while also being able to 
take advantage of the initial data releases of all these new projects. 
%% LSST:: Each patch of sky it images will be visited 1000 times during the survey,




\smallskip
\smallskip
\noindent
\textbf{\textsc{Quasars, Dark Energy and Very Wide Field Surveys: }}
In the community's future is the prospect of very wide-field
ground-based surveys using both imaging and spectroscopy, in the North
and Southern Hemisphere. Building on {\it (i)} my leadership
experience and heritage from being an integral part of BOSS and {\it
(ii)} my world-leading expertise in quasar target selection,
demographics, and physical properties, I will continue to lead the
scientific development of using quasars as large-scale structure (LSS)
tracers in these new surveys.

\smallskip \smallskip
\noindent 
%{\bf \underline {Outline of Future Research:}}
Prior to SDSS-III BOSS, quasars lagged behind massive galaxies as good
tracers of LSS. However, with the evolution of Baryon Acoustic
Oscillation (BAO) and Redshift-Space Distortion (RSD) studies using
BOSS, and in particular the Lyman-$\alpha$ forest (Ly$\alpha$F),
quasars are now seen as key objects in accessing the high-$z$
Universe. Moreover, in the DESI era, there is huge potential to extend
this reach further, first for BAO/RSD and also as Standard Candles. 


\begin{itemize}
\item{{\bf Quasars as Cosmological Probes {\sc I:} Baryon Acoustic
      Oscillations.} I co-led the first investigation that successfully used
    quasars as point test particles to measure the BAO signature
    [15]. This measurement was a cross-correlation with the Ly$\alpha$F,
    but demonstrated that quasars themselves, despite their lower number
    density can trace LSS sufficiently well for BAO studies. This
    provides access to geometry measurements at $z>1$, further
    constraining the Hubble Parameter at high-$z$ and testing the current
    $\Lambda$CDM model.  {\it DESI will be ``the ultimate quasar survey'',
      with sub-per cent measurements of the distance scale (from BAO alone)
      to redshifts of $z\sim3$ using `tracer' and Ly$\alpha$F quasars.}}
  
\item{{\bf Quasars as Cosmological Probes {\sc II:} RSDs with
      Quasars.}  As I proved with the original SDSS sample ([31]), quasars
    can also be used to measure RSDs.  Measurements of the normalized
    growth rate, $f\sigma_{8}$, from RSD using quasars at $z\sim1.5$ is
    crucial since model predictions diverge at these redshifts for
    $\Lambda$CDM+G.R. compared to $f(R)$ gravity, DGP braneworld, and varying
    Gravitational Constant models ([32, 33]). {\it Even with the local
      GW170817 measurements, a range of gravity theories persist and
      testing General Relativity at cosmological scales remains imperative.}}
  
\item{{\bf Quasars as Cosmological Probes {\sc III:} Reverberation
      Mapping Campaigns.}  Recently, [34-37] demonstrated that by using the
    tight relationship between the luminosity of a quasar and the radius
    of its broad-line region, established via reverberation mapping, one
    is able to determine the luminosity distances to quasars.  This means
    that quasars can now be used as standard, or more accurately,
    ``standardizable'' candles (in a very similar way to Type Ia
    supernovae). {\it Thus with light-curve data from LSST and a baseline
      for repeat spectroscopy from SDSS, DESI is ideally placed to exploit
      this new method, with very different systematics to the BAO, and
      access a redshift regime $z>4$ where even the Ly$\alpha$F will not be
      able to offer cosmology constraints.}}

\end{itemize}




\begin{figure}[h]
  \begin{center}
   \hspace{-0.5cm}
%   trim=l b r t
    \includegraphics[height=5.5cm,width=18.0cm] %, trim={0.05cm 0 0.05cm 0},clip]
   {figures/WISE_SDSSzoomHSC_ERQ-image_v3.pdf}
    \vspace{-10pt}
   \caption{%\small   
\footnotesize 
%     \scriptsize
 %    \tiny
The IR and optical imaging of J2323-0100, an archetype of the
``Extremely Red Quasars'' (ERQs) at $z\approx2.5$ and a {\it JWST}
target. Shown are WISE {\it (left)}, where the quasar booms out as
indicated by the arrow; the SDSS image {\it (middle left)} with
zoom-in {\it (middle right)} on the optically faint source, and new
HSC imaging {\it (right)}, which shows tantalizing evidence for a
faint companion galaxy. Optical rest-frame spectra of J2323-0100,
revealed very broad (FWHM = 2500-5000 km s$^{-1}$), strongly
blue-shifted (by up to 1500 km s$^{-1}$) \oiii\ $\lambda$5007\AA\
emission lines in the ERQs. This is suggestive of active outflows and
potentially evidence for AGN feedback in action at the height of SMBH
activity.
}
  \vspace{-12pt}
 \label{fig:ERQ}
\end{center}
\end{figure}


\smallskip
\smallskip
\noindent
The P.I. has become a world-leader in the field of extragalactic
quasar observational astrophysics.  Moreover, the University of
Edinburgh is now poised to be an astronomical data centre nexus, with
access to the two largest datasets in our proposal; LSST and {\it
Euclid}.  The P.I. has built their career on this science case, and
has already been a P.I.  of a science group (as part of a
collaboration) with prodigious scientific output (400 published,
peer-reviewed papers and counting).

\smallskip
\smallskip
\noindent
%\section*{\textcolor{Cerulean}{4. Feasibility, Projects and Methodology}}
\textbf{\textsc{P.I.'s Experience and Track Record:}}
The P.I. has an established track record of managing science teams and groups, e.g., 
\begin{itemize}
  \item The P.I. was the Chair
     of the SDSS-III:BOSS Quasar Science Working Group, managing a
     group of senior professors, other postdocs and graduate students.
     The scientific yield from the BOSS Quasar Survey was extremely high 
     with \href{https://tinyurl.com/ycxd8lb6}{over 400 journal publications} having 
     used the BOSS Quasar catalogs and datasets.
 
    \item The P.I. spent a considerable amount of time in 2017 working as the
     Chief Data Scientist for a San Francisco Bay Area 
     tech start-up\footnote{Due to legal immigration issues, this venture is no longer being pursued.}. 

     \item The P.I. is an STFC Ernest Rutherford Fellow, with a budget of \euro615,000 on award. 
\end{itemize}


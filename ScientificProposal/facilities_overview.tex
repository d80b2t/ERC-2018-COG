
\begin{framed}
 \begin{tcolorbox}
   \begin{center} 
     Overview of Facilities and Surveys related to this proposal
   \end{center}
 \end{tcolorbox}
  
\noindent
\textbf{\textsc{Imminent:}}

The {\bf Sloan Digital Sky Survey (SDSS):} An ongoing project,
currently in its fourth phase, SDSS-IV.  {\bf The P.I. was a leading
member of the SDSS-III: Baryon Oscillation Spectroscopic Survey (BOSS;
see Track Record and C.V.).} The fifth generation of Sloan Digital Sky
Surveys, SDSS-V will be an all-sky, multi-epoch spectroscopic survey,
yielding spectra of over 6 million objects during its lifetime. In
particular, the SDSS-V Black Hole Mapper (BHM) will focus on
long-term, time-domain studies of AGN, including direct measurement of
black hole masses and changing-look quasars, and on the optical
characterization of eROSITA X-ray sources. Data taking is due to start
in 2020. Access would be through a \euro184,100 `buy-in', which allows
access for the P.I. and one PDRA.  {\it Data Products: Repeat spectra
in the North and Southern Hemisphere for 500,000 bright QSOs.} \\

The {\bf Dark Energy Spectroscopic Instrument (DESI) Survey:} is a 5
year cosmology survey that will be conducted on the Mayall 4-meter
telescope at Kitt Peak National Observatory starting in 2019. It uses
the 5,000 fiber Dark Energy Spectroscopic Instrument and will obtain
optical spectra for $\approx$20 million galaxies and quasars.  {\bf
The P.I. helped write the original science case and proposal for DESI
\citep{Schlegel2011} but having left the U.S./LBNL, no longer has data
access rights.}  The DESI Survey starts in late 2019 and data access
is through a \euro200,100 `buy-in', which allows access for the
P.I. and two PDRAs.  {\it Data Products: Spectra of 1e6 quasars across
14,000 deg$^{2}$ of the Northern Sky.} \\

The {\bf Large Synoptic Survey Telescope (LSST)} project will conduct
a 10-year survey of the sky, imaging the full Southern Sky every 3
nights. The LSST survey is designed to address four science areas
(Understanding the Mysterious Dark Matter and Dark Energy; Hazardous
Asteroids and the Remote Solar System; The Transient Optical Sky; The
Formation and Structure of the Milky Way) and is an absolutely unique
facility as far as areal, temporal and wavelength coverage. The U.K.
is a member of LSST.  {\it Data Products: $ugrizY$ broadband optical
and near-infrared imaging for 20,000 deg$^2$.  Images the full
Southern Sky every 3 days.} \\

\textit{\textbf{Euclid}} is an ESA Medium Class mission to map the
geometry of the dark Universe.  It aims to understand why the
expansion of the Universe is accelerating and what the nature of the
source responsible for this acceleration (``dark energy'') is.  The
mission will investigate the distance-redshift relationship and the
evolution of cosmic structures by measuring shapes and redshifts of
galaxies and clusters of galaxies out to redshifts $\sim$2, or
equivalently to a look-back time of 10 billion years. {\it Euclid} will 
also discover a range of near-infrared (NIR) detected quasars,  
 {\it Euclid} is planned for launch in mid-2021.  {\it Data Products: Very broadband
optical and 3 filter near-infrared space-based imaging for 15,000
deg$^2$.} \\

The {\bf 4-metre Multi-Object Spectroscopic Telescope (4MOST):} is a
fibre-fed spectroscopic survey facility on the VISTA telescope with a
large enough field-of-view to survey a large fraction of the southern
sky. The facility will be able to simultaneously obtain spectra of
2,400 objects distributed over a field-of-view of 4 square degrees.
The initial Galactic and Extragalactic surveys will operate over a
five-year period delivering spectra for $\geq$25 million objects over
$\gtrsim$15,000 deg. 4MOST will commence science operations in early
2022. {\it Data Products: } 5MOST will operate continuously for an
initial five-year public survey delivering spectra for $\geq$25
million object over 15,000 deg$^{2}$.\\

The {\bf James Webb Space Telescope (JWST)} is a space telescope
developed in coordination among NASA, the European Space Agency, and
the Canadian Space Agency. It is scheduled to be launched in June
2019. The telescope will offer unprecedented resolution and
sensitivity from 0.6 to 27$\mu$m. JWST is a partnership between NASA,
ESA and the Canadian Space Agency.  In particular, ESA's contributions
to JWST include (but are not limited to) the NIRSpec instrument and
the Optical Bench Assembly of the MIRI instrument.  In return for
these contributions, ESA gains full partnership in JWST and secures
full access to the JWST observatory for astronomers from ESA Member
States on identical terms to those of today on the {\it Hubble Space
Telescope}. {\it Data Products: Revolutionary optical to mid-infrared
deep-field imaging and spectra.  Unique access to wavelengths
$\lambda>2\mu$m, inaccessible from the ground, ideal for high-$z$
quasar studies.} \\


The {\bf Extended Roentgen Survey with an Imaging Telescope Array
(eROSITA)} is the main instrument on the Spektr-RG mission, an
international high-energy astrophysics observatory.  Set to launch in
2019 with both high sensitivity and a large FOV, eROSITA will discover
as many new X-ray sources in its first twelve months as are known
today, after more than 50 years of X-ray astronomy.  SDSS-V will
provide optical spectroscopic measurements including identifications
and redshifts, of $\sim$400,000 eROSITA X-ray sources detected in the
first 1.5 years of the all sky survey.  In addition, SDSS-V's BHM will
characterize numerous serendipitous discoveries, extreme and rare
objects, transients, and other peculiar variables found in the eROSITA
survey \citep{Merloni2012}, and expand an optical+X-ray quasar sample
with implications for observational cosmological constraints
\citep[e.g.][]{Risaliti_Lusso2015}.\\

\underline{Notes:} 4MOST has full access to the full LSST
footprint. LSST will overlap half (7,500 deg$^2$) of the {\it Euclid}
footprint. Data access to eROSITA sources is via an MOU with  
SDSS-V. \\

\hrulefill 

\noindent
\textbf{\textsc{Ongoing:}} 

The {\bf Wide-field Infrared Survey Explorer (WISE)} is a NASA
infrared-wavelength astronomical space telescope launched in December
2009 and is still operation (as at the time of writing, in its
``NEOWISE-R'' mission phase). WISE performed an all-sky astronomical
survey with images at 3.4, 4.6, 12 and 22$\mu$m using a 40cm (16 in)
diameter infrared telescope in Earth orbit.  {\bf The P.I. is a world
expert in quasar identification using WISE \citep[e.g., ][]{Ross2012,
Ross2015, Timlin2016, Timlin2018} and exploiting mid-infrared light
curve data.} \\

The \textbf{ESA {\it Gaia}} mission is an ongoing mission to chart a
three-dimensional map of our Galaxy, the Milky Way, in the process
revealing the composition, formation and evolution of the Galaxy. Gaia
is providing unprecedented positional and radial velocity measurements
with the accuracies needed to produce a stereoscopic and kinematic
census of about $\sim$one billion stars in our Galaxy and throughout
the Local Group. This amounts to about 1 per cent of the Galactic
stellar population.
\end{framed}
